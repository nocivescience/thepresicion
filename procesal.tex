ESTABLECE CODIGO PROCESAL PENAL

    Teniendo presente que el H. Congreso Nacional ha dado su aprobación al siguiente

    Proyecto de ley:

    ''CODIGO PROCESAL PENAL


    Libro Primero

    Disposiciones generales

    Título I
    Principios básicos
    Artículo 1º.- Juicio previo y única persecución. Ninguna persona podrá ser condenada o penada, ni sometida a una de las medidas de seguridad establecidas en este Código, sino en virtud de una sentencia fundada, dictada por un tribunal imparcial. Toda persona tiene derecho a un juicio previo, oral y público, desarrollado en conformidad con las normas de este cuerpo legal.
    La persona condenada, absuelta o sobreseída definitivamente por sentencia ejecutoriada, no podrá ser sometida a un nuevo procedimiento penal por el mismo hecho.

    Artículo 2º.- Juez natural. Nadie podrá ser juzgado por comisiones especiales, sino por el tribunal que señalare la ley y que se hallare establecido por ésta con anterioridad a la perpetración del hecho.

    Artículo 3°.- Exclusividad de la investigación penal. El ministerio público dirigirá en forma exclusiva la investigación de los hechos constitutivos de delito, los que determinaren la participación punible y los que acreditaren la inocencia del imputado, en la forma prevista por la Constitución y la ley.

    Artículo 4º.- Presunción de inocencia del imputado. Ninguna persona será considerada culpable ni tratada como tal en tanto no fuere condenada por una sentencia firme.

    Artículo 5º.- Legalidad de las medidas privativas o restrictivas de libertad. No se podrá citar, arrestar, detener, someter a prisión preventiva ni aplicar cualquier otra forma de privación o restricción de libertad a ninguna persona, sino en los casos y en la forma señalados por la Constitución y las leyes.
    Las disposiciones de este Código que autorizan la restricción de la libertad o de otros derechos del imputado o del ejercicio de alguna de sus facultades serán interpretadas restrictivamente y no se podrán aplicar por analogía.


    Artículo 6º.- Protección de la víctima. El ministerio público estará obligado a velar por la protección de la víctima del delito en todas las etapas del procedimiento penal. Por su parte, el tribunal garantizará conforme a la ley la vigencia de sus derechos durante el procedimiento.
    El fiscal deberá promover durante el curso del procedimiento acuerdos patrimoniales, medidas cautelares u otros mecanismos que faciliten la reparación del daño causado a la víctima. Este deber no importará el ejercicio de las acciones civiles que pudieren corresponderle a la víctima.
    Asimismo, la policía y los demás organismos auxiliares deberán otorgarle un trato acorde con su condición de víctima, procurando facilitar al máximo su participación en los trámites en que debiere intervenir.


    Artículo 7º.- Calidad de imputado. Las facultades, derechos y garantías que la Constitución Política de la República, este Código y otras leyes reconocen al imputado, podrán hacerse valer por la persona a quien se atribuyere participación en un hecho punible desde la primera actuación del procedimiento dirigido en su contra y hasta la completa ejecución de la sentencia.
    Para este efecto, se entenderá por primera actuación del procedimiento cualquiera diligencia o gestión, sea de investigación, de carácter cautelar o de otra especie, que se realizare por o ante un tribunal con competencia en lo criminal, el ministerio público o la policía, en la que se atribuyere a una persona responsabilidad en un hecho punible.
    En las investigaciones iniciadas por el Ministerio Público, los funcionarios policiales o de Gendarmería de Chile, de las Fuerzas Armadas y los funcionarios de los servicios de su dependencia, en cumplimiento del deber, exclusivamente en el marco de funciones de resguardo del orden público, tales como las que se ejercen durante estados de excepción constitucional, protección de la infraestructura crítica, resguardo de fronteras, y funciones de policía cuando correspondan, o cuando se desempeñan en el marco de sus funciones fiscalizadoras, que se encuentren en el caso previsto en el párrafo tercero del numeral 6° del artículo 10 del Código Penal, serán considerados como víctimas o testigos, según corresponda, para todos los efectos legales, a menos que las diligencias permitan atribuirles participación punible. En este último caso adquirirán la calidad de imputado, y podrán hacer valer las facultades, derechos y garantías propias de éste.

    Artículo 8º.- Ámbito de la defensa. El imputado tendrá derecho a ser defendido por un letrado desde la primera actuación del procedimiento dirigido en su contra. Todo imputado que carezca de abogado tendrá derecho irrenunciable a que el Estado le proporcione uno. La designación del abogado la efectuará el juez antes de que tenga lugar la primera actuación judicial del procedimiento que requiera la presencia de dicho imputado.
    El imputado tendrá derecho a formular los planteamientos y alegaciones que considerare oportunos, así como a intervenir en todas las actuaciones judiciales y en las demás actuaciones del procedimiento, salvas las excepciones expresamente previstas en este Código.

    Artículo 9º.- Autorización judicial previa. Toda actuación del procedimiento que privare al imputado o a un tercero del ejercicio de los derechos que la Constitución asegura, o lo restringiere o perturbare, requerirá de autorización judicial previa.
    En consecuencia, cuando una diligencia de investigación pudiere producir alguno de tales efectos, el fiscal deberá solicitar previamente autorización al juez de garantía.
    Tratándose de casos urgentes, en que la inmediata autorización u orden judicial sea indispensable para el éxito de la diligencia, podrá ser solicitada y otorgada por cualquier medio idóneo al efecto, tales como teléfono, fax, correo electrónico u otro, sin perjuicio de la constancia posterior, en el registro correspondiente. No obstante lo anterior, en caso de una detención se deberá entregar por el funcionario policial que la practique una constancia de aquélla, con indicación del tribunal que la expidió, del delito que le sirve de fundamento y de la hora en que se emitió.

    Artículo 10.- Cautela de garantías. En cualquiera etapa del procedimiento en que el juez de garantía estimare que el imputado no está en condiciones de ejercer los derechos que le otorgan las garantías judiciales consagradas en la Constitución Política, en las leyes o en los tratados internacionales ratificados por Chile y que se encuentren vigentes, adoptará, de oficio o a petición de parte, las medidas necesarias para permitir dicho ejercicio.
    Si esas medidas no fueren suficientes para evitar que pudiere producirse una afectación sustancial de los derechos del imputado, el juez ordenará la suspensión del procedimiento por el menor tiempo posible y citará a los intervinientes a una audiencia que se celebrará con los que asistan. Con el mérito de los antecedentes reunidos y de lo que en dicha audiencia se expusiere, resolverá la continuación del procedimiento o decretará el sobreseimiento temporal del mismo.
    Con todo, no podrá entenderse que existe afectación sustancial de los derechos del imputado cuando se acredite, por el Ministerio Público o el abogado querellante, que la suspensión del procedimiento solicitada por el imputado o su abogado sólo persigue dilatar el proceso.

    Artículo 11.- Aplicación temporal de la ley procesal penal. Las leyes procesales penales serán aplicables a los procedimientos ya iniciados, salvo cuando, a juicio del tribunal, la ley anterior contuviere disposiciones más favorables al imputado.

    Artículo 12.- Intervinientes. Para los efectos regulados en este Código, se considerará intervinientes en el procedimiento al fiscal, al imputado, al defensor, a la víctima y al querellante, desde que realizaren cualquier actuación procesal o desde el momento en que la ley les permitiere ejercer facultades determinadas.
    Artículo 13.- Efecto en Chile de las sentencias penales de tribunales extranjeros. Tendrán valor en Chile las sentencias penales extranjeras. En consecuencia, nadie podrá ser juzgado ni sancionado por un delito por el cual hubiere sido ya condenado o absuelto por una sentencia firme de acuerdo a la ley y al procedimiento de un país extranjero, a menos que el juzgamiento en dicho país hubiere obedecido al propósito de sustraer al individuo de su responsabilidad penal por delitos de competencia de los tribunales nacionales o, cuando el imputado lo solicitare expresamente, si el proceso respectivo no hubiere sido instruido de conformidad con las garantías de un debido proceso o lo hubiere sido en términos que revelaren falta de intención de juzgarle seriamente.
    En tales casos, la pena que el sujeto hubiere cumplido en el país extranjero se le imputará a la que debiere cumplir en Chile, si también resultare condenado.
    La ejecución de las sentencias penales extranjeras se sujetará a lo que dispusieren los tratados internacionales ratificados por Chile y que se encontraren vigentes.
    Título II
    Actividad procesal

    Párrafo 1º Plazos
    Artículo 14.- Días y horas hábiles. Todos los días y horas serán hábiles para las actuaciones del procedimiento penal y no se suspenderán los plazos por la interposición de días feriados.
    No obstante, cuando un plazo de días concedido a los intervinientes venciere en día feriado, se considerará ampliado hasta las veinticuatro horas del día siguiente que no fuere feriado.
    Artículo 15.- Cómputo de plazos de horas. Los plazos de horas establecidos en este Código comenzarán a correr inmediatamente después de ocurrido el hecho que fijare su iniciación, sin interrupción.
    Artículo 16.- Plazos fatales e improrrogables. Los plazos establecidos en este Código son fatales e improrrogables, a menos que se indicare expresamente lo contrario.
    Artículo 17.- Nuevo plazo. El que, por un hecho que no le fuere imputable, por defecto en la notificación, por fuerza mayor o por caso fortuito, se hubiere visto impedido de ejercer un derecho o desarrollar una actividad dentro del plazo establecido por la ley, podrá solicitar al tribunal un nuevo plazo, que le podrá ser otorgado por el mismo período. Dicha solicitud deberá formularse dentro de los cinco días siguientes a aquél en que hubiere cesado el impedimento.
    Artículo 18.- Renuncia de plazos. Los intervinientes en el procedimiento podrán renunciar, total o parcialmente, a los plazos establecidos a su favor, por manifestación expresa.
    Si el plazo fuere común, la abreviación o la renuncia requerirán el consentimiento de todos los intervinientes y la aprobación del tribunal.
    Párrafo 2º Comunicaciones entre autoridades
    Artículo 19.- Requerimientos de información, contenido y formalidades. Todas las autoridades y órganos del Estado deberán realizar las diligencias y proporcionar, sin demora, la información que les requirieren el ministerio público y los tribunales con competencia penal. El requerimiento contendrá la fecha y lugar de expedición, los antecedentes necesarios para su cumplimiento, el plazo que se otorgare para que se llevare a efecto y la determinación del fiscal o tribunal requirente.
    Con todo, tratándose de informaciones o documentos que en virtud de la ley tuvieren carácter secreto, el requerimiento se atenderá observando las prescripciones de la ley respectiva, si las hubiere, y, en caso contrario, adoptándose las precauciones que aseguraren que la información no será divulgada.
    Si la autoridad requerida retardare el envío de los antecedentes solicitados o se negare a enviarlos, a pretexto de su carácter secreto o reservado y el fiscal estimare indispensable la realización de la actuación, remitirá los antecedentes al fiscal regional quien, si compartiere esa apreciación, solicitará a la Corte de Apelaciones respectiva que, previo informe de la autoridad de que se tratare, recabado por la vía que considerare más rápida, resuelva la controversia. La Corte adoptará esta decisión en cuenta. Si fuere el tribunal el que requiriere la información, formulará dicha solicitud directamente ante la Corte de Apelaciones.
    Si la razón invocada por la autoridad requerida para no enviar los antecedentes solicitados fuere que su publicidad pudiere afectar la seguridad nacional, la cuestión deberá ser resuelta por la Corte Suprema.
    Aun cuando la Corte llamada a resolver la controversia rechazare el requerimiento del fiscal, por compartir el juicio de la autoridad a la que se hubieren requerido los antecedentes, podrá ordenar que se suministren al ministerio público o al tribunal los datos que le parecieren necesarios para la adopción de decisiones relativas a la investigación o para el pronunciamiento de resoluciones judiciales.
    Las resoluciones que los ministros de Corte pronunciaren para resolver estas materias no los inhabilitarán para conocer, en su caso, los recursos que se dedujeren en la causa de que se tratare.



    Artículo 20.- Solicitudes entre tribunales. Cuando un tribunal debiere requerir de otro la realización de una diligencia dentro del territorio jurisdiccional de éste, le dirigirá directamente la solicitud, sin más menciones que la indicación de los antecedentes necesarios para la cabal comprensión de la solicitud y las demás expresadas en el inciso primero del artículo anterior.
    Si el tribunal requerido rechazare el cumplimiento del trámite o diligencia indicado en la solicitud, o si transcurriere el plazo fijado para su cumplimiento sin que éste se produjere, el tribunal requirente podrá dirigirse directamente al superior jerárquico del primero para que ordene, agilice o gestione directamente la petición.
    Artículo 20 bis. Tramitación de solicitudes de asistencia internacional. Las solicitudes de autoridades competentes de país extranjero para que se practiquen diligencias en Chile serán remitidas directamente al Ministerio Público, el que solicitará la intervención del juez de garantía del lugar en que deban practicarse, cuando la naturaleza de las diligencias lo hagan necesario de acuerdo con las disposiciones de la ley chilena.

    Artículo 21.- Forma de realizar las comunicaciones. Las comunicaciones señaladas en los artículos precedentes podrán realizarse por cualquier medio idóneo, sin perjuicio del posterior envío de la documentación que fuere pertinente.
    Párrafo 3º Comunicaciones y citaciones del
ministerio público
    Artículo 22.- Comunicaciones del ministerio público. Cuando el ministerio público estuviere obligado a comunicar formalmente alguna actuación a los demás intervinientes en el procedimiento, deberá hacerlo, bajo su responsabilidad, por cualquier medio razonable que resultare eficaz. Será de cargo del ministerio público acreditar la circunstancia de haber efectuado la comunicación.
    Si un interviniente probare que por la deficiencia de la comunicación se hubiere encontrado impedido de ejercer oportunamente un derecho o desarrollar alguna actividad dentro del plazo establecido por la ley, podrá solicitar un nuevo plazo, el que le será concedido bajo las condiciones y circunstancias previstas en el artículo 17.
    Artículo 23.- Citación del ministerio público. Cuando en el desarrollo de su actividad de investigación el fiscal requiriere la comparecencia de una persona, podrá citarla por cualquier medio idóneo. Si la persona citada no compareciere, el fiscal podrá ocurrir ante el juez de garantía para que lo autorice a conducirla compulsivamente a su presencia.
    Con todo, el fiscal no podrá recabar directamente la comparecencia personal de las personas o autoridades a que se refiere el artículo 300. Si la declaración de dichas personas o autoridades fuere necesaria, procederá siempre previa autorización del juez de garantía y conforme lo establece el artículo 301.
    Párrafo 4º Notificaciones y citaciones judiciales
    Artículo 24.- Funcionarios habilitados. Las notificaciones de las resoluciones judiciales se realizarán por los funcionarios del tribunal que hubiere expedido la resolución, que hubieren sido designados para cumplir esta función por el juez presidente del comité de jueces, a propuesta del administrador del tribunal.
    El tribunal podrá ordenar que una o más notificaciones determinadas se practicaren por otro ministro de fe.
    Artículo 25.- Contenido. La notificación deberá incluir una copia íntegra de la resolución de que se tratare, con la identificación del proceso en el que recayere, a menos que la ley expresamente ordenare agregar otros antecedentes, o que el juez lo estimare necesario para la debida información del notificado o para el adecuado ejercicio de sus derechos.
    Artículo 26.- Señalamiento de domicilio de los intervinientes en el procedimiento. En su primera intervención en el procedimiento los intervinientes deberán ser conminados por el juez, por el ministerio público, o por el funcionario público que practicare la primera notificación, a indicar un domicilio dentro de los límites urbanos de la ciudad en que funcionare el tribunal respectivo y en el cual puedan practicárseles las notificaciones posteriores. Asimismo, deberán comunicar cualquier cambio de su domicilio.
    En caso de omisión del señalamiento del domicilio o de la comunicación de sus cambios, o de cualquier inexactitud del mismo o de la inexistencia del domicilio indicado, las resoluciones que se dictaren se notificarán por el estado diario. Para tal efecto, los intervinientes en el procedimiento deberán ser advertidos de esta circunstancia, lo que se hará constar en el acta que se levantare.
    El mismo apercibimiento se formulará al imputado que fuere puesto en libertad, a menos que ello fuere consecuencia de un sobreseimiento definitivo o de una sentencia absolutoria ejecutoriados.
    Artículo 27.- Notificación al ministerio público. El ministerio público será notificado en sus oficinas, para lo cual deberá indicar su domicilio dentro de los límites urbanos de la ciudad en que funcionare el tribunal e informar a éste de cualquier cambio del mismo.
    Artículo 28.- Notificación a otros intervinientes. Cuando un interviniente en el procedimiento contare con defensor o mandatario constituido en él, las notificaciones deberán ser hechas solamente a éste, salvo que la ley o el tribunal dispusiere que también se notifique directamente a aquél.
    Artículo 29.- Notificaciones al imputado privado de libertad. Las notificaciones que debieren realizarse al imputado privado de libertad se le harán en persona en el establecimiento o recinto en que permaneciere, aunque éste se hallare fuera del territorio jurisdiccional del tribunal, mediante la entrega, por un funcionario del establecimiento y bajo la responsabilidad del jefe del mismo, del texto de la resolución respectiva. Al efecto, el tribunal podrá remitir dichas resoluciones, así como cualquier otro antecedente que considerare relevante, por cualquier medio de comunicación idóneo, tales como fax, correo electrónico u otro.
    Si la persona a quien se debiere notificar no supiere o no pudiere leer, la resolución le será leída por el funcionario encargado de notificarla.
    No obstante lo dispuesto en el inciso primero, el tribunal, podrá disponer, por resolución fundada y de manera excepcional, que la notificación de determinadas resoluciones al imputado privado de libertad sea practicada en el recinto en que funcione.
    Artículo 30.- Notificaciones de las resoluciones en las audiencias judiciales. Las resoluciones pronunciadas durante las audiencias judiciales se entenderán notificadas a los intervinientes en el procedimiento que hubieren asistido o debido asistir a las mismas. De estas notificaciones se dejará constancia en el estado diario, pero su omisión no invalidará la notificación.
    Los interesados podrán pedir copias de los registros en que constaren estas resoluciones, las que se expedirán sin demora.
    Artículo 31.- Otras formas de notificación. Cualquier interviniente en el procedimiento podrá proponer para sí otras formas de notificación, que el tribunal podrá aceptar si, en su opinión, resultaren suficientemente eficaces y no causaren indefensión.
    Artículo 32.- Normas aplicables a las notificaciones. En lo no previsto en este párrafo, las notificaciones que hubieren de practicarse a los intervinientes en el procedimiento penal se regirán por las normas contempladas en el Título VI del Libro I del Código de Procedimiento Civil.
    Artículo 33.- Citaciones judiciales. Cuando fuere necesario citar a alguna persona para llevar a cabo una actuación ante el tribunal, se le notificará la resolución que ordenare su comparecencia.
    Se hará saber a los citados el tribunal ante el cual debieren comparecer, su domicilio, la fecha y hora de la audiencia, la identificación del proceso de que se tratare y el motivo de su comparecencia. Al mismo tiempo se les advertirá que la no comparecencia injustificada dará lugar a que sean conducidos por medio de la fuerza pública, que quedarán obligados al pago de las costas que causaren y que pueden imponérseles sanciones. También se les deberá indicar que, en caso de impedimento, deberán comunicarlo y justificarlo ante el tribunal, con anterioridad a la fecha de la audiencia, si fuere posible.
    El tribunal podrá ordenar que el imputado que no compareciere injustificadamente sea detenido o sometido a prisión preventiva hasta la realización de la actuación respectiva. Tratándose de los testigos, peritos u otras personas cuya presencia se requiriere, podrán ser arrestados hasta la realización de la actuación por un máximo de veinticuatro horas e imponérseles, además, una multa de hasta quince unidades tributarias mensuales.
    Si quien no concurriere injustificadamente fuere el defensor o el fiscal, se le aplicará lo dispuesto en el artículo 287.
    Párrafo 5º Resoluciones y otras actuaciones
judiciales
    Artículo 34.- Poder coercitivo. En el ejercicio de sus funciones, el tribunal podrá ordenar directamente la intervención de la fuerza pública y disponer todas las medidas necesarias para el cumplimiento de las actuaciones que ordenare y la ejecución de las resoluciones que dictare.
    Artículo 35.- Nulidad de las actuaciones delegadas. La delegación de funciones en empleados subalternos para realizar actuaciones en que las leyes requirieren la intervención del juez producirá la nulidad de las mismas.
    Artículo 36.- Fundamentación. Será obligación del tribunal fundamentar las resoluciones que dictare, con excepción de aquellas que se pronunciaren sobre cuestiones de mero trámite. La fundamentación expresará sucintamente, pero con precisión, los motivos de hecho y de derecho en que se basaren las decisiones tomadas.
    La simple relación de los documentos del procedimiento o la mención de los medios de prueba o solicitudes de los intervinientes no sustituirá en caso alguno la fundamentación.
    Artículo 37.- Firma de las resoluciones. Las resoluciones judiciales serán suscritas por el juez o por todos los miembros del tribunal que las dictare. Si alguno de los jueces no pudiere firmar se dejará constancia del impedimento.
    No obstante lo anterior, bastará el registro de la audiencia respecto de las resoluciones que se dictaren en ella.
    Artículo 38.- Plazos generales para dictar las resoluciones. Las cuestiones debatidas en una audiencia deberán ser resueltas en ella.
    Las presentaciones escritas serán resueltas por el tribunal antes de las veinticuatro horas siguientes a su recepción.
    Párrafo 6°. Registro de las actuaciones judiciales
      Artículo 39. Reglas generales. De las actuaciones realizadas por o ante el juez de garantía, el tribunal de juicio oral en lo penal, las Cortes de Apelaciones y la Corte Suprema se levantará un registro en la forma señalada en este párrafo.
    En todo caso, las sentencias y demás resoluciones que pronunciare el tribunal serán registradas en su integridad.
    El registro se efectuará por cualquier medio apto para producir fe, que permita garantizar la conservación y la reproducción de su contenido.

    Artículo 40.- DEROGADO

    Artículo 41. Registro de actuaciones ante los tribunales con competencia en materia penal. Las audiencias ante los jueces con competencia en materia penal se registrarán en forma íntegra por cualquier medio que asegure su fidelidad, tal como audio digital, video u otro soporte tecnológico equivalente.

    Artículo 42.- Valor del registro del juicio oral. El registro del juicio oral demostrará el modo en que se hubiere desarrollado la audiencia, la observancia de las formalidades previstas para ella, las personas que hubieren intervenido y los actos que se hubieren llevado a cabo. Lo anterior es sin perjuicio de lo dispuesto en el artículo 359, en lo que corresponda.
    La omisión de formalidades del registro sólo lo privará de valor cuando ellas no pudieren ser suplidas con certeza sobre la base de otros elementos contenidos en el mismo o de otros antecedentes confiables que dieren testimonio de lo ocurrido en la audiencia.
    Artículo 43.- Conservación de los registros. Mientras dure la investigación o el respectivo proceso, la conservación de los registros estará a cargo del juzgado de garantía y del tribunal de juicio oral en lo penal respectivo, de conformidad a lo previsto en el Código Orgánico de Tribunales.
    Cuando, por cualquier causa, se viere dañado el soporte material del registro afectando su contenido, el tribunal ordenará reemplazarlo en todo o parte por una copia fiel, que obtendrá de quien la tuviere, si no dispusiere de ella directamente.
    Si no existiere copia fiel, las resoluciones se dictarán nuevamente, para lo cual el tribunal reunirá los antecedentes que le permitan fundamentar su preexistencia y contenido, y las actuaciones se repetirán con las formalidades previstas para cada caso. En todo caso, no será necesario volver a dictar las resoluciones o repetir las actuaciones que sean el antecedente de resoluciones conocidas o en etapa de cumplimiento o ejecución.
    Artículo 44.- Examen del registro y certificaciones. Salvas las excepciones expresamente previstas en la ley, los intervinientes siempre tendrán acceso al contenido de los registros.
    Los registros podrán también ser consultados por terceros cuando dieren cuenta de actuaciones que fueren públicas de acuerdo con la ley, a menos que, durante la investigación o la tramitación de la causa, el tribunal restringiere el acceso para evitar que se afecte su normal substanciación o el principio de inocencia.
    En todo caso, los registros serán públicos transcurridos cinco años desde la realización de las actuaciones consignadas en ellos.
    A petición de un interviniente o de cualquier persona, el funcionario competente del tribunal expedirá copias fieles de los registros o de la parte de ellos que fuere pertinente, con sujeción a lo dispuesto en los incisos anteriores.
    Además dicho funcionario certificará si se hubieren deducido recursos en contra de la sentencia definitiva.
    Párrafo 7º Costas
    Artículo 45.- Pronunciamiento sobre costas. Toda resolución que pusiere término a la causa o decidiere un incidente deberá pronunciarse sobre el pago de las costas del procedimiento.
    Artículo 46.- Contenido. Las costas del procedimiento penal comprenderán tanto las procesales como las personales.
    Artículo 47 .- Condena. Las costas serán de cargo del condenado.
    La víctima que abandonare la acción civil soportará las costas que su intervención como parte civil hubiere causado. También las soportará el querellante que abandonare la querella.
    No obstante lo dispuesto en los incisos anteriores, el tribunal, por razones fundadas que expresará determinadamente, podrá eximir total o parcialmente del pago de las costas, a quien debiere soportarlas.
    Artículo 48.- Absolución y sobreseimiento definitivo. Cuando el imputado fuere absuelto o sobreseído definitivamente, el ministerio público será condenado en costas, salvo que hubiere formulado la acusación en cumplimiento de la orden judicial a que se refiere el inciso segundo del artículo 462 o cuando el tribunal estime razonable eximirle por razones fundadas.
    En dicho evento será también condenado el querellante, salvo que el tribunal lo eximiere del pago, total o parcialmente, por razones fundadas que expresará determinadamente.

    Artículo 49.- Distribución de costas. Cuando fueren varios los intervinientes condenados al pago de las costas, el tribunal fijará la parte o proporción que corresponderá soportar a cada uno de ellos.
    Artículo 50.- Personas exentas. Los fiscales, los abogados y los mandatarios de los intervinientes en el procedimiento no podrán ser condenados personalmente al pago de las costas, salvo los casos de notorio desconocimiento del derecho o de grave negligencia en el desempeño de sus funciones, en los cuales se les podrá imponer, por resolución fundada, el pago total o parcial de las costas.
    Artículo 51.- Gastos. Cuando fuere necesario efectuar un gasto cuyo pago correspondiere a los intervinientes, el tribunal estimará su monto y dispondrá su consignación anticipada.
    En todo caso, el Estado soportará los gastos de los intervinientes que gozaren del privilegio de pobreza.
    Párrafo 8º Normas supletorias
    Artículo 52.- Aplicación de normas comunes a todo procedimiento. Serán aplicables al procedimiento penal, en cuanto no se opusieren a lo estatuido en este Código o en leyes especiales, las normas comunes a todo procedimiento contempladas en el Libro I del Código de Procedimiento Civil.
    Título III
    Acción penal

    Párrafo 1º Clases de acciones
    Artículo 53.- Clasificación de la acción penal. La acción penal es pública o privada.
    La acción penal pública para la persecución de todo delito que no esté sometido a regla especial deberá ser ejercida de oficio por el ministerio público. Podrá ser ejercida, además, por las personas que determine la ley, con arreglo a las disposiciones de este Código. Se concede siempre acción penal pública para la persecución de los delitos cometidos contra menores de edad.
    La acción penal privada sólo podrá ser ejercida por la víctima.
    Excepcionalmente, la persecución de algunos delitos de acción penal pública requiere la denuncia previa de la víctima.
    Artículo 54.- Delitos de acción pública previa instancia particular. En los delitos de acción pública previa instancia particular no podrá procederse de oficio sin que, a lo menos, el ofendido por el delito hubiere denunciado el hecho a la justicia, al ministerio público o a la policía.
    Tales delitos son:
    a) Las lesiones previstas en los artículos 399 y 494, número 5º, del Código Penal;
    b) La violación de domicilio;
    c) La violación de secretos prevista en los artículos 231 y 247, inciso segundo, del Código Penal;
    d) Las amenazas previstas en los artículos 296 y 297 del Código Penal;
    e) Los previstos en la ley N° 19.039, que establece normas aplicables a los privilegios industriales y protección de los derechos de propiedad industrial;
    f) La comunicación fraudulenta de secretos de la fábrica en que el imputado hubiere estado o estuviere empleado, y
    g) Los que otras leyes señalaren en forma expresa.
    A falta del ofendido por el delito, podrán denunciar el hecho las personas indicadas en el inciso segundo del artículo 108, de conformidad a lo previsto en esa disposición.
    Cuando el ofendido se encontrare imposibilitado de realizar libremente la denuncia, o cuando quienes pudieren formularla por él se encontraren imposibilitados de hacerlo o aparecieren implicados en el hecho, el ministerio público podrá proceder de oficio.
    Iniciado el procedimiento, éste se tramitará de acuerdo con las normas generales relativas a los delitos de acción pública.
    Artículo 55.- Delitos de acción privada. No podrán ser ejercidas por otra persona que la víctima, las acciones que nacen de los siguientes delitos:
    a) La calumnia y la injuria;
    b) La falta descrita en el número 11 del artículo 496 del Código Penal;
    c) La provocación a duelo y el denuesto o descrédito público por no haberlo aceptado, y d) El matrimonio del menor llevado a efecto sin el consentimiento de las personas designadas por la ley y celebrado de acuerdo con el funcionario llamado a autorizarlo.
    Artículo 56.- Renuncia de la acción penal. La acción penal pública no se extingue por la renuncia de la persona ofendida.
    Pero se extinguen por esa renuncia la acción penal privada y la civil derivada de cualquier clase de delitos.
    Si el delito es de aquellos que no pueden ser perseguidos sin previa instancia particular, la renuncia de la víctima a denunciarlo extinguirá la acción penal, salvo que se tratare de delito perpetrado contra menores de edad.
    Esta renuncia no la podrá realizar el ministerio público.
    Artículo 57.- Efectos relativos de la renuncia. La renuncia de la acción penal sólo afectará al renunciante y a sus sucesores, y no a otras personas a quienes también correspondiere la acción.
    Artículo 58.- Responsabilidad penal. La acción penal, fuere pública o privada, no puede entablarse sino contra las personas responsables del delito.
    La responsabilidad penal sólo puede hacerse efectiva en las personas naturales. Por las personas jurídicas responden los que hubieren intervenido en el acto punible, sin perjuicio de la responsabilidad civil que las afectare.
    Párrafo 2º Acciones civiles
    Artículo 59.- Principio general. La acción civil que tuviere por objeto únicamente la restitución de la cosa, deberá interponerse siempre durante el respectivo procedimiento penal, de conformidad a lo previsto en el artículo 189.
    Asimismo, durante la tramitación del procedimiento penal la víctima podrá deducir respecto del imputado, con arreglo a las prescripciones de este Código, todas las restantes acciones que tuvieren por objeto perseguir las responsabilidades civiles derivadas del hecho punible. La víctima podrá también ejercer esas acciones civiles ante el tribunal civil correspondiente. Con todo, admitida a tramitación la demanda civil en el procedimiento penal, no se podrá deducir nuevamente ante un tribunal civil.
    Con la sola excepción indicada en el inciso primero, las otras acciones encaminadas a obtener la reparación de las consecuencias civiles del hecho punible que interpusieren personas distintas de la víctima, o se dirigieren contra personas diferentes del imputado, deberán plantearse ante el tribunal civil que fuere competente de acuerdo a las reglas generales.
    Artículo 60.- Oportunidad para interponer la demanda civil. La demanda civil en el procedimiento penal deberá interponerse en la oportunidad prevista en el artículo 261, por escrito y cumpliendo con los requisitos exigidos por el artículo 254 del Código de Procedimiento Civil. La demanda civil del querellante deberá deducirse conjuntamente con su escrito de adhesión o acusación.
    La demanda civil deberá contener la indicación de los medios de prueba, en los mismos términos expresados en el artículo 259.
    Artículo 61.- Preparación de la demanda civil. Sin perjuicio de lo dispuesto en el artículo anterior, con posterioridad a la formalización de la investigación la víctima podrá preparar la demanda civil solicitando la práctica de diligencias que considerare necesarias para esclarecer los hechos que serán objeto de su demanda, aplicándose, en tal caso, lo establecido en los artículos 183 y 184.
    Asimismo, se podrá cautelar la demanda civil, solicitando alguna de las medidas previstas en el artículo 157.
    La preparación de la demanda civil interrumpe la prescripción. No obstante, si no se dedujere demanda en la oportunidad prevista en el artículo precedente, la prescripción se considerará como no interrumpida.
    Artículo 62.- Actuación del demandado. El imputado deberá oponer las excepciones que corresponda y contestar la demanda civil en la oportunidad señalada en el artículo 263. Podrá, asimismo, señalar los vicios formales de que adoleciere la demanda civil, requiriendo su corrección.
    En su contestación, deberá indicar cuáles serán los medios probatorios de que pensare valerse, del modo previsto en el artículo 259.
    Artículo 63.- Incidentes relacionados con la demanda y su contestación. Todos los incidentes y excepciones deducidos con ocasión de la interposición o contestación de la demanda deberán resolverse durante la audiencia de preparación del juicio oral, sin perjuicio de lo establecido en el artículo 270.
    Artículo 64.- Desistimiento y abandono. La víctima podrá desistirse de su acción en cualquier estado del procedimiento.
    Se considerará abandonada la acción civil interpuesta en el procedimiento penal, cuando la víctima no compareciere, sin justificación, a la audiencia de preparación del juicio oral o a la audiencia del juicio oral.
    Artículo 65.- Efectos de la extinción de la acción civil. Extinguida la acción civil no se entenderá extinguida la acción penal para la persecución del hecho punible.
    Artículo 66.- Efectos del ejercicio exclusivo de la acción civil. Cuando sólo se ejerciere la acción civil respecto de un hecho punible de acción privada se considerará extinguida, por esa circunstancia, la acción penal.
    Para estos efectos no constituirá ejercicio de la acción civil la solicitud de diligencias destinadas a preparar la demanda civil o a asegurar su resultado, que se formulare en el procedimiento penal.
    Artículo 67.- Independencia de la acción civil respecto de la acción penal. La circunstancia de dictarse sentencia absolutoria en materia penal no impedirá que se de lugar a la acción civil, si fuere legalmente procedente.
    Artículo 68.- Curso de la acción civil ante suspensión o terminación del procedimiento penal. Si antes de comenzar el juicio oral, el procedimiento penal continuare de conformidad a las normas que regulan el procedimiento abreviado, o por cualquier causa terminare o se suspendiere, sin decisión acerca de la acción civil que se hubiere deducido oportunamente, la prescripción continuará interrumpida siempre que la víctima presentare su demanda ante el tribunal civil competente en el término de sesenta días siguientes a aquél en que, por resolución ejecutoriada, se dispusiere la suspensión o terminación del procedimiento penal.
    En este caso, la demanda y la resolución que recayere en ella se notificarán por cédula y el juicio se sujetará a las reglas del procedimiento sumario. Si la demanda no fuere deducida ante el tribunal civil competente dentro del referido plazo, la prescripción continuará corriendo como si no se hubiere interrumpido.
    Si en el procedimiento penal se hubieren decretado medidas destinadas a cautelar la demanda civil, éstas se mantendrán vigentes por el plazo indicado en el inciso primero, tras el cual quedarán sin efecto si, solicitadas oportunamente, el tribunal civil no las mantuviere.
    Si, comenzado el juicio oral, se dictare sobreseimiento de acuerdo a las prescripciones de este Código, el tribunal deberá continuar con el juicio para el solo conocimiento y fallo de la cuestión civil.
    Título IV
    Sujetos procesales

    Párrafo 1º El tribunal
    Artículo 69.- Denominaciones. Salvo que se disponga expresamente lo contrario, cada vez que en este Código se hiciere referencia al juez, se entenderá que se alude al juez de garantía; si la referencia fuere al tribunal de juicio oral en lo penal, deberá entenderse hecha al tribunal colegiado encargado de conocer el juicio mencionado.
    Por su parte, la mención de los jueces se entenderá hecha a los jueces de garantía, a los jueces del tribunal de juicio oral en lo penal o a todos ellos, según resulte del contexto de la disposición en que se utilice. De igual manera se entenderá la alusión al tribunal, que puede corresponder al juez de garantía, al tribunal de juicio oral en lo penal, a la Corte de Apelaciones o a la Corte Suprema
    Artículo 70 .- Juez de garantía competente. El juez de garantía llamado por la ley a conocer las gestiones a que de lugar el respectivo procedimiento se pronunciará sobre las autorizaciones judiciales previas que solicitare el ministerio público para realizar actuaciones que privaren, restringieren o perturbaren el ejercicio de derechos asegurados por la Constitución.
    Si la detención se practicare en un lugar que se encontrare fuera del territorio jurisdiccional del juez que hubiere emitido la orden, será también competente para conocer de la audiencia judicial del detenido el juez de garantía del lugar donde se hubiere practicado la detención, cuando la orden respectiva hubiere emanado de un juez con competencia en una ciudad asiento de Corte de Apelaciones diversa. Cuando en la audiencia judicial se decretare la prisión preventiva del imputado, el juez deberá ordenar su traslado inmediato al establecimiento penitenciario del territorio jurisdiccional del juez del procedimiento. Lo previsto en este inciso no tendrá aplicación cuando la orden de detención emanare de un juez de garantía de la Región Metropolitana y ésta se practicare dentro del territorio de la misma, caso en el cual la primera audiencia judicial siempre deberá realizarse ante el juzgado naturalmente competente.
    En los demás casos, cuando debieren efectuarse actuaciones fuera del territorio jurisdiccional del juzgado de garantía y se tratare de diligencias u órdenes urgentes, el Ministerio Público también podrá pedir la autorización directamente al juez de garantía del lugar. Una vez realizada la diligencia o cumplida la orden, el Ministerio Público dará cuenta a la brevedad al juez de garantía del procedimiento.

    Artículo 71.- Atribuciones de dirección de las audiencias y disciplina dentro de ellas. Las reglas contempladas en el Párrafo 3º del Título III del Libro Segundo serán aplicables durante las audiencias que se celebraren ante el juez de garantía, correspondiendo a este último el ejercicio de las facultades que se le entregan al presidente de la sala o al tribunal de juicio oral en lo penal en dichas disposiciones.
    Artículo 72.- Facultades durante conflictos de competencia. Si se suscitare un conflicto de competencia entre jueces de varios juzgados de garantía en relación con el conocimiento de una misma causa criminal, mientras no se dirimiere dicha competencia, cada uno de ellos estará facultado para realizar las actuaciones urgentes y otorgar las autorizaciones que, con el mismo carácter, les solicitare el ministerio público.
    De los jueces entre quienes se hubiere suscitado la contienda, aquél en cuyo territorio jurisdiccional se encontraren quienes estuvieren privados de libertad en la causa resolverá sobre su libertad.
    Artículo 73.- Efectos de la resolución que dirime la competencia. Dirimida la competencia, serán puestas inmediatamente a disposición del juez competente las personas que se encontraren privadas de libertad, así como los antecedentes que obraren en poder de los demás jueces que hubieren intervenido.
    Todas las actuaciones practicadas ante los jueces que resultaren incompetentes serán válidas, sin necesidad de ratificación por el juez que fuere declarado competente.
    Artículo 74.- Preclusión de los conflictos de competencia. Transcurridos tres días desde la notificación de la resolución que fijare fecha para la realización de la audiencia del juicio oral, la incompetencia territorial del tribunal del juicio oral en lo penal no podrá ser declarada de oficio ni promovida por las partes.
    Si durante la audiencia de preparación del juicio oral se planteare un conflicto de competencia, no se suspenderá la tramitación, pero no se pronunciará la resolución a que alude el artículo 277 mientras no se resolviere el conflicto.
    Artículo 75.- Inhabilitación del juez de garantía. Planteada la inhabilitación del juez de garantía, quien debiere subrogarlo conforme a la ley continuará conociendo de todos los trámites anteriores a la audiencia de preparación del juicio oral, la que no se realizará hasta que se resolviere la inhabilitación.
    Artículo 76.- Inhabilitación de los jueces del tribunal del juicio oral. Las solicitudes de inhabilitación de los jueces del tribunal de juicio oral deberán plantearse, a más tardar, dentro de los tres días siguientes a la notificación de la resolución que fijare fecha para el juicio oral, y se resolverán con anterioridad al inicio de la respectiva audiencia.
    Cuando los hechos que constituyeren la causal de implicancia o recusación llegaren a conocimiento de la parte con posterioridad al vencimiento del plazo previsto en el inciso anterior y antes del inicio del juicio oral, el incidente respectivo deberá ser promovido al iniciarse la audiencia del juicio oral.
    Con posterioridad al inicio de la audiencia del juicio oral, no podrán deducirse incidentes relativos a la inhabilitación de los jueces que integraren el tribunal. Con todo, si cualquiera de los jueces advirtiere un hecho nuevo constitutivo de causal de inhabilidad, el tribunal podrá declararla de oficio.
    El tribunal continuará funcionando con exclusión del o de los miembros inhabilitados, si éstos pudieren ser reemplazados de inmediato en virtud de lo dispuesto en el inciso quinto del artículo 281, o si continuare integrado por, a lo menos, dos jueces que hubieren concurrido a toda la audiencia. En este último caso, deberán alcanzar unanimidad para pronunciar la sentencia definitiva. Si no se cumpliere alguna de estas condiciones, se anulará todo lo obrado en el juicio oral.
    Párrafo 2º El ministerio público
    Artículo 77.- Facultades. Los fiscales ejercerán y sustentarán la acción penal pública en la forma prevista por la ley. Con ese propósito practicarán todas las diligencias que fueren conducentes al éxito de la investigación y dirigirán la actuación de la policía, con estricta sujeción al principio de objetividad consagrado en la Ley Orgánica Constitucional del Ministerio Público.
    Artículo 78.- Información y protección a las víctimas. Será deber de los fiscales durante todo el procedimiento adoptar medidas, o solicitarlas, en su caso, para proteger a las víctimas de los delitos; facilitar su intervención en el mismo y evitar o disminuir al mínimo cualquier perturbación que hubieren de soportar con ocasión de los trámites en que debieren intervenir.
    Los fiscales estarán obligados a realizar, entre otras, las siguientes actividades a favor de la víctima:
    a) Entregarle información acerca del curso y resultado del procedimiento, de sus derechos y de las actividades que debiere realizar para ejercerlos.
    b) Ordenar por sí mismos o solicitar al tribunal, en su caso, las medidas destinadas a la protección de la víctima y su familia frente a probables hostigamientos, amenazas o atentados.
    c) Informarle sobre su eventual derecho a indemnización y la forma de impetrarlo, y remitir los antecedentes, cuando correspondiere, al organismo del Estado que tuviere a su cargo la representación de la víctima en el ejercicio de las respectivas acciones civiles.
    d) Escuchar a la víctima antes de solicitar o resolver la suspensión del procedimiento o su terminación por cualquier causa.
    Si la víctima hubiere designado abogado, el ministerio público estará obligado a realizar también a su respecto las actividades señaladas en las letras a) y d) precedentes.
    Artículo 78 bis.- Protección de la integridad física y psicológica de las personas objeto del tráfico ilícito de migrantes y víctimas de trata de personas. El Ministerio Público adoptará las medidas necesarias, o las solicitará, en su caso, tendientes a asegurar la protección de las víctimas de estos delitos durante el proceso penal, teniendo presente la especial condición de vulnerabilidad que las afecta.
    Cuando se trate de menores de dieciocho años, los servicios públicos a cargo de la protección de la infancia y la adolescencia deberán facilitar su acceso a las prestaciones especializadas que requieran, especialmente, aquellas tendientes a su recuperación integral y a la revinculación familiar, si fuere procedente de acuerdo al interés superior del menor de edad.
    Inciso Derogado.





NOTA
    El Art. primero transitorio de la ley 21.057, establece que la modificación introducida al presente artículo comenzará a regir de manera gradual, en plazos contados desde la publicación del Reglamento: Primera etapa: seis meses después, respecto de las regiones XV, I, II, VII, XI y XII. Segunda etapa: dieciocho meses después, respecto de las regiones III, IV, VIII, IX y XIV. Tercera etapa:  treinta meses después, comprendiendo las regiones V, VI, X y Metropolitana.
    Párrafo 3º La policía
    Artículo 79.- Función de la policía en el procedimiento penal. La Policía de Investigaciones de Chile será auxiliar del ministerio público en las tareas de investigación y deberá llevar a cabo las diligencias necesarias para cumplir los fines previstos en este Código, en especial en los artículos 180, 181 y 187, de conformidad a las instrucciones que le dirigieren los fiscales. Tratándose de delitos que dependieren de instancia privada se estará a lo dispuesto en los artículos 54 y 400 de este Código. Asimismo, le corresponderá ejecutar las medidas de coerción que se decretaren.
    Carabineros de Chile, en el mismo carácter de auxiliar del ministerio público, deberá desempeñar las funciones previstas en el inciso precedente cuando el fiscal a cargo del caso así lo dispusiere.
    Sin perjuicio de lo previsto en los incisos anteriores, tratándose de investigaciones en las que apareciere necesario el carácter auxiliar de Gendarmería de Chile para la realización de diligencias de investigación en el interior de establecimientos penales, el Ministerio Público también podrá impartirle instrucciones. En estos casos Gendarmería de Chile deberá actuar de conformidad con lo dispuesto en este Código.

    Artículo 80.- Dirección del ministerio público. Los funcionarios señalados en el artículo anterior que, en cada caso, cumplieren funciones previstas en este Código, ejecutarán sus tareas bajo la dirección y responsabilidad de los fiscales y de acuerdo a las instrucciones que éstos les impartieren para los efectos de la investigación, sin perjuicio de su dependencia de las autoridades de la institución a la que pertenecieren.
    También deberán cumplir las órdenes que les dirigieren los jueces para la tramitación del procedimiento.
    Los funcionarios antes mencionados deberán cumplir de inmediato y sin más trámite las órdenes que les impartieren los fiscales y los jueces, cuya procedencia, conveniencia y oportunidad no podrán calificar, sin perjuicio de requerir la exhibición de la autorización judicial previa, cuando correspondiere, salvo los casos urgentes a que se refiere el inciso final del artículo 9º, en los cuales la autorización judicial se exhibirá posteriormente.

    Artículo 81.- Comunicaciones entre el ministerio público y la policía. Las comunicaciones que los fiscales y la policía debieren dirigirse en relación con las actividades de investigación de un caso particular se realizarán en la forma y por los medios más expeditos posibles.
    Artículo 82.- Imposibilidad de cumplimiento. El funcionario de la policía que, por cualquier causa, se encontrare impedido de cumplir una orden que hubiere recibido del ministerio público o de la autoridad judicial, pondrá inmediatamente esta circunstancia en conocimiento de quien la hubiere emitido y de su superior jerárquico en la institución a que perteneciere.
    El fiscal o el juez que hubiere emitido la orden podrá sugerir o disponer las modificaciones que estimare convenientes para su debido cumplimiento, o reiterar la orden, si en su concepto no existiere imposibilidad.
    Artículo 83.- Actuaciones de la policía sin orden previa. Corresponderá a los funcionarios de Carabineros de Chile y de la Policía de Investigaciones de Chile realizar las siguientes actuaciones, sin necesidad de recibir previamente instrucciones particulares de los fiscales:
    a) Prestar auxilio a la víctima;
    b) Practicar la detención en los casos de flagrancia, conforme a la ley;
    c) Resguardar el sitio del suceso. Deberán preservar siempre todos los lugares donde se hubiere cometido un delito o se encontraren señales o evidencias de su perpetración, fueren éstos abiertos o cerrados, públicos o privados. Para el cumplimiento de este deber, procederán a su inmediata clausura o aislamiento, impedirán el acceso a toda persona ajena a la investigación y evitarán que se alteren, modifiquen o borren de cualquier forma los rastros o vestigios del hecho, o que se remuevan o trasladen los instrumentos usados para llevarlo a cabo.
    El personal policial experto deberá recoger, identificar y conservar bajo sello los objetos, documentos o instrumentos de cualquier clase que parecieren haber servido a la comisión del hecho investigado, sus efectos o los que pudieren ser utilizados como medios de prueba, para ser remitidos a quien correspondiere, dejando constancia, en el registro que se levantare, de la individualización completa del o los funcionarios policiales que llevaren a cabo esta diligencia;
    En aquellos casos en que en la localidad donde ocurrieren los hechos no exista personal policial experto y la evidencia pueda desaparecer, el personal policial que hubiese llegado al sitio del suceso deberá recogerla y guardarla en los términos indicados en el párrafo precedente y hacer entrega de ella al Ministerio Público, a la mayor brevedad posible.
    En el caso de delitos flagrantes cometidos en zonas rurales o de difícil acceso, la policía deberá practicar de inmediato las primeras diligencias de investigación pertinentes, dando cuenta al fiscal que corresponda de lo hecho, a la mayor brevedad. Asimismo, el personal policial realizará siempre las diligencias señaladas en la presente letra cuando reciba denuncias conforme a lo señalado en la letra e) de este artículo y dará cuenta al fiscal que corresponda inmediatamente después de realizarlas. Lo anterior tendrá lugar sólo respecto de los delitos que determine el Ministerio Público a través de las instrucciones generales a que se refiere el artículo 87. En dichas instrucciones podrá limitarse esta facultad cuando se tratare de denuncias relativas a hechos lejanos en el tiempo.
    d) Identificar a los testigos y consignar las declaraciones que éstos prestaren voluntariamente, en los casos de delitos flagrantes, en que se esté resguardando el sitio del suceso, o cuando se haya recibido una denuncia en los términos de la letra b) de este artículo. Fuera de los casos anteriores, los funcionarios policiales deberán consignar siempre las declaraciones que voluntariamente presten testigos sobre la comisión de un delito o de sus partícipes o sobre cualquier otro antecedente que resulte útil para el esclarecimiento de un delito y la determinación de sus autores y partícipes, debiendo comunicar o remitir a la brevedad dicha información al Ministerio Público, todo lo anterior de acuerdo con las instrucciones generales que dicte el Fiscal Nacional según lo dispuesto en el artículo 87;
    e) Recibir las denuncias del público, y
    f) Efectuar las demás actuaciones que dispusieren otros cuerpos legales.
    Artículo 84.- Información al ministerio público. Recibida una denuncia, la policía informará inmediatamente y por el medio más expedito al ministerio público. Sin perjuicio de ello, procederá, cuando correspondiere, a realizar las actuaciones previstas en el artículo precedente, respecto de las cuales se aplicará, asimismo, la obligación de información inmediata.
    Artículo 85.- Control de identidad. Los funcionarios policiales señalados en el artículo 83 deberán, además, sin orden previa de los fiscales, solicitar la identificación de cualquier persona en los casos fundados, en que, según las circunstancias, estimaren que exista algún indicio de que ella hubiere cometido o intentado cometer un crimen, simple delito o falta; de que se dispusiere a cometerlo; de que pudiere suministrar informaciones útiles para la indagación de un crimen, simple delito o falta; o en el caso de la persona que se encapuche o emboce para ocultar, dificultar o disimular su identidad. El funcionario policial deberá otorgar a la persona facilidades para encontrar y exhibir estos instrumentos.
    Procederá también tal solicitud cuando los funcionarios policiales tengan algún antecedente que les permita inferir que una determinada persona tiene alguna orden de detención pendiente.
    La identificación se realizará en el lugar en que la persona se encontrare, por medio de documentos de identificación expedidos por la autoridad pública, como cédula de identidad, licencia de conducir o pasaporte. El funcionario policial deberá otorgar a la persona facilidades para encontrar y exhibir estos instrumentos.
    Durante este procedimiento, sin necesidad de nuevo indicio, la policía podrá proceder al registro de las vestimentas, equipaje o vehículo de la persona cuya identidad se controla, y cotejar la existencia de las órdenes de detención que pudieren afectarle. La policía procederá a la detención, sin necesidad de orden judicial y en conformidad a lo dispuesto en el artículo 129, de quienes se sorprenda, a propósito del registro, en alguna de las hipótesis del artículo 130, así como de quienes al momento del cotejo registren orden de detención pendiente.
    En caso de negativa de una persona a acreditar su identidad, o si habiendo recibido las facilidades del caso no le fuere posible hacerlo, la policía la conducirá a la unidad policial más cercana para fines de identificación. En dicha unidad se le darán facilidades para procurar una identificación satisfactoria por otros medios distintos de los ya mencionados, dejándola en libertad en caso de obtenerse dicho resultado, previo cotejo de la existencia de órdenes de detención que pudieren afectarle. Si no resultare posible acreditar su identidad, se le tomarán huellas digitales, las que sólo podrán ser usadas para fines de identificación y, cumplido dicho propósito, serán destruidas.
    El conjunto de procedimientos detallados en los incisos precedentes no deberá extenderse por un plazo superior a ocho horas, transcurridas las cuales la persona que ha estado sujeta a ellos deberá ser puesta en libertad, salvo que existan indicios de que ha ocultado su verdadera identidad o ha proporcionado una falsa, caso en el cual se estará a lo dispuesto en el inciso siguiente.
    Si la persona se niega a acreditar su identidad o se encuentra en la situación indicada en el inciso anterior, se procederá a su detención como autora de la falta prevista y sancionada en el Nº 5 del artículo 496 del Código Penal. El agente policial deberá informar, de inmediato, de la detención al fiscal, quien podrá dejarla sin efecto u ordenar que el detenido sea conducido ante el juez dentro de un plazo máximo de veinticuatro horas, contado desde que la detención se hubiere practicado. Si el fiscal nada manifestare, la policía deberá presentar al detenido ante la autoridad judicial en el plazo indicado.
    Los procedimientos dirigidos a obtener la identidad de una persona en conformidad a los incisos precedentes, deberán realizarse en la forma más expedita posible, y el abuso en su ejercicio podrá ser constitutivo del delito previsto y sancionado en el artículo 255 del Código Penal.
    Si no pudiere lograrse la identificación por los documentos expedidos por la autoridad pública, las policías podrán utilizar medios tecnológicos de identificación para concluir con el procedimiento de identificación de que se trata.




    Artículo 86.- Derechos de la persona sujeta a control de identidad. En cualquier caso que hubiere sido necesario conducir a la unidad policial a la persona cuya identidad se tratare de averiguar en virtud del artículo precedente, el funcionario que practicare el traslado deberá informarle verbalmente de su derecho a que se comunique a su familia o a la persona que indicare, de su permanencia en el cuartel policial. El afectado no podrá ser ingresado a celdas o calabozos, ni mantenido en contacto con personas detenidas.
    Artículo 87.- Instrucciones generales. Sin perjuicio de las instrucciones particulares que el fiscal impartiere en cada caso, el ministerio público regulará mediante instrucciones generales la forma en que la policía cumplirá las funciones previstas en los artículos 83 y 85, así como la forma de proceder frente a hechos de los que tomare conocimiento y respecto de los cuales los datos obtenidos fueren insuficientes para estimar si son constitutivos de delito. Asimismo, podrá impartir instrucciones generales relativas a la realización de diligencias inmediatas para la investigación de determinados delitos.
    Artículo 87 bis.- Se considerará falta contra el buen servicio de los funcionarios policiales el incumplimiento de las instrucciones impartidas por los fiscales a las policías, dando lugar a las responsabilidades administrativas que correspondan, conforme lo establecen los respectivos reglamentos.


    Artículo 88.- Solicitud de registros de actuaciones. El ministerio público podrá requerir en cualquier momento los registros de las actuaciones de la policía.
    Artículo 89.- Examen de vestimentas, equipaje o vehículos. Se podrá practicar el examen de las vestimentas que llevare el detenido, del equipaje que portare o del vehículo que condujere.
    Para practicar el examen de vestimentas, se comisionará a personas del mismo sexo del imputado y se guardarán todas las consideraciones compatibles con la correcta ejecución de la diligencia.

    Artículo 90.- Levantamiento del cadáver. En los casos de muerte en la vía pública, y sin perjuicio de las facultades que corresponden a los órganos encargados de la persecución penal, la descripción a que se refiere el artículo 181 y la orden de levantamiento del cadáver podrán ser realizadas por el jefe de la unidad policial correspondiente, en forma personal o por intermedio de un funcionario de su dependencia, quien dejará registro de lo obrado, en conformidad a las normas generales de este Código.
    Artículo 91.- Declaraciones del imputado ante la policía. La policía sólo podrá interrogar autónomamente al imputado en presencia de su defensor. Si éste no estuviere presente durante el interrogatorio, las preguntas se limitarán a constatar la identidad del sujeto.
    Si, en ausencia del defensor, el imputado manifestare su deseo de declarar, la policía tomará las medidas necesarias para que declare inmediatamente ante el fiscal. Si esto no fuere posible, la policía podrá consignar las declaraciones que se allanare a prestar, bajo la responsabilidad y con la autorización del fiscal. El defensor podrá incorporarse siempre y en cualquier momento a esta diligencia.
    Artículo 92.- Prohibición de informar. Los funcionarios policiales no podrán informar a los medios de comunicación social acerca de la identidad de detenidos, imputados, víctimas, testigos, ni de otras personas que se encontraren o pudieren resultar vinculadas a la investigación de un hecho punible.
    Párrafo 4º El imputado

    I.- Derechos y garantías del imputado
    Artículo 93.- Derechos y garantías del imputado. Todo imputado podrá hacer valer, hasta la terminación del proceso, los derechos y garantías que le confieren las leyes.
    En especial, tendrá derecho a:
    a) Que se le informe de manera específica y clara acerca de los hechos que se le imputaren y los derechos que le otorgan la Constitución y las leyes;
    b) Ser asistido por un abogado desde los actos iniciales de la investigación;
    c) Solicitar de los fiscales diligencias de investigación destinadas a desvirtuar las imputaciones que se le formularen;
    d) Solicitar directamente al juez que cite a una audiencia, a la cual podrá concurrir con su abogado o sin él, con el fin de prestar declaración sobre los hechos materia de la investigación;
    e) Solicitar que se active la investigación y conocer su contenido, salvo en los casos en que alguna parte de ella hubiere sido declarada secreta y sólo por el tiempo que esa declaración se prolongare;
    f) Solicitar el sobreseimiento definitivo de la causa y recurrir contra la resolución que lo rechazare;
    g) Guardar silencio o, en caso de consentir en prestar declaración, a no hacerlo bajo juramento. Sin perjuicio de lo dispuesto en los artículos 91 y 102, al ser informado el imputado del derecho que le asiste conforme a esta letra, respecto de la primera declaración que preste ante el fiscal o la policía, según el caso, deberá señalársele lo siguiente: "Tiene derecho a guardar silencio. El ejercicio de este derecho no le ocasionará ninguna consecuencia legal adversa; sin embargo, si renuncia a él, todo lo que manifieste podrá ser usado en su contra.";
    h) No ser sometido a tortura ni a otros tratos crueles, inhumanos o degradantes, e
    i) No ser juzgado en ausencia, sin perjuicio de las responsabilidades que para él derivaren de la situación de rebeldía.

    Artículo 94.- Imputado privado de libertad. El imputado privado de libertad tendrá, además, las siguientes garantías y derechos:
    a) A que se le exprese específica y claramente el motivo de su privación de libertad y, salvo el caso de delito flagrante, a que se le exhiba la orden que la dispusiere;
    b) A que el funcionario a cargo del procedimiento de detención o de aprehensión le informe de los derechos a que se refiere el inciso segundo del artículo 135;
    c) A ser conducido sin demora ante el tribunal que hubiere ordenado su detención;
    d) A solicitar del tribunal que le conceda la libertad;
    e) A que el encargado de la guardia del recinto policial al cual fuere conducido informe, en su presencia, al familiar o a la persona que le indicare, que ha sido detenido o preso, el motivo de la detención o prisión y el lugar donde se encontrare;
    f) A entrevistarse privadamente con su abogado de acuerdo al régimen del establecimiento de detención, el que sólo contemplará las restricciones necesarias para el mantenimiento del orden y la seguridad del recinto;
    g) A tener, a sus expensas, las comodidades y ocupaciones compatibles con la seguridad del recinto en que se encontrare, y
    h) A recibir visitas y comunicarse por escrito o por cualquier otro medio, salvo lo dispuesto en el artículo 151.
    Artículo 95.- Amparo ante el juez de garantía. Toda persona privada de libertad tendrá derecho a ser conducida sin demora ante un juez de garantía, con el objeto de que examine la legalidad de su privación de libertad y, en todo caso, para que examine las condiciones en que se encontrare, constituyéndose, si fuere necesario, en el lugar en que ella estuviere. El juez podrá ordenar la libertad del afectado o adoptar las medidas que fueren procedentes.
    El abogado de la persona privada de libertad, sus parientes o cualquier persona en su nombre podrán siempre ocurrir ante el juez que conociere del caso o aquél del lugar donde aquélla se encontrare, para solicitar que ordene que sea conducida a su presencia y se ejerzan las facultades establecidas en el inciso anterior.
    Con todo, si la privación de libertad hubiere sido ordenada por resolución judicial, su legalidad sólo podrá impugnarse por los medios procesales que correspondan ante el tribunal que la hubiere dictado, sin perjuicio de lo establecido en el artículo 21 de la Constitución Política de la República.
    Artículo 96.- Derechos de los abogados. Todo abogado tendrá derecho a requerir del funcionario encargado de cualquier lugar de detención o prisión, la confirmación de encontrarse privada de libertad una persona determinada en ese o en otro establecimiento del mismo servicio y que se ubicare en la comuna.
    En caso afirmativo y con el acuerdo del afectado, el abogado tendrá derecho a conferenciar privadamente con él y, con su consentimiento, a recabar del encargado del establecimiento la información consignada en la letra a) del artículo 94.
    Si fuere requerido, el funcionario encargado deberá extender, en el acto, una constancia de no encontrarse privada de libertad en el establecimiento la persona por la que se hubiere consultado.
    Artículo 97.- Obligación de cumplimiento e información. El tribunal, los fiscales y los funcionarios policiales dejarán constancia en los respectivos registros, conforme al avance del procedimiento, de haber cumplido las normas legales que establecen los derechos y garantías del imputado.
    Artículo 98.- Declaración del imputado como medio de defensa. Durante todo el procedimiento y en cualquiera de sus etapas el imputado tendrá siempre derecho a prestar declaración, como un medio de defenderse de la imputación que se le dirigiere.
    La declaración judicial del imputado se prestará en audiencia a la cual podrán concurrir los intervinientes en el procedimiento, quienes deberán ser citados al efecto.
    La declaración del imputado no podrá recibirse bajo juramento. El juez o, en su caso, el presidente del tribunal, se limitará a exhortarlo a que diga la verdad y a que responda con claridad y precisión las preguntas que se le formularen. Regirá, correspondientemente, lo dispuesto en el artículo 326.
    Si con ocasión de su declaración judicial, el imputado o su defensor solicitaren la práctica de diligencias de investigación, el juez podrá recomendar al ministerio público la realización de las mismas, cuando lo considerare necesario para el ejercicio de la defensa y el respeto del principio de objetividad.
    Si el imputado no supiere la lengua castellana o si fuere sordo o mudo, se procederá a tomarle declaración de conformidad al artículo 291, incisos tercero y cuarto.
    II.- Imputado rebelde
    Artículo 99.- Causales de rebeldía. El imputado será declarado rebelde:
    a) Cuando, decretada judicialmente su detención o prisión preventiva, no fuere habido, o
    b) Cuando, habiéndose formalizado la investigación en contra del que estuviere en país extranjero, no fuere posible obtener su extradición.
    Artículo 100.- Declaración de rebeldía. La declaración de rebeldía del imputado será pronunciada por el tribunal ante el que debiere comparecer.
    Artículo 101.- Efectos de la rebeldía. Declarada la rebeldía, las resoluciones que se dictaren en el procedimiento se tendrán por notificadas personalmente al rebelde en la misma fecha en que se pronunciaren.
    La investigación no se suspenderá por la declaración de rebeldía y el procedimiento continuará hasta la realización de la audiencia de preparación del juicio oral, en la cual se podrá sobreseer definitiva o temporalmente la causa de acuerdo al mérito de lo obrado. Si la declaración de rebeldía se produjere durante la etapa de juicio oral, el procedimiento se sobreseerá temporalmente, hasta que el imputado compareciere o fuere habido.
    El sobreseimiento afectará sólo al rebelde y el procedimiento continuará con respecto a los imputados presentes.
    El imputado que fuere habido pagará las costas causadas con su rebeldía, a menos que justificare debidamente su ausencia.
    Párrafo 5º La defensa
    Artículo 102.- Derecho a designar libremente a un defensor. Desde la primera actuación del procedimiento y hasta la completa ejecución de la sentencia que se dictare, el imputado tendrá derecho a designar libremente uno o más defensores de su confianza. Si no lo tuviere, el juez procederá a hacerlo, en los términos que señale la ley respectiva. En todo caso, la designación del defensor deberá tener lugar antes de la realización de la primera audiencia a que fuere citado el imputado.
    Si el imputado se encontrare privado de libertad, cualquier persona podrá proponer para aquél un defensor determinado, o bien solicitar se le nombre uno. Conocerá de dicha petición el juez de garantía competente o aquél correspondiente al lugar en que el imputado se encontrare.
    El juez dispondrá la comparecencia del imputado a su presencia, con el objeto de que acepte la designación del defensor.
    Si el imputado prefiriere defenderse personalmente, el tribunal lo autorizará sólo cuando ello no perjudicare la eficacia de la defensa; en caso contrario, le designará defensor letrado, sin perjuicio del derecho del imputado a formular planteamientos y alegaciones por sí mismo, según lo dispuesto en el artículo 8º.

    Artículo 103.- Efectos de la ausencia del defensor. La ausencia del defensor en cualquier actuación en que la ley exigiere expresamente su participación acarreará la nulidad de la misma, sin perjuicio de lo señalado en el artículo 286.
    Artículo 103 bis.- Sanciones al defensor que no asistiere o abandonare la audiencia injustificadamente. La ausencia injustificada del defensor a la audiencia del juicio oral, a la de preparación del mismo o del procedimiento abreviado, como asimismo a cualquiera de las sesiones de éstas, si se desarrollaren en varias, se sancionará con la suspensión del ejercicio de la profesión, la que no podrá ser inferior a quince ni superior a sesenta días. En idéntica sanción incurrirá el defensor que abandonare injustificadamente alguna de las mencionadas audiencias, mientras éstas se estuvieren desarrollando.
    El tribunal impondrá la sanción después de escuchar al afectado y recibir la prueba que ofreciere, si la estimare procedente.


    Artículo 104.- Derechos y facultades del defensor. El defensor podrá ejercer todos los derechos y facultades que la ley reconoce al imputado, a menos que expresamente se reservare su ejercicio a este último en forma personal.
    Artículo 105.- Defensa de varios imputados en un mismo proceso. La defensa de varios imputados podrá ser asumida por un defensor común, a condición de que las diversas posiciones que cada uno de ellos sustentare no fueren incompatibles entre sí.
    Si el tribunal advirtiere una situación de incompatibilidad la hará presente a los afectados y les otorgará un plazo para que la resuelvan o para que designen los defensores que se requirieren a fin de evitar la incompatibilidad de que se tratare. Si, vencido el plazo, la situación de incompatibilidad no hubiere sido resuelta o no hubieren sido designados el o los defensores necesarios, el mismo tribunal determinará los imputados que debieren considerarse sin defensor y procederá a efectuar los nombramientos que correspondieren.
    Artículo 106.- Renuncia o abandono de la defensa. La renuncia formal del defensor no lo liberará de su deber de realizar todos los actos inmediatos y urgentes que fueren necesarios para impedir la indefensión del imputado.
    Sin perjuicio de lo anterior, no podrá ser presentada la mencionada renuncia del abogado defensor dentro de los diez días previos a la realización de la audiencia de juicio oral, como tampoco dentro de los siete días previos a la realización de la audiencia de preparación de juicio.
    El abogado defensor que renunciare a su cargo en los plazos señalados en el inciso anterior, o abandonare o dejare de asistir injustificadamente a las audiencias mencionadas en el artículo 103 bis, será sancionado con la suspensión del ejercicio de la profesión en los términos previstos en el citado precepto.
    En el caso de renuncia del defensor o en cualquier situación de abandono de hecho de la defensa, el tribunal deberá designar de oficio un defensor penal público que la asuma, a menos que el imputado se procurare antes un defensor de su confianza. Con todo, tan pronto este defensor hubiere aceptado el cargo, cesará en sus funciones el designado por el tribunal.

    Artículo 107.- Designación posterior. La designación de un defensor penal público no afectará el derecho del imputado a elegir posteriormente otro de su confianza; pero la sustitución no producirá efectos hasta que el defensor designado aceptare el mandato y fijare domicilio.
    Párrafo 6º La víctima
    Artículo 108.- Concepto. Para los efectos de este Código, se considera víctima al ofendido por el delito.
    En los delitos cuya consecuencia fuere la muerte del ofendido y en los casos en que éste no pudiere ejercer los derechos que en este Código se le otorgan, se considerará víctima:
    a) al cónyuge o al conviviente civil y a los hijos;
    b) a los ascendientes;
    c) al conviviente;
    d) a los hermanos, y
    e) al adoptado o adoptante.
    Para los efectos de su intervención en el procedimiento, la enumeración precedente constituye un orden de prelación, de manera que la intervención de una o más personas pertenecientes a una categoría excluye a las comprendidas en las categorías siguientes.

    Artículo 109.- Derechos de la víctima. La víctima podrá intervenir en el procedimiento penal conforme a lo establecido en este Código, y tendrá, entre otros, los siguientes derechos:
    a) Solicitar medidas de protección frente a probables hostigamientos, amenazas o atentados en contra suya o de su familia;
    b) Presentar querella;
    c) Ejercer contra el imputado acciones tendientes a perseguir las responsabilidades civiles provenientes del hecho punible;
    d) Ser oída, si lo solicitare, por el fiscal antes de que éste pidiere o se resolviere la suspensión del procedimiento o su terminación anticipada;
    e) Ser oída, si lo solicitare, por el tribunal antes de pronunciarse acerca del sobreseimiento temporal o definitivo u otra resolución que pusiere término a la causa, y
    f) Impugnar el sobreseimiento temporal o definitivo o la sentencia absolutoria, aun cuando no hubiere intervenido en el procedimiento.
    Tratándose de los delitos previstos en el Código Penal, en los artículos 141, inciso final; 142, inciso final; 150 A; 150 D; 361; 362; 363; 365 bis; 366, incisos primero y segundo; 366 bis; 366 quáter; 367; 367 ter; 372 bis; 411 quáter, cuando se cometan con fines de explotación sexual, y 433, número 1, en relación con la violación, así como también cualquier delito sobre violencia en contra de las mujeres, las víctimas tendrán además derecho a:
     
    a) Contar con acceso a asistencia y representación judicial.
    b) No ser enjuiciada, estigmatizada, discriminada ni cuestionada por su relato, conductas o estilo de vida.
    c) Obtener una respuesta oportuna, efectiva y justificada.
    d) Que se realice una investigación con debida diligencia desde un enfoque intersectorial, incorporando la perspectiva de género y de derechos humanos.
    e) Recibir protección a través de las medidas contempladas en la legislación, cuando se encuentre amenazada o vulnerada su vida, integridad física, indemnidad sexual o libertad personal.
    f) La protección de sus datos personales y los de sus hijas e hijos, respecto de terceros, y de su intimidad, honra y seguridad, para lo cual el tribunal que conozca del respectivo procedimiento podrá disponer las medidas que sean pertinentes.
    g) Participar en el procedimiento recibiendo información clara, oportuna y completa de la causa. En particular, podrán obtener información de la causa personalmente, sin necesidad de requerir dicha información a través de un abogado.
    h) Que se adopten medidas para prevenir la victimización secundaria, esto es, evitar toda consecuencia negativa que puedan sufrir con ocasión de su interacción en el proceso penal. Con dicha finalidad, la denuncia debe ser recibida en condiciones que garanticen el respeto por su seguridad, privacidad y dignidad. La negativa o renuencia a recibir la denuncia se considerará una infracción grave al principio de probidad administrativa.
    Asimismo, tendrá derecho a que su declaración sea recibida en el tiempo más próximo desde la denuncia, por personal capacitado de Carabineros de Chile, de la Policía de Investigaciones de Chile o del Ministerio Público y cuente con el soporte necesario para evitar que vuelva a realizarse durante la etapa de investigación, a menos que ello sea indispensable para el esclarecimiento de los hechos o que la propia víctima lo requiera. La declaración judicial deberá ser recibida por jueces capacitados, y se garantizará en los casos referidos, el respeto por la seguridad, privacidad y dignidad de la víctima.
    Los derechos precedentemente señalados no podrán ser ejercidos por quien fuere imputado del delito respectivo, sin perjuicio de los derechos que le correspondieren en esa calidad.

    Artículo 109 bis.- Medidas de protección especiales para víctimas de delitos de violencia sexual. En los delitos contemplados en el Código Penal, en los artículos 141, inciso final; 142, inciso final; 150 A; 150 D; 361; 362; 363; 365 bis; 366, incisos primero y segundo; 366 bis; 366 quáter; 367; 367 ter; 372 bis; 411 quáter, cuando se cometan con fines de explotación sexual, y 433, número 1, en relación con la violación, el juez de garantía y el tribunal de juicio oral en lo penal, de oficio o a petición de alguno de los intervinientes, deberá adoptar una o más de las siguientes medidas para proteger la identidad, intimidad, integridad física, sexual y psíquica de la o las víctimas:
     
    a) Suprimir de las actas de las audiencias todo nombre, dirección o cualquier otra información que pudiera servir para identificar a las víctimas, sus familiares o testigos, directa o indirectamente.
    b) Prohibir a los intervinientes que entreguen información o formulen declaraciones a los medios de comunicación social relativas a la identidad de la o las víctimas, a menos que ellas consientan de manera libre e informada en dar a conocer su identidad.
    c) Impedir el acceso de personas determinadas o del público en general a la sala de audiencia, y ordenar su salida de ella, si alguna de las víctimas lo solicita.
    d) Prohibir a los medios de comunicación social el acceso a la sala de audiencia, si alguna de las víctimas lo solicita.
    e) Decretar alguna de las medidas establecidas en el artículo 308 para favorecer su declaración judicial.
     
    El Ministerio Público y los tribunales de justicia deberán tomar todas las medidas que correspondan para impedir la identificación de la o las víctimas por parte de terceras personas ajenas al proceso penal, a menos que ellas consientan de manera libre e informada en dar a conocer su identidad.
    Artículo 109 ter.- Deber de prevención de la victimización secundaria. Las personas e instituciones que intervienen en el proceso penal, en sus etapas de denuncia, investigación y juzgamiento tienen el deber de prevenir la victimización secundaria, esto es, evitar toda consecuencia negativa que puedan sufrir las víctimas con ocasión de su interacción en el proceso penal.
    Anualmente Carabineros de Chile, la Policía de Investigaciones de Chile, el Servicio Médico Legal, el Ministerio de Salud, el Ministerio Público, la Defensoría Penal Pública y el Poder Judicial realizarán planes de formación y perfeccionamiento que aborden la prevención de la victimización secundaria y la perspectiva de género en el proceso penal y fomenten una protección especial de las víctimas de violencia de género.
    Artículo 110.- Información a personas que no hubieren intervenido en el procedimiento. En los casos a que se refiere el inciso segundo del artículo 108, si ninguna de las personas enunciadas en ese precepto hubiere intervenido en el procedimiento, el ministerio público informará sus resultados al cónyuge del ofendido por el delito o, en su defecto, a alguno de los hijos u otra de esas personas.

    Artículo 110 bis.- Designación de curador ad litem. En los casos en que las víctimas menores de edad de los delitos establecidos en los Párrafos 5, 6 y 6 bis del Título VII del Libro Segundo, y en los artículos 141, incisos cuarto y quinto; 142; 372 bis; 390; 391; 395; 397, número 1; 411 bis; 411 ter; 411 quáter, y 433, número 1, todos del Código Penal, carezcan de representante legal o cuando, por motivos fundados, se estimare que sus intereses son independientes o contradictorios con los de aquel a quien corresponda representarlos, el juez podrá designarles un curador ad litem de cualquier institución que se dedique a la defensa, promoción o protección de los derechos de la infancia.



NOTA
      El Art. primero transitorio de la ley 21.057, establece que la modificación introducida al presente artículo comenzará a regir de manera gradual, en plazos contados desde la publicación del Reglamento: Primera etapa: seis meses después, respecto de las regiones XV, I, II, VII, XI y XII. Segunda etapa: dieciocho meses después, respecto de las regiones III, IV, VIII, IX y XIV.  Tercera etapa:  treinta meses después, comprendiendo las regiones V, VI, X y Metropolitana.
    Párrafo 7º El querellante
    Artículo 111.- Querellante. La querella podrá ser interpuesta por la víctima, su representante legal o su heredero testamentario.
    También se podrá querellar cualquier persona capaz de parecer en juicio domiciliada en la provincia, respecto de hechos punibles cometidos en la misma que constituyeren delitos terroristas, o delitos cometidos por un funcionario público que afectaren derechos de las personas garantizados por la Constitución o contra la probidad pública.
    Los órganos y servicios públicos sólo podrán interponer querella cuando sus respectivas leyes orgánicas les otorguen expresamente las potestades correspondientes.


    Artículo 112.- Oportunidad para presentar la querella. La querella podrá presentarse en cualquier momento, mientras el fiscal no declarare cerrada la investigación.
    Admitida a tramitación, el juez la remitirá al ministerio público y el querellante podrá hacer uso de los derechos que le confiere el artículo 261.
    Artículo 113.- Requisitos de la querella. Toda querella criminal deberá presentarse por escrito ante el juez de garantía y deberá contener:
    a) La designación del tribunal ante el cual se entablare;
    b) El nombre, apellido, profesión u oficio y domicilio del querellante;
    c) El nombre, apellido, profesión u oficio y residencia del querellado, o una designación clara de su persona, si el querellante ignorare aquellas circunstancias. Si se ignoraren dichas determinaciones, siempre se podrá deducir querella para que se proceda a la investigación del delito y al castigo de el o de los culpables;
    d) La relación circunstanciada del hecho, con expresión del lugar, año, mes, día y hora en que se hubiere ejecutado, si se supieren;
    e) La expresión de las diligencias cuya práctica se solicitare al ministerio público, y
    f) La firma del querellante o la de otra persona a su ruego, si no supiere o no pudiere firmar.

    Artículo 114.- Inadmisibilidad de la querella. La querella no será admitida a tramitación por el juez de garantía:
    a) Cuando fuere presentada extemporáneamente, de acuerdo a lo establecido en el artículo 112;
    b) Cuando, habiéndose otorgado por el juez de garantía un plazo de tres días para subsanar los defectos que presentare por falta de alguno de los requisitos señalados en el artículo 113, el querellante no realizare las modificaciones pertinentes dentro de dicho plazo;
    c) Cuando los hechos expuestos en ella no fueren constitutivos de delito;
    d) Cuando de los antecedentes contenidos en ella apareciere de manifiesto que la responsabilidad penal del imputado se encuentra extinguida. En este caso, la declaración de inadmisibilidad se realizará previa citación del ministerio público, y
    e) Cuando se dedujere por persona no autorizada por la ley.
    Artículo 115.- Apelación de la resolución. La resolución que declarare inadmisible la querella será apelable, pero sin que en la tramitación del recurso pueda disponerse la suspensión del procedimiento.
    La resolución que admitiere a tramitación la querella será inapelable.
    Artículo 116.- Prohibición de querella. No podrán querellarse entre sí, sea por delitos de acción pública o privada:
    a) Los cónyuges, a no ser por delito que uno hubiere cometido contra el otro o contra sus hijos, o por el delito de bigamia.
    b) Los convivientes civiles, a no ser por delito que uno hubiere cometido contra el otro o contra sus hijos.
    c) Los consanguíneos en toda la línea recta, los colaterales y afines hasta el segundo grado, a no ser por delitos cometidos por unos contra los otros, o contra su cónyuge o hijos.

    Artículo 117.- Querella rechazada. Cuando no se diere curso a una querella en que se persiguiere un delito de acción pública o previa instancia particular, por aplicación de alguna de las causales previstas en las letras a) y b) del artículo 114, el juez la pondrá en conocimiento del ministerio público para ser tenida como denuncia, siempre que no le constare que la investigación del hecho hubiere sido iniciada de otro modo.
    Artículo 118.- Desistimiento. El querellante podrá desistirse de su querella en cualquier momento del procedimiento. En ese caso, tomará a su cargo las costas propias y quedará sujeto a la decisión general sobre costas que dictare el tribunal al finalizar el procedimiento.
    Artículo 119.- Derechos del querellado frente al desistimiento. El desistimiento de la querella dejará a salvo el derecho del querellado para ejercer, a su vez, la acción penal o civil a que dieren lugar la querella o acusación calumniosa, y a demandar los perjuicios que le hubiere causado en su persona o bienes y las costas.
    Se exceptúa el caso en que el querellado hubiere aceptado expresamente el desistimiento del querellante.
    Artículo 120.- Abandono de la querella. El tribunal, de oficio o a petición de cualquiera de los intervinientes, declarará abandonada la querella por quien la hubiere interpuesto:
    a) Cuando no adhiriere a la acusación fiscal o no acusare particularmente en la oportunidad que correspondiere;
    b) Cuando no asistiere a la audiencia de preparación del juicio oral sin causa debidamente justificada, y
    c) Cuando no concurriere a la audiencia del juicio oral o se ausentare de ella sin autorización del tribunal.
    La resolución que declarare el abandono de la querella será apelable, sin que en la tramitación del recurso pueda disponerse la suspensión del procedimiento. La resolución que negare lugar al abandono será inapelable.
    Artículo 121.- Efectos del abandono. La declaración del abandono de la querella impedirá al querellante ejercer los derechos que en esa calidad le confiere este Código.
    Título V
    Medidas cautelares personales

    Párrafo 1º Principio general
    Artículo 122.- Finalidad y alcance. Las medidas cautelares personales sólo serán impuestas cuando fueren absolutamente indispensables para asegurar la realización de los fines del procedimiento y sólo durarán mientras subsistiere la necesidad de su aplicación.
    Estas medidas serán siempre decretadas por medio de resolución judicial fundada.
    Párrafo 2º Citación
    Artículo 123.- Oportunidad de la citación judicial. Cuando fuere necesaria la presencia del imputado ante el tribunal, éste dispondrá su citación, de acuerdo con lo previsto en el artículo 33.
    Artículo 124. Exclusión de otras medidas. Cuando la imputación se refiriere a faltas, o delitos que la ley no sancionare con penas privativas ni restrictivas de libertad, no se podrán ordenar medidas cautelares que recaigan sobre la libertad del imputado, con excepción de la citación.
    Lo dispuesto en el inciso anterior no tendrá lugar en los casos a que se refiere el inciso cuarto del artículo 134 o cuando procediere el arresto por falta de comparecencia, la detención o la prisión preventiva de acuerdo a lo dispuesto en el artículo 33.

    Artículo 124 bis.- Tratándose del caso previsto en los párrafos tercero y final del numeral 6 del artículo 10 del Código Penal, no se podrán ordenar medidas cautelares que recaigan sobre la libertad del imputado, con excepción de la citación y las medidas cautelares previstas en los literales d) y g) del artículo 155. Lo anterior, no será aplicable si en el curso de la investigación surgen antecedentes calificados que justifiquen la existencia de un delito.


    Párrafo 3º Detención
    Artículo 125.- Procedencia de la detención. Ninguna persona podrá ser detenida sino por orden de funcionario público expresamente facultado por la ley y después que dicha orden le fuere intimada en forma legal, a menos que fuere sorprendida en delito flagrante y, en este caso, para el único objeto de ser conducida ante la autoridad que correspondiere.
    Artículo 126.- Presentación voluntaria del imputado. El imputado contra quien se hubiere emitido orden de detención por cualquier autoridad competente podrá ocurrir siempre ante el juez que correspondiere a solicitar un pronunciamiento sobre su procedencia o la de cualquier otra medida cautelar.
    Artículo 127.- Detención judicial. Salvo en los casos contemplados en el artículo 124, el tribunal, a solicitud del ministerio público, podrá ordenar la detención del imputado para ser conducido a su presencia, sin previa citación, cuando de otra manera la comparecencia pudiera verse demorada o dificultada.
    Además, podrá decretarse la detención del imputado por un hecho al que la ley asigne una pena privativa de libertad de crimen.
    Tratándose de hechos a los que la ley asigne las penas de crimen o simple delito, el juez podrá considerar como razón suficiente para ordenar la detención la circunstancia de que el imputado haya concurrido voluntariamente ante el fiscal o la policía, y reconocido voluntariamente su participación en ellos.
    También se decretará la detención del imputado cuya presencia en una audiencia judicial fuere condición de ésta y que, legalmente citado, no compareciere sin causa justificada.
    La resolución que denegare la orden de detención será susceptible del recurso de apelación por el Ministerio Público.

    Artículo 128.- Detención por cualquier tribunal. Todo tribunal, aunque no ejerza jurisdicción en lo criminal, podrá dictar órdenes de detención contra las personas que, dentro de la sala de su despacho, cometieren algún crimen o simple delito, conformándose a las disposiciones de este Título.
    Artículo 129.- Detención en caso de flagrancia. Cualquier persona podrá detener a quien sorprendiere en delito flagrante, debiendo entregar inmediatamente al aprehendido a la policía, al ministerio público o a la autoridad judicial más próxima.
    Los agentes policiales estarán obligados a detener a quienes sorprendieren in fraganti en la comisión de un delito. En el mismo acto, la policía podrá proceder al registro de las vestimentas, equipaje o vehículo de la persona detenida, debiendo cumplir con lo señalado en el inciso segundo del artículo 89 de este Código.
    No obstará a la detención la circunstancia de que la persecución penal requiriere instancia particular previa, si el delito flagrante fuere de aquellos previstos y sancionados en los artículos 361 a 366 quater del Código Penal.
    La policía deberá, asimismo, detener al sentenciado a penas privativas de libertad que hubiere quebrantado su condena, al que se fugare estando detenido, al que tuviere orden de detención pendiente, a quien fuere sorprendido en violación flagrante de las medidas cautelares personales que se le hubieren impuesto, al que fuere sorprendido infringiendo las condiciones impuestas en virtud de las letras a), b), c) y d) del artículo 17 ter de la ley Nº 18.216 y al que violare la condición del artículo 238, letra b), que le hubiere sido impuesta para la protección de otras personas.
    Sin perjuicio de lo señalado en el inciso anterior, el tribunal que correspondiere deberá, en caso de quebrantamiento de condena y tan pronto tenga conocimiento del mismo, despachar la respectiva orden de detención en contra del condenado.
    En los casos de que trata este artículo, la policía podrá ingresar a un lugar cerrado, mueble o inmueble, cuando se encontrare en actual persecución del individuo a quien debiere detener, para practicar la respectiva detención. En este caso, la policía podrá registrar el lugar e incautar los objetos y documentos vinculados al caso que dio origen a la persecución, dando aviso de inmediato al fiscal, quien los conservará. Lo anterior procederá sin perjuicio de lo establecido en el artículo 215.



    Artículo 130.- Situación de flagrancia. Se entenderá que se encuentra en situación de flagrancia:
    a) El que actualmente se encontrare cometiendo el delito;
    b) El que acabare de cometerlo;
    c) El que huyere del lugar de comisión del delito y fuere designado por el ofendido u otra persona como autor o cómplice;
    d) El que, en un tiempo inmediato a la perpetración de un delito, fuere encontrado con objetos procedentes de aquél o con señales, en sí mismo o en sus vestidos, que permitieren sospechar su participación en él, o con las armas o instrumentos que hubieren sido empleados para cometerlo, y
    e) El que las víctimas de un delito que reclamen auxilio, o testigos presenciales, señalaren como autor o cómplice de un delito que se hubiere cometido en un tiempo inmediato.
    f) El que aparezca en un registro audiovisual cometiendo un crimen o simple delito al cual la policía tenga acceso en un tiempo inmediato.
    Para los efectos de lo establecido en las letras d), e) y f) se entenderá por tiempo inmediato todo aquel que transcurra entre la comisión del hecho y la captura del imputado, siempre que no hubieren transcurrido más de doce horas.




    Artículo 131.- Plazos de la detención. Cuando la detención se practicare en cumplimiento de una orden judicial, los agentes policiales que la hubieren realizado o el encargado del recinto de detención conducirán inmediatamente al detenido a presencia del juez que hubiere expedido la orden. Si ello no fuere posible por no ser hora de despacho, el detenido podrá permanecer en el recinto policial o de detención hasta el momento de la primera audiencia judicial, por un período que en caso alguno excederá las veinticuatro horas.
    Cuando la detención se practicare en virtud de los artículos 129 y 130, el agente policial que la hubiere realizado o el encargado del recinto de detención deberán informar de ella al ministerio público dentro de un plazo máximo de doce horas. El fiscal podrá dejar sin efecto la detención u ordenar que el detenido sea conducido ante el juez dentro de un plazo máximo de veinticuatro horas, contado desde que la detención se hubiere practicado. Si el fiscal nada manifestare, la policía deberá presentar el detenido ante la autoridad judicial en el plazo indicado.
    Cuando el fiscal ordene poner al detenido a disposición del juez, deberá, en el mismo acto, dar conocimiento de esta situación al abogado de confianza de aquél o a la Defensoría Penal Pública.
    Para los efectos de poner a disposición del juez al detenido, las policías cumplirán con su obligación legal dejándolo bajo la custodia de Gendarmería del respectivo tribunal.

      Artículo 132. Comparecencia judicial. A la primera audiencia judicial del detenido deberá concurrir el fiscal o el abogado asistente del fiscal. La ausencia de éstos dará lugar a la liberación del detenido. No obstante lo anterior, el juez podrá suspender la audiencia por un plazo breve y perentorio no superior a dos horas, con el fin de permitir la concurrencia del fiscal o su abogado asistente. Transcurrido este plazo sin que concurriere ninguno de ellos, se procederá a la liberación del detenido.
    En todo caso, el juez deberá comunicar la ausencia del fiscal o de su abogado asistente al fiscal regional respectivo a la mayor brevedad, con el objeto de determinar la eventual responsabilidad disciplinaria que correspondiere.
    En la audiencia, el fiscal o el abogado asistente del fiscal actuando expresamente facultado por éste, procederá directamente a formalizar la investigación y a solicitar las medidas cautelares que procedieren, siempre que contare con los antecedentes necesarios y que se encontrare presente el defensor del imputado. En el caso de que no pudiere procederse de la manera indicada, el fiscal o el abogado asistente del fiscal actuando en la forma señalada, podrá solicitar una ampliación del plazo de detención hasta por tres días, con el fin de preparar su presentación. El juez accederá a la ampliación del plazo de detención cuando estimare que los antecedentes justifican esa medida.
    En todo caso, la declaración de ilegalidad de la detención no impedirá que el fiscal o el abogado asistente del fiscal pueda formalizar la investigación y solicitar las medidas cautelares que sean procedentes, de conformidad con lo dispuesto en el inciso anterior, pero no podrá solicitar la ampliación de la detención. La declaración de ilegalidad de la detención no producirá efecto de cosa juzgada en relación con las solicitudes de exclusión de prueba que se hagan oportunamente, de conformidad con lo previsto en el artículo 276.




    Artículo 132 bis.- Apelación de la resolución que declara la ilegalidad de la detención. Tratándose de los delitos establecidos en los artículos 141, 142, 361, 362, 365 bis, 390, 390 bis, 390 ter, 391, 433, 436 y 440 del Código Penal, en las leyes N°17.798 y N°20.000 que tengan penas de crimen o simple delito, y de los delitos de castración, mutilaciones y lesiones contra miembros de Carabineros, de la Policía de Investigaciones y de Gendarmería de Chile, en el ejercicio de sus funciones, la resolución que declare la ilegalidad de la detención será apelable por el fiscal o el abogado asistente del fiscal en el solo efecto devolutivo. En los demás casos no será apelable.

    Artículo 133.- Ingreso de personas detenidas. Los encargados de los establecimientos penitenciarios no podrán aceptar el ingreso de personas sino en virtud de órdenes judiciales.
    Artículo 134.- Citación, registro y detención en casos de flagrancia. Quien fuere sorprendido por la policía in fraganti cometiendo un hecho de los señalados en el artículo 124, será citado a la presencia del fiscal, previa comprobación de su domicilio.
    La policía podrá registrar las vestimentas, el equipaje o el vehículo de la persona que será citada.
    Asimismo, podrá conducir al imputado al recinto policial, para efectuar allí la citación.
    No obstante lo anterior, el imputado podrá ser detenido si hubiere cometido alguna de las faltas contempladas en el Código Penal, en los artículos 494, N°s. 4 y 5, y 19, exceptuando en este último caso los hechos descritos en los artículos 189 y 233; 494 bis, 495 N° 21, y 496, Nos. 3, 5 y 26.
    En todos los casos señalados en el inciso anterior, el agente policial deberá informar al fiscal, de inmediato, de la detención, para los  efectos de lo dispuesto en el inciso segundo del artículo 131. El fiscal comunicará su decisión al defensor en el momento que la adopte.
    El procedimiento indicado en el inciso primero podrá ser utilizado asimismo cuando, tratándose de un simple delito y no siendo posible conducir al imputado inmediatamente ante el juez, el funcionario a cargo del recinto policial considerare que existen suficientes garantías de su oportuna comparecencia.




    Artículo 135.- Información al detenido. El funcionario público a cargo del procedimiento de detención deberá informar al afectado acerca del motivo de la detención, al momento de practicarla.
    Asimismo, le informará acerca de los derechos establecidos en los artículos 93, letras a), b) y g), y 94, letras f) y g), de este Código. Con todo, si, por las circunstancias que rodearen la detención, no fuere posible proporcionar inmediatamente al detenido la información prevista en este inciso, ella le será entregada por el encargado de la unidad policial a la cual fuere conducido. Se dejará constancia en el libro de guardia del recinto policial del hecho de haberse proporcionado la información, de la forma en que ello se hubiere realizado, del funcionario que la hubiere entregado y de las personas que lo hubieren presenciado.
    La información de derechos prevista en el inciso anterior podrá efectuarse verbalmente, o bien por escrito, si el detenido manifestare saber leer y encontrarse en condiciones de hacerlo. En este último caso, se le entregará al detenido un documento que contenga una descripción clara de esos derechos, cuyo texto y formato determinará el ministerio público.
    En los casos comprendidos en el artículo 138, la información prevista en los incisos precedentes será entregada al afectado en el lugar en que la detención se hiciere efectiva, sin perjuicio de la constancia respectiva en el libro de guardia.
    Artículo 136.- Fiscalización del cumplimiento del deber de información. El fiscal y, en su caso, el juez, deberán cerciorarse del cumplimiento de lo previsto en el artículo precedente. Si comprobaren que ello no hubiere ocurrido, informarán de sus derechos al detenido y remitirán oficio, con los antecedentes respectivos, a la autoridad competente, con el objeto de que aplique las sanciones disciplinarias correspondientes o inicie las investigaciones penales que procedieren.
    Artículo 137. Difusión de derechos. En todo recinto policial, de los juzgados de garantía, de los tribunales de juicio oral en lo penal, del Ministerio Público y de la Defensoría Penal Pública, deberá exhibirse en lugar destacado y claramente visible al público, un cartel en el cual se consignen los derechos de las víctimas y aquéllos que les asisten a las personas que son detenidas. Asimismo, en todo recinto de detención policial y casa de detención deberá exhibirse un cartel en el cual se consignen los derechos de los detenidos. El texto y formato de estos carteles serán determinados por el Ministerio de Justicia.

    Artículo 138.- Detención en la residencia del imputado. La detención del que se encontrare en los casos previstos en el párrafo segundo del número 6º del artículo 10 del Código Penal se hará efectiva en su residencia. Si el detenido tuviere su residencia fuera de la ciudad donde funcionare el tribunal competente, la detención se hará efectiva en la residencia que aquél señalare dentro de la ciudad en que se encontrare el tribunal.
    Párrafo 4º Prisión preventiva
    Artículo 139.- Procedencia de la prisión preventiva. Toda persona tiene derecho a la libertad personal y a la seguridad individual.
    La prisión preventiva procederá cuando las demás medidas cautelares personales fueren estimadas por el juez como insuficientes para asegurar las finalidades del procedimiento, la seguridad del ofendido o de la sociedad.

    Artículo 140.- Requisitos para ordenar la prisión preventiva. Una vez formalizada la investigación, el tribunal, a petición del Ministerio Público o del querellante, podrá decretar la prisión preventiva del imputado siempre que el solicitante acreditare que se cumplen los siguientes requisitos:

    a) Que existen antecedentes que justificaren la existencia del delito que se investigare;
    b) Que existen antecedentes que permitieren presumir fundadamente que el imputado ha tenido participación en el delito como autor, cómplice o encubridor, y
    c) Que existen antecedentes calificados que permitieren al tribunal considerar que la prisión preventiva es indispensable para el éxito de diligencias precisas y determinadas de la investigación, o que la libertad del imputado es peligrosa para la seguridad de la sociedad o del ofendido, o que existe peligro de que el imputado se dé a la fuga, conforme a las disposiciones de los incisos siguientes.
    Se entenderá especialmente que la prisión preventiva es indispensable para el éxito de la investigación cuando existiere sospecha grave y fundada de que el imputado pudiere obstaculizar la investigación mediante la destrucción, modificación, ocultación o falsificación de elementos de prueba; o cuando pudiere inducir a coimputados, testigos, peritos o terceros para que informen falsamente o se comporten de manera desleal o reticente.
    Para estimar si la libertad del imputado resulta o no peligrosa para la seguridad de la sociedad, el tribunal deberá considerar especialmente alguna de las siguientes circunstancias: la gravedad de la pena asignada al delito; el número de delitos que se le imputare y el carácter de los mismos; la existencia de procesos pendientes, y el hecho de haber actuado en grupo o pandilla.
    Se entenderá especialmente que la libertad del imputado constituye un peligro para la seguridad de la sociedad, cuando los delitos imputados tengan asignada pena de crimen en la ley que los consagra; cuando el imputado hubiere sido condenado con anterioridad por delito al que la ley señale igual o mayor pena, sea que la hubiere cumplido efectivamente o no; cuando los delitos imputados consistieren en atentados contra la vida o la integridad física de miembros de Carabineros de Chile, de la Policía de Investigaciones de Chile, funcionarios de las Fuerzas Armadas y de los servicios de su dependencia o de Gendarmería de Chile en razón de su cargo o con motivo u ocasión del ejercicio de sus funciones, que tengan asignada una pena igual o superior a la de presidio menor en su grado máximo en la ley que los consagra; cuando se encontrare sujeto a alguna medida cautelar personal como orden de detención judicial pendiente u otras, en libertad condicional o gozando de alguno de los beneficios alternativos a la ejecución de las penas privativas o restrictivas de libertad contemplados en la ley.
    Se entenderá que la seguridad del ofendido se encuentra en peligro por la libertad del imputado cuando existieren antecedentes calificados que permitieren presumir que éste realizará atentados en contra de aquél, o en contra de su familia o de sus bienes.
    Para efectos del inciso cuarto, sólo se considerarán aquellas órdenes de detención pendientes que se hayan emitido para concurrir ante un tribunal, en calidad de imputado.








NOTA
      La letra b) del Artículo 3 de la Ley 20603, publicada el 27.06.2012, modifica el presente artículo en el sentido de reemplazar en el inciso cuarto, la oración "gozando de alguno de los beneficios alternativos a la ejecución de las penas alternativas o restrictivas de libertad contemplados en la ley" por lo siguiente: "cumpliendo alguna de las penas sustitutivas a la ejecución de las penas privativas o restrictivas de libertad contempladas en la ley", pero dicha modificación no fue posible hacerla debido a una inconsistencia en el texto.
    Artículo 141. Improcedencia de la prisión preventiva. No se podrá ordenar la prisión preventiva:

    a) Cuando el delito imputado estuviere sancionado únicamente con penas pecuniarias o privativas de derechos;

    b) Cuando se tratare de delitos de acción privada, y
    c) Cuando el imputado se encontrare cumpliendo efectivamente una pena privativa de libertad. Si por cualquier motivo fuere a cesar el cumplimiento efectivo de la pena y el fiscal o el querellante estimaren necesaria la prisión preventiva o alguna de las medidas previstas en el Párrafo 6º, podrá solicitarlas anticipadamente, de conformidad a las disposiciones de este Párrafo, a fin de que, si el tribunal acogiere la solicitud, la medida se aplique al imputado en cuanto cese el cumplimiento efectivo de la pena, sin solución de continuidad.

      Podrá en todo caso decretarse la prisión preventiva en los eventos previstos en el inciso anterior, cuando el imputado hubiere incumplido alguna de las medidas cautelares previstas en el Párrafo 6° de este Título o cuando el tribunal considerare que el imputado pudiere incumplir con su obligación de permanecer en el lugar del juicio hasta su término y presentarse a los actos del procedimiento como a la ejecución de la sentencia, inmediatamente que fuere requerido o citado de conformidad a los artículos 33 y 123. Se decretará también la prisión preventiva del imputado que no asistiere a la audiencia del juicio oral, resolución que se dictará en la misma audiencia, a petición del fiscal o del querellante.

    Artículo 142.- Tramitación de la solicitud de prisión preventiva. La solicitud de prisión preventiva podrá plantearse verbalmente en la audiencia de formalización de la investigación, en la audiencia de preparación del juicio oral o en la audiencia del juicio oral.
    También podrá solicitarse en cualquier etapa de la investigación, respecto del imputado contra quien se hubiere formalizado ésta, caso en el cual el juez fijará una audiencia para la resolución de la solicitud, citando a ella al imputado, su defensor y a los demás intervinientes.
    La presencia del imputado y su defensor constituye un requisito de validez de la audiencia en que se resolviere la solicitud de prisión preventiva.
    Una vez expuestos los fundamentos de la solicitud por quien la hubiere formulado, el tribunal oirá en todo caso al defensor, a los demás intervinientes si estuvieren presentes y quisieren hacer uso de la palabra y al imputado.
    Artículo 143.- Resolución sobre la prisión preventiva. Al concluir la audiencia el tribunal se pronunciará sobre la prisión preventiva por medio de una resolución fundada, en la cual expresará claramente los antecedentes calificados que justificaren la decisión.
    Artículo 144.- Modificación y revocación de la resolución sobre la prisión preventiva. La resolución que ordenare o rechazare la prisión preventiva será modificable de oficio o a petición de cualquiera de los intervinientes, en cualquier estado del procedimiento.
    Cuando el imputado solicitare la revocación de la prisión preventiva el tribunal podrá rechazarla de plano; asimismo, podrá citar a todos los intervinientes a una audiencia, con el fin de abrir debate sobre la subsistencia de los requisitos que autorizan la medida.
    Si la prisión preventiva hubiere sido rechazada, ella podrá ser decretada con posterioridad en una audiencia, cuando existieren otros antecedentes que, a juicio del tribunal, justificaren discutir nuevamente su procedencia.



    Artículo 145.- Substitución de la prisión preventiva y revisión de oficio. En cualquier momento del procedimiento el tribunal, de oficio o a petición de parte, podrá substituir la prisión preventiva por alguna de las medidas que se contemplan en las disposiciones del Párrafo 6º de este Título.
    Transcurridos seis meses desde que se hubiere ordenado la prisión preventiva o desde el último debate oral en que ella se hubiere decidido, el tribunal citará de oficio a una audiencia, con el fin de considerar su cesación o prolongación.
    Artículo 146.- Caución para reemplazar la prisión preventiva. Cuando la prisión preventiva hubiere sido o debiere ser impuesta únicamente para garantizar la comparecencia del imputado al juicio y a la eventual ejecución de la pena, el tribunal podrá autorizar su reemplazo por una caución económica suficiente, cuyo monto fijará.
    La caución podrá consistir en el depósito por el imputado u otra persona de dinero o valores, la constitución de prendas o hipotecas, o la fianza de una o más personas idóneas calificadas por el tribunal.

    Artículo 147.- Ejecución de las cauciones económicas. En los casos de rebeldía o cuando el imputado se sustrajere a la ejecución de la pena, se procederá a ejecutar la garantía de acuerdo con las reglas generales y se entregará el monto que se obtuviere a la Corporación Administrativa del Poder Judicial.
    Si la caución hubiere sido constituida por un tercero, producida alguna de las circunstancias a que se refiere el inciso anterior, el tribunal ordenará ponerla en conocimiento del tercero interesado, apercibiéndolo con que si el imputado no compareciere dentro de cinco días, se procederá a hacer efectiva la caución.
    En ambos casos, si la caución no consistiere en dinero o valores, actuará como ejecutante el Consejo de Defensa del Estado, para lo cual el tribunal procederá a poner los antecedentes en su conocimiento, oficiándole al efecto.
    Artículo 148.- Cancelación de la caución. La caución será cancelada y devueltos los bienes afectados, siempre que no hubieren sido ejecutados con anterioridad:
    a) Cuando el imputado fuere puesto en prisión preventiva;
    b) Cuando, por resolución firme, se absolviere al imputado, se sobreseyere la causa o se suspendiere condicionalmente el procedimiento, y
    c) Cuando se comenzare a ejecutar la pena privativa de libertad o se resolviere que ella no debiere ejecutarse en forma efectiva, siempre que previamente se pagaren la multa y las costas que impusiere la sentencia.
    Artículo 149.- Recursos relacionados con la medida de prisión preventiva. La resolución que ordenare, mantuviere, negare lugar o revocare la prisión preventiva será apelable cuando hubiere sido dictada en una audiencia. No obstará a la procedencia del recurso, la circunstancia de haberse decretado, a petición de cualquiera de los intervinientes, alguna de las medidas cautelares señaladas en el artículo 155. En los demás casos no será susceptible de recurso alguno.
    Tratándose de los delitos establecidos en los artículos 141, 142, 292, 293, 361, 362, 363, 365 bis, 366 incisos primero y segundo, 366 bis, 390, 390 bis, 390 ter, 391, 411 bis, 411 ter, 411 quáter, 433, 436 y 440 del Código Penal, en las leyes N°17.798 y N°20.000 y de los delitos de castración, mutilaciones y lesiones contra miembros de Carabineros, de la Policía de Investigaciones y de Gendarmería de Chile, en el ejercicio de sus funciones, el imputado que hubiere sido puesto a disposición del tribunal en calidad de detenido o se encontrare en prisión preventiva no podrá ser puesto en libertad mientras no se encontrare ejecutoriada la resolución que negare, sustituyere o revocare la prisión preventiva. El recurso de apelación contra esta resolución deberá interponerse en la misma audiencia, gozará de preferencia para su vista y fallo y será agregado extraordinariamente a la tabla el mismo día de su ingreso al Tribunal de Alzada, o a más tardar a la del día siguiente hábil. Cada Corte de Apelaciones deberá establecer una sala de turno que conozca estas apelaciones en días feriados.
    En los casos en que no sea aplicable lo dispuesto en el inciso anterior, estando pendiente el recurso contra la resolución que dispone la libertad, para impedir la posible fuga del imputado la Corte de Apelaciones respectiva tendrá la facultad de decretar una orden de no innovar, desde luego y sin esperar la vista del recurso de apelación del fiscal o del querellante.







    Artículo 150.- Ejecución de la medida de prisión preventiva. El tribunal será competente para supervisar la ejecución de la prisión preventiva que ordenare en las causas de que conociere. A él corresponderá conocer de las solicitudes y presentaciones realizadas con ocasión de la ejecución de la medida.
    La prisión preventiva se ejecutará en establecimientos especiales, diferentes de los que se utilizaren para los condenados o, al menos, en lugares absolutamente separados de los destinados para estos últimos.
    El imputado será tratado en todo momento como inocente. La prisión preventiva se cumplirá de manera tal que no adquiera las características de una pena, ni provoque otras limitaciones que las necesarias para evitar la fuga y para garantizar la seguridad de los demás internos y de las personas que cumplieren funciones o por cualquier motivo se encontraren en el recinto.
    El tribunal deberá adoptar y disponer las medidas necesarias para la protección de la integridad física del imputado, en especial aquellas destinadas a la separación de los jóvenes y no reincidentes respecto de la población penitenciaria de mayor peligrosidad.
    El tribunal podrá excepcionalmente conceder al imputado permiso de salida por resolución fundada y por el tiempo estrictamente necesario para el cumplimiento de los fines del referido permiso, siempre que se asegure convenientemente que no se vulnerarán los objetivos de la prisión preventiva.
    INCISO SUPRIMIDO.
    Cualquier restricción que la autoridad penitenciaria impusiere al imputado deberá ser inmediatamente comunicada al tribunal, con sus fundamentos. Éste podrá dejarla sin efecto si la considerare ilegal o abusiva, convocando, si lo estimare necesario, a una audiencia para su examen.




    Artículo 151.- Prohibición de comunicaciones. El tribunal podrá, a petición del fiscal, restringir o prohibir las comunicaciones del detenido o preso hasta por un máximo de diez días, cuando considerare que ello resulta necesario para el exitoso desarrollo de la investigación. En todo caso esta facultad no podrá restringir el acceso del imputado a su abogado en los términos del artículo 94, letra f), ni al propio tribunal. Tampoco se podrá restringir su acceso a una apropiada atención médica.
    El tribunal deberá instruir a la autoridad encargada del recinto en que el imputado se encontrare acerca del modo de llevar a efecto la medida, el que en ningún caso podrá consistir en el encierro en celdas de castigo.
    Artículo 152.- Límites temporales de la prisión preventiva. El tribunal, de oficio o a petición de cualquiera de los intervinientes, decretará la terminación de la prisión preventiva cuando no subsistieren los motivos que la hubieren justificado.
    En todo caso, cuando la duración de la prisión preventiva hubiere alcanzado la mitad de la pena privativa de libertad que se pudiere esperar en el evento de dictarse sentencia condenatoria, o de la que se hubiere impuesto existiendo recursos pendientes, el tribunal citará de oficio a una audiencia, con el fin de considerar su cesación o prolongación.
    Artículo 153.- Término de la prisión preventiva por absolución o sobreseimiento. El tribunal deberá poner término a la prisión preventiva cuando dictare sentencia absolutoria y cuando decretare sobreseimiento definitivo o temporal, aunque dichas resoluciones no se encontraren ejecutoriadas.
    En los casos indicados en el inciso precedente, se podrá imponer alguna de las medidas señaladas en el párrafo 6º de este Título, cuando se consideraren necesarias para asegurar la presencia del imputado.
    Párrafo 5º Requisitos comunes a la prisión
preventiva y a la detención
    Artículo 154.- Orden Judicial. Toda orden de prisión preventiva o de detención será expedida por escrito por el tribunal y contendrá: a) El nombre y apellidos de la persona que debiere
ser detenida o aprehendida o, en su defecto, las
circunstancias que la individualizaren o determinaren;
    b) El motivo de la prisión o detención, y
    c) La indicación de ser conducido de inmediato ante
el tribunal, al establecimiento penitenciario o lugar
público de prisión o detención que determinará, o de
permanecer en su residencia, según correspondiere.
      Lo dispuesto en este artículo se entenderá sin
perjuicio de lo previsto en el artículo 9º para los
casos urgentes.

    Párrafo 6º Otras medidas cautelares personales
    Artículo 155.- Enumeración y aplicación de otras medidas cautelares personales. Para garantizar el éxito de las diligencias de investigación o la seguridad de la sociedad, proteger al ofendido o asegurar la comparecencia del imputado a las actuaciones del procedimiento o ejecución de la sentencia, después de formalizada la investigación el tribunal, a petición del fiscal, del querellante o la víctima, podrá imponer al imputado una o más de las siguientes medidas:
    a) La privación de libertad, total o parcial, en su casa o en la que el propio imputado señalare, si aquélla se encontrare fuera de la ciudad asiento del tribunal;
    b) La sujeción a la vigilancia de una persona o institución determinada, las que informarán periódicamente al juez;
    c) La obligación de presentarse periódicamente ante el juez o ante la autoridad que él designare;
    d) La prohibición de salir del país, de la localidad en la cual residiere o del ámbito territorial que fijare el tribunal;
    e) La prohibición de asistir a determinadas reuniones, recintos o espectáculos públicos, o de visitar determinados lugares;
    f) La prohibición de comunicarse con personas determinadas, siempre que no se afectare el derecho a defensa;
    g) La prohibición de aproximarse al ofendido o su familia y, en su caso, la obligación de abandonar el hogar que compartiere con aquél;
    h) La prohibición de poseer, tener o portar armas de fuego, municiones o cartuchos, y
    i) La obligación del imputado de abandonar un inmueble determinado.
    El tribunal podrá imponer una o más de estas medidas según resultare adecuado al caso y ordenará las actuaciones y comunicaciones necesarias para garantizar su cumplimiento.
    La procedencia, duración, impugnación y ejecución de estas medidas cautelares se regirán por las disposiciones aplicables a la prisión preventiva, en cuanto no se opusieren a lo previsto en este Párrafo.




    Artículo 156.- Suspensión temporal de otras medidas cautelares personales. El tribunal podrá dejar temporalmente sin efecto las medidas contempladas en este Párrafo, a petición del afectado por ellas, oyendo al fiscal y previa citación de los demás intervinientes que hubieren participado en la audiencia en que se decretaron, cuando estimare que ello no pone en peligro los objetivos que se tuvieron en vista al imponerlas. Para estos efectos, el juez podrá admitir las cauciones previstas en el artículo 146.
    Artículo 156 bis.- Medidas  cautelares especiales. En los casos de investigaciones por fraude en el otorgamiento de licencias médicas, el tribunal podrá, en la oportunidad y a petición de las personas señaladas en el artículo 155, decretar la suspensión de la facultad de emitir dichas licencias mientras dure la investigación o por el menor plazo que, fundadamente, determine.
    Título VI

    Medidas cautelares reales
    Artículo 157.- Procedencia de las medidas cautelares reales. Durante la etapa de investigación, el ministerio público o la víctima podrán solicitar por escrito al juez de garantía que decrete respecto del imputado, una o más de las medidas precautorias autorizadas en el Título V del Libro Segundo del Código de Procedimiento Civil. En estos casos, las solicitudes respectivas se substanciarán y regirán de acuerdo a lo previsto en el Título IV del mismo Libro. Con todo, concedida la medida, el plazo para presentar la demanda se extenderá hasta la oportunidad prevista en el artículo 60.
    Del mismo modo, al deducir la demanda civil, la víctima podrá solicitar que se decrete una o más de dichas medidas.
    El Ministerio Público deberá solicitar las medidas cautelares que correspondan para asegurar bienes suficientes con el fin de hacer efectivo el comiso de las ganancias provenientes del delito o, de proceder, el comiso por valor equivalente de instrumentos o efectos del delito. Para estos efectos, el juez podrá autorizar la retención de dineros o cosas muebles que se encuentren en poder del imputado o de terceros, o en cuentas de bancos o en fondos generales administrados por terceros.


    Artículo 157 bis.- Concesión de medidas sin audiencia del afectado. Las medidas solicitadas para asegurar bienes sobre los cuales hacer efectivo el comiso de ganancias o de valor equivalente de bienes o efectos podrán ser decretadas sin audiencia del afectado.
    Si se procede de este modo, el juez deberá fijar un plazo no inferior a treinta días ni superior a ciento veinte días para que el Ministerio Público formalice la investigación respectiva. Transcurrido este plazo sin que se produzca la formalización, o sin que el Ministerio Público solicite la mantención de la medida con ocasión de la formalización, la medida quedará sin efecto.

    Artículo 158.- Recurso de apelación. Serán apelables las resoluciones que negaren o dieren lugar a las medidas previstas en este Título.
    Título VII
    Nulidades procesales
    Artículo 159.- Procedencia de las nulidades procesales. Sólo podrán anularse las actuaciones o diligencias judiciales defectuosas del procedimiento que ocasionaren a los intervinientes un perjuicio reparable únicamente con la declaración de nulidad. Existe perjuicio cuando la inobservancia de las formas procesales atenta contra las posibilidades de actuación de cualquiera de los intervinientes en el procedimiento.
    Artículo 160.- Presunción de derecho del perjuicio. Se presumirá de derecho la existencia del perjuicio si la infracción hubiere impedido el pleno ejercicio de las garantías y de los derechos reconocidos en la Constitución, o en las demás leyes de la República.
    Artículo 161.- Oportunidad para solicitar la nulidad. La declaración de nulidad procesal se deberá impetrar, en forma fundada y por escrito, incidentalmente, dentro de los cinco días siguientes a aquél en que el perjudicado hubiere tomado conocimiento fehaciente del acto cuya invalidación persiguiere, a menos que el vicio se hubiere producido en una actuación verificada en una audiencia, pues en tal caso deberá impetrarse verbalmente antes del término de la misma audiencia. Con todo, no podrá reclamarse la nulidad de actuaciones verificadas durante la etapa de investigación después de la audiencia de preparación del juicio oral. La solicitud de nulidad presentada extemporáneamente será declarada inadmisible.
    Artículo 162.- Titulares de la solicitud de declaración de nulidad. Sólo podrá solicitar la declaración de nulidad el interviniente en el procedimiento perjudicado por el vicio y que no hubiere concurrido a causarlo.
    Artículo 163.- Nulidad de oficio. Si el tribunal estimare haberse producido un acto viciado y la nulidad no se hubiere saneado aún, lo pondrá en conocimiento del interviniente en el procedimiento a quien estimare que la nulidad le ocasiona un perjuicio, a fin de que proceda como creyere conveniente a sus derechos, a menos de que se tratare de una nulidad de las previstas en el artículo 160, caso en el cual podrá declararla de oficio.
    Artículo 164.- Saneamiento de la nulidad. Las nulidades quedarán subsanadas si el interviniente en el procedimiento perjudicado no impetrare su declaración oportunamente, si aceptare expresa o tácitamente los efectos del acto y cuando, a pesar del vicio, el acto cumpliere su finalidad respecto de todos los interesados, salvo en los casos previstos en el artículo 160.
    Artículo 165.- Efectos de la declaración de nulidad. La declaración de nulidad del acto conlleva la de los actos consecutivos que de él emanaren o dependieren.
    El tribunal, al declarar la nulidad, determinará concretamente cuáles son los actos a los que ella se extendiere y, siendo posible, ordenará que se renueven, rectifiquen o ratifiquen.
    Con todo, la declaración de nulidad no podrá retrotraer el procedimiento a etapas anteriores, a pretexto de repetición del acto, rectificación del error o cumplimiento del acto omitido, salvo en los casos en que ello correspondiere de acuerdo con las normas del recurso de nulidad. De este modo, si durante la audiencia de preparación del juicio oral se declarare la nulidad de actuaciones judiciales realizadas durante la etapa de investigación, el tribunal no podrá ordenar la reapertura de ésta. Asimismo, las nulidades declaradas durante el desarrollo de la audiencia del juicio oral no retrotraerán el procedimiento a la etapa de investigación o a la audiencia de preparación del juicio oral.
    La solicitud de nulidad constituirá preparación suficiente del recurso de nulidad para el caso que el tribunal no resolviere la cuestión de conformidad a lo solicitado.
    Libro Segundo
    Procedimiento ordinario

    Título I
    Etapa de investigación

    Párrafo 1º Persecución penal pública
    Artículo 166.- Ejercicio de la acción penal. Los delitos de acción pública serán investigados con arreglo a las disposiciones de este Título.
    Cuando el ministerio público tomare conocimiento de la existencia de un hecho que revistiere caracteres de delito, con el auxilio de la policía, promoverá la persecución penal, sin que pueda suspender, interrumpir o hacer cesar su curso, salvo en los casos previstos en la ley.
    Tratándose de delitos de acción pública previa instancia particular, no podrá procederse sin que, a lo menos, se hubiere denunciado el hecho con arreglo al artículo 54, salvo para realizar los actos urgentes de investigación o los absolutamente necesarios para impedir o interrumpir la comisión del delito.
    En los delitos previstos en los artículos 459 y 460 del Código Penal, recibida la denuncia el fiscal comunicará los hechos a la Dirección General de Aguas del Ministerio de Obras Públicas.

    Artículo 167.- Archivo provisional. En tanto no se hubiere producido la intervención del juez de garantía en el procedimiento, el ministerio público podrá archivar provisionalmente aquellas investigaciones en las que no aparecieren antecedentes que permitieren desarrollar actividades conducentes al esclarecimiento de los hechos.
    Si el delito mereciere pena aflictiva, el fiscal deberá someter la decisión sobre archivo provisional a la aprobación del Fiscal Regional.
    La víctima podrá solicitar al ministerio público la reapertura del procedimiento y la realización de diligencias de investigación. Asimismo, podrá reclamar de la denegación de dicha solicitud ante las autoridades del ministerio público.
    Artículo 168.- Facultad para no iniciar investigación. En tanto no se hubiere producido la intervención del juez de garantía en el procedimiento, el fiscal podrá abstenerse de toda investigación, cuando los hechos relatados en la denuncia no fueren constitutivos de delito o cuando los antecedentes y datos suministrados permitieren establecer que se encuentra extinguida la responsabilidad penal del imputado. Esta decisión será siempre fundada y se someterá a la aprobación del juez de garantía.
    Artículo 169.- Control judicial. En los casos contemplados en los dos artículos anteriores, la víctima podrá provocar la intervención del juez de garantía deduciendo la querella respectiva.
    Si el juez admitiere a tramitación la querella, el fiscal deberá seguir adelante la investigación conforme a las reglas generales.
    Artículo 170.- Principio de oportunidad. Los fiscales del ministerio público podrán no iniciar la persecución penal o abandonar la ya iniciada cuando se tratare de un hecho que no comprometiere gravemente el interés público, a menos que la pena mínima asignada al delito excediere la de presidio o reclusión menores en su grado mínimo o que se tratare de un delito cometido por un funcionario público en el ejercicio de sus funciones.
    El ejercicio de esta facultad se regulará mediante instrucciones generales dictadas por el Ministerio Público, con el objetivo de establecer un uso racional de la misma.
    Para estos efectos, el fiscal deberá emitir una decisión motivada, la que comunicará al juez de garantía. Éste, a su vez, la notificará a los intervinientes, si los hubiere.
    Dentro de los diez días siguientes a la comunicación de la decisión del fiscal, el juez, de oficio o a petición de cualquiera de los intervinientes, podrá dejarla sin efecto cuando considerare que aquél ha excedido sus atribuciones en cuanto la pena mínima prevista para el hecho de que se tratare excediere la de presidio o reclusión menores en su grado mínimo, o se tratare de un delito cometido por un funcionario público en el ejercicio de sus funciones. También la dejará sin efecto cuando, dentro del mismo plazo, la víctima manifestare de cualquier modo su interés en el inicio o en la continuación de la persecución penal.
    La decisión que el juez emitiere en conformidad al inciso anterior obligará al fiscal a continuar con la persecución penal.
    Una vez vencido el plazo señalado en el inciso tercero o rechazada por el juez la reclamación respectiva, los intervinientes contarán con un plazo de diez días para reclamar de la decisión del fiscal ante las autoridades del ministerio público.
    Conociendo de esta reclamación, las autoridades del ministerio público deberán verificar si la decisión del fiscal se ajusta a las políticas generales del servicio y a las normas que hubieren sido dictadas al respecto. Transcurrido el plazo previsto en el inciso precedente sin que se hubiere formulado reclamación o rechazada ésta por parte de las autoridades del ministerio público, se entenderá extinguida la acción penal respecto del hecho de que se tratare.
    La extinción de la acción penal de acuerdo a lo previsto en este artículo no perjudicará en modo alguno el derecho a perseguir por la vía civil las responsabilidades pecuniarias derivadas del mismo hecho.

    Artículo 171.- Cuestiones prejudiciales civiles. Siempre que para el juzgamiento criminal se requiriere la resolución previa de una cuestión civil de que debiere conocer, conforme a la ley, un tribunal que no ejerciere jurisdicción en lo penal, se suspenderá el procedimiento criminal hasta que dicha cuestión se resolviere por sentencia firme.
    Esta suspensión no impedirá que se verifiquen actuaciones urgentes y estrictamente necesarias para conferir protección a la víctima o a testigos o para establecer circunstancias que comprobaren los hechos o la participación del imputado y que pudieren desaparecer.
    Cuando se tratare de un delito de acción penal pública, el ministerio público deberá promover la iniciación de la causa civil previa e intervendrá en ella hasta su término, instando por su pronta conclusión.
    Párrafo 2º Inicio del procedimiento
    Artículo 172.- Formas de inicio. La investigación de un hecho que revistiere caracteres de delito podrá iniciarse de oficio por el ministerio público, por denuncia o por querella.
    Artículo 173.- Denuncia. Cualquier persona podrá comunicar directamente al ministerio público el conocimiento que tuviere de la comisión de un hecho que revistiere caracteres de delito.

    También se podrá formular la denuncia ante los funcionarios de Carabineros de Chile, de la Policía de Investigaciones, de Gendarmería de Chile en los casos de los delitos cometidos dentro de los recintos penitenciarios, o ante cualquier tribunal con competencia criminal, todos los cuales deberán hacerla llegar de inmediato al ministerio público.
    Artículo 174.- Forma y contenido de la denuncia. La denuncia podrá formularse por cualquier medio y deberá contener la identificación del denunciante, el señalamiento de su domicilio, la narración circunstanciada del hecho, la designación de quienes lo hubieren cometido y de las personas que lo hubieren presenciado o que tuvieren noticia de él, todo en cuanto le constare al denunciante.
    En el caso de la denuncia verbal se levantará un registro en presencia del denunciante, quien lo firmará junto con el funcionario que la recibiere. La denuncia escrita será firmada por el denunciante. En ambos casos, si el denunciante no pudiere firmar, lo hará un tercero a su ruego.
    Artículo 175.- Denuncia obligatoria. Estarán obligados a denunciar:
    a) Los miembros de Carabineros de Chile, de la Policía de Investigaciones de Chile y de Gendarmería, todos los delitos que presenciaren o llegaren a su noticia. Los miembros de las Fuerzas Armadas estarán también obligados a denunciar todos los delitos de que tomaren conocimiento en el ejercicio de sus funciones;
    b) Los fiscales y los demás empleados públicos, los delitos de que tomaren conocimiento en el ejercicio de sus funciones y, especialmente, en su caso, los que notaren en la conducta ministerial de sus subalternos;
    c) Los jefes de puertos, aeropuertos, estaciones de trenes o buses o de otros medios de locomoción o de carga, los capitanes de naves o de aeronaves comerciales que naveguen en el mar territorial o en el espacio territorial, respectivamente, y los conductores de los trenes, buses u otros medios de transporte o carga, los delitos que se cometieren durante el viaje, en el recinto de una estación, puerto o aeropuerto o a bordo del buque o aeronave;
    d) Los jefes de establecimientos hospitalarios o de clínicas particulares y, en general, los profesionales en medicina, odontología, química, farmacia y de otras ramas relacionadas con la conservación o el restablecimiento de la salud, y los que ejercieren prestaciones auxiliares de ellas, que notaren en una persona o en un cadáver señales de envenenamiento o de otro delito;
    e) Los directores, inspectores y profesores de establecimientos educacionales de todo nivel, los delitos que afectaren a los alumnos o que hubieren tenido lugar en el establecimiento.
    La denuncia realizada por alguno de los obligados en este artículo eximirá al resto, y
    f) Los jefes de establecimientos de salud, públicos o privados, y los sostenedores y directores de establecimientos educacionales, públicos o privados, respecto de los delitos perpetrados contra los profesionales y funcionarios de dichos establecimientos al interior de sus dependencias o mientras éstos se encontraren en el ejercicio de sus funciones o en razón, con motivo u ocasión de ellas. La misma obligación tendrán los directores de los Servicios Locales de Educación respecto de estos delitos, cuando ocurran en los establecimientos educacionales que formen parte del territorio de su competencia.

    Artículo 176.- Plazo para efectuar la denuncia. Las personas indicadas en el artículo anterior deberán hacer la denuncia dentro de las veinticuatro horas siguientes al momento en que tomaren conocimiento del hecho criminal. Respecto de los capitanes de naves o de aeronaves, este plazo se contará desde que arribaren a cualquier puerto o aeropuerto de la República.
    Artículo 177.- Incumplimiento de la obligación de denunciar. Las personas indicadas en el artículo 175 que omitieren hacer la denuncia que en él se prescribe incurrirán en la pena prevista en el artículo 494 del Código Penal, o en la señalada en disposiciones especiales, en lo que correspondiere.
    La pena por el delito en cuestión no será aplicable cuando apareciere que quien hubiere omitido formular la denuncia arriesgaba la persecución penal propia, del cónyuge, de su conviviente o de ascendientes, descendientes o hermanos.
    Artículo 178.- Responsabilidad y derechos del denunciante. El denunciante no contraerá otra responsabilidad que la correspondiente a los delitos que hubiere cometido por medio de la denuncia o con ocasión de ella. Tampoco adquirirá el derecho a intervenir posteriormente en el procedimiento, sin perjuicio de las facultades que pudieren corresponderle en el caso de ser víctima del delito.
    Artículo 179.- Autodenuncia. Quien hubiere sido imputado por otra persona de haber participado en la comisión de un hecho ilícito, tendrá el derecho de concurrir ante el ministerio público y solicitar se investigue la imputación de que hubiere sido objeto.
    Si el fiscal respectivo se negare a proceder, la persona imputada podrá recurrir ante las autoridades superiores del ministerio público, a efecto de que revisen tal decisión.
    Párrafo 3º Actuaciones de la investigación
    Artículo 180.- Investigación de los fiscales. Los fiscales dirigirán la investigación y podrán realizar por sí mismos o encomendar a la policía todas las diligencias de investigación que consideraren conducentes al esclarecimiento de los hechos.
    Sin perjuicio de lo dispuesto en el Párrafo 1º de este Título, dentro de las veinticuatro horas siguientes a que tomare conocimiento de la existencia de un hecho que revistiere caracteres de delito de acción penal pública por alguno de los medios previstos en la ley, el fiscal deberá proceder a la práctica de todas aquellas diligencias pertinentes y útiles al esclarecimiento y averiguación del mismo, de las circunstancias relevantes para la aplicación de la ley penal, de los partícipes del hecho y de las circunstancias que sirvieren para verificar su responsabilidad. Asimismo, deberá impedir que el hecho denunciado produzca consecuencias ulteriores.
    Los fiscales podrán exigir información de toda persona o funcionario público, los que no podrán excusarse de proporcionarla, salvo en los casos expresamente exceptuados por la ley. Los notarios, archiveros y conservadores de bienes raíces, y demás organismos, autoridades y funcionarios públicos, deberán realizar las actuaciones y diligencias y otorgar los informes, antecedentes y copias de instrumentos que los fiscales les solicitaren, en forma gratuita y exentos de toda clase de derechos e impuestos.

    Artículo 181.- Actividades de la investigación. Para los fines previstos en el artículo anterior, la investigación se llevará a cabo de modo de consignar y asegurar todo cuanto condujere a la comprobación del hecho y a la identificación de los partícipes en el mismo. Así, se hará constar el estado de las personas, cosas o lugares, se identificará a los testigos del hecho investigado y se consignarán sus declaraciones. Del mismo modo, si el hecho hubiere dejado huellas, rastros o señales, se tomará nota de ellos y se los especificará detalladamente, se dejará constancia de la descripción del lugar en que aquél se hubiere cometido, del estado de los objetos que en él se encontraren y de todo otro dato pertinente.
    Para el cumplimiento de los fines de la investigación se podrá disponer la práctica de operaciones científicas, la toma de fotografías, filmación o grabación y, en general, la reproducción de imágenes, voces o sonidos por los medios técnicos que resultaren más adecuados, requiriendo la intervención de los organismos especializados. En estos casos, una vez verificada la operación se certificará el día, hora y lugar en que ella se hubiere realizado, el nombre, la dirección y la profesión u oficio de quienes hubieren intervenido en ella, así como la individualización de la persona sometida a examen y la descripción de la cosa, suceso o fenómeno que se reprodujere o explicare. En todo caso se adoptarán las medidas necesarias para evitar la alteración de los originales objeto de la operación.
    Artículo 182.- Secreto de las actuaciones de investigación. Las actuaciones de investigación realizadas por el ministerio público y por la policía serán secretas para los terceros ajenos al procedimiento.
    El imputado y los demás intervinientes en el procedimiento podrán examinar y obtener copias, a su cargo, de los registros y documentos de la investigación fiscal y podrán examinar los de la investigación policial.
    El fiscal podrá disponer que determinadas actuaciones, registros o documentos sean mantenidas en secreto respecto del imputado o de los demás intervinientes, cuando lo considerare necesario para la eficacia de la investigación. En tal caso deberá identificar las piezas o actuaciones respectivas, de modo que no se vulnere la reserva y fijar un plazo no superior a cuarenta días para la mantención del secreto, el cual podrá ser ampliado por el mismo período, por una sola vez, con motivos fundados. Esta ampliación no será oponible ni al imputado ni a su defensa.
    El imputado o cualquier otro interviniente podrá solicitar del juez de garantía que ponga término al secreto o que lo limite, en cuanto a su duración, a las piezas o actuaciones abarcadas por él, o a las personas a quienes afectare.
    Sin perjuicio de lo dispuesto en los incisos anteriores, no se podrá decretar el secreto sobre la declaración del imputado o cualquier otra actuación en que hubiere intervenido o tenido derecho a intervenir, las actuaciones en las que participare el tribunal, ni los informes evacuados por peritos, respecto del propio imputado o de su defensor.
    Los funcionarios que hubieren participado en la investigación y las demás personas que, por cualquier motivo, tuvieren conocimiento de las actuaciones de la investigación estarán obligados a guardar secreto respecto de ellas.


    Artículo 183.- Proposición de diligencias. Durante la investigación, tanto el imputado como los demás intervinientes en el procedimiento podrán solicitar al fiscal todas aquellas diligencias que consideraren pertinentes y útiles para el esclarecimiento de los hechos. El fiscal deberá pronunciarse dentro de los diez días siguientes a la solicitud y ordenará que se lleven a efecto aquellas que estimare conducentes.
    Si el fiscal rechazare la solicitud o no se pronunciare dentro del plazo establecido en el inciso anterior, se podrá reclamar ante las autoridades del Ministerio Público según lo disponga la ley orgánica constitucional respectiva, dentro del plazo de cinco días contado desde el rechazo o desde el vencimiento del señalado plazo, con el propósito de obtener un pronunciamiento definitivo acerca de la procedencia de la diligencia.

    Artículo 184.- Asistencia a diligencias. Durante la investigación, el fiscal podrá permitir la asistencia del imputado o de los demás intervinientes a las actuaciones o diligencias que debiere practicar, cuando lo estimare útil. En todo caso, podrá impartirles instrucciones obligatorias conducentes al adecuado desarrollo de la actuación o diligencia y podrá excluirlos de la misma en cualquier momento.
    Artículo 185.- Agrupación y separación de investigaciones. El fiscal podrá investigar separadamente cada delito de que conociere. No obstante, podrá desarrollar la investigación conjunta de dos o más delitos, cuando ello resultare conveniente. Asimismo, en cualquier momento podrá separar las investigaciones que se llevaren en forma conjunta.
    Cuando dos o más fiscales se encontraren investigando los mismos hechos y con motivo de esta circunstancia se afectaren los derechos de la defensa del imputado, éste podrá pedir al superior jerárquico o al superior jerárquico común, en su caso, que resuelva cuál tendrá a su cargo el caso.
    Artículo 186.- Control judicial anterior a la formalización de la investigación. Cualquier persona que se considerare afectada por una investigación que no se hubiere formalizado judicialmente, podrá pedir al juez de garantía que le ordene al fiscal informar acerca de los hechos que fueren objeto de ella. También podrá el juez fijarle un plazo para que formalice la investigación.
    Artículo 187.- Objetos, documentos e instrumentos. Los objetos, documentos e instrumentos de cualquier clase que parecieren haber servido o haber estado destinados a la comisión del hecho investigado, o los que de él provinieren, o los que pudieren servir como medios de prueba, así como los que se encontraren en el sitio del suceso a que se refiere la letra c) del artículo 83, serán recogidos, identificados y conservados bajo sello. En todo caso, se levantará un registro de la diligencia, de acuerdo con las normas generales.
    Si los objetos, documentos e instrumentos se encontraren en poder del imputado o de otra persona, se procederá a su incautación, de conformidad a lo dispuesto en este Título. Con todo, tratándose de objetos, documentos e instrumentos que fueren hallados en poder del imputado respecto de quien se practicare detención en ejercicio de la facultad prevista en el artículo 83 letra b) o se encontraren en el sitio del suceso, se podrá proceder a su incautación en forma inmediata.



    Artículo 187 bis.- Enajenación temprana de especies. A solicitud del Ministerio Público, el juez de garantía podrá disponer la enajenación temprana de los bienes incautados, siempre que se trate de vehículos motorizados, o se trate de bienes sujetos a corrupción, susceptibles de próximo deterioro y cuya conservación sea difícil o muy dispendiosa.
    Para estos efectos, el juez de garantía deberá oficiar a la Dirección General del Crédito Prendario para que informe sobre la tasación del respectivo bien. En caso de que éste deba ser destruido por dicho organismo por carecer de valor, el juez de garantía así deberá decretarlo en la resolución.
    Si el bien figura inscrito en algún registro público, sea que acredite o no propiedad, el juez de garantía, antes de resolver la enajenación temprana, deberá citar a quienes figuren como titulares de derechos en dichos registros. En caso de que el citado no comparezca a la audiencia de enajenación temprana, se procederá en su ausencia.
    La enajenación se llevará a cabo por la Dirección General del Crédito Prendario en subasta pública cuando la resolución que disponga la enajenación se encuentre firme o ejecutoriada.
    El monto de lo obtenido en la subasta será depositado en el Banco del Estado de Chile, en cuentas o valores reajustables y con intereses.
    En el evento que la sentencia sea absolutoria o no establezca el comiso de las especies enajenadas, el precio de la venta, sus reajustes e intereses serán restituidos a quien corresponda. En caso contrario se destinarán a la Corporación Administrativa del Poder Judicial.
    Artículo 188.- Conservación de las especies. Las especies recogidas durante la investigación serán conservadas bajo la custodia del ministerio público, quien deberá tomar las medidas necesarias para evitar que se alteren de cualquier forma.
    Podrá reclamarse ante el juez de garantía por la inobservancia de las disposiciones antes señaladas, a fin que adopte las medidas necesarias para la debida preservación e integridad de las especies recogidas.
    Los intervinientes tendrán acceso a esas especies, con el fin de reconocerlas o realizar alguna pericia, siempre que fueren autorizados por el ministerio público o, en su caso, por el juez de garantía. El ministerio público llevará un registro especial en el que conste la identificación de las personas que fueren autorizadas para reconocerlas o manipularlas, dejándose copia, en su caso, de la correspondiente autorización.
    Artículo 189.- Reclamaciones o tercerías. Las reclamaciones o tercerías que los intervinientes o terceros entablaren durante la investigación con el fin de obtener la restitución de objetos recogidos o incautados se tramitarán ante el juez de garantía. La resolución que recayere en el artículo así tramitado se limitará a declarar el derecho del reclamante sobre dichos objetos, pero no se efectuará la devolución de éstos sino hasta después de concluido el procedimiento, a menos que el tribunal considerare innecesaria su conservación.
    Lo dispuesto en el inciso precedente no se extenderá a las cosas hurtadas, robadas o estafadas, las cuales se entregarán al dueño o legítimo tenedor en cualquier estado del procedimiento, una vez comprobado su dominio o tenencia por cualquier medio y establecido su valor.
    En todo caso, se dejará constancia mediante fotografías u otros medios que resultaren convenientes de las especies restituidas o devueltas en virtud de este artículo.

    Artículo 190.- Testigos ante el ministerio público. Durante la etapa de investigación, los testigos citados por el fiscal estarán obligados a comparecer a su presencia y prestar declaración ante el mismo o ante su abogado asistente, salvo aquellos exceptuados únicamente de comparecer a que se refiere el artículo 300. El fiscal o el abogado asistente del fiscal no podrán exigir del testigo el juramento o promesa previstos en el artículo 306.
    Si el testigo citado no compareciere sin justa causa o, compareciendo, se negare injustificadamente a declarar, se le impondrán, respectivamente, las medidas de apremio previstas en el inciso primero y las sanciones contempladas en el inciso segundo del artículo 299.
    Tratándose de testigos que fueren empleados públicos o de una empresa del Estado, el organismo público o la empresa respectiva adoptará las medidas correspondientes, las que serán de su cargo si irrogaren gastos, para facilitar la comparecencia del testigo, sea que se encontrare en el país o en el extranjero.



    Artículo 191.- Anticipación de prueba. Al concluir la declaración del testigo, el fiscal o el abogado asistente del fiscal, en su caso, le hará saber la obligación que tiene de comparecer y declarar durante la audiencia del juicio oral, así como de comunicar cualquier cambio de domicilio o de morada hasta esa oportunidad.
    Si, al hacérsele la prevención prevista en el inciso anterior, el testigo manifestare la imposibilidad de concurrir a la audiencia del juicio oral, por tener que ausentarse a larga distancia o por existir motivo que hiciere temer la sobreviniencia de su muerte, su incapacidad física o mental, o algún otro obstáculo semejante, el fiscal podrá solicitar del juez de garantía que se reciba su declaración anticipadamente.
    En los casos previstos en el inciso precedente, el juez deberá citar a todos aquellos que tuvieren derecho a asistir al juicio oral, quienes tendrán todas las facultades previstas para su participación en la audiencia del juicio oral.
    Sin perjuicio de lo anterior, la inasistencia del imputado válidamente emplazado no obstará a la validez de la audiencia en la que se rinda la prueba anticipada.

    Artículo 191 bis.- Anticipación de prueba de menores de edad. Derogado.-




NOTA
      El Art. primero transitorio de la ley 21.057, establece que la modificación introducida al presente artículo comenzará a regir de manera gradual, en plazos contados desde la publicación del Reglamento: Primera etapa: seis meses después, respecto de las regiones XV, I, II, VII, XI y XII.  Segunda etapa: dieciocho meses después, respecto de las regiones III, IV, VIII, IX y XIV.  Tercera etapa:  treinta meses después, comprendiendo las regiones V, VI, X y Metropolitana.
NOTA 1
      El artículo 7 N° 1 de la ley 21.522, publicada el 30.12.2022, modifica el presente artículo en el sentido de reemplazar la expresión "párrafos 5 y 6" por "párrafos 5, 6 y 6 bis" los cuales pertenecen al Libro Segundo, Título VII, del Código Penal; ello no obstante haber sido derogado previamente por el artículo 32 N° 3 de la ley 21.057, que regula las entrevistas grabadas en video de menores de edad. Esta misma ley establece, en su artículo 1°, que su ámbito de aplicación, comprende los párrafos indicados del Código Penal.
    Artículo 191 ter.- Anticipación de prueba con el fin de evitar la victimización secundaria. El fiscal podrá solicitar al juez de garantía que se reciba la declaración anticipada de aquellas víctimas de alguno de los delitos contemplados en el Código Penal, en los artículos 141 inciso final; 150 A; 150 D; 361; 365 bis; 366 incisos primero y segundo; 372 bis; 411 quáter, cuando se cometan con fines de explotación sexual, y 433, número 1, cuando se cometa violación, con el fin de evitar victimización secundaria.     
    En los casos previstos en el inciso precedente, el juez deberá citar a todos aquellos que tuvieren derecho a asistir al juicio oral, quienes tendrán todas las facultades previstas para su participación en la audiencia del juicio oral.
    Sin perjuicio de lo anterior, la inasistencia del imputado válidamente emplazado no obstará a la validez de la audiencia en la que se rinda la prueba anticipada.
    Artículo 192.- Anticipación de prueba testimonial en el extranjero. Si el testigo se encontrare en el extranjero y no pudiere aplicarse lo previsto en el inciso final del artículo 190, el fiscal podrá solicitar al juez de garantía que también se reciba su declaración anticipadamente.
    Para ese efecto, se recibirá la declaración del testigo, según resultare más conveniente y expedito, ante un cónsul chileno o ante el tribunal del lugar en que se hallare.
    La petición respectiva se hará llegar, por conducto de la Corte de Apelaciones correspondiente, al Ministerio de Relaciones Exteriores para su diligenciamiento, y en ella se individualizarán los intervinientes a quienes deberá citarse para que concurran a la audiencia en que se recibirá la declaración, en la cual podrán ejercer todas las facultades que les corresponderían si se tratase de una declaración prestada durante la audiencia del juicio oral.
    Si se autorizare la práctica de esta diligencia en el extranjero y ella no tuviere lugar, el ministerio público deberá pagar a los demás intervinientes que hubieren comparecido a la audiencia los gastos en que hubieren incurrido, sin perjuicio de lo que se resolviere en cuanto a costas.
    Artículo 193.- Comparecencia del imputado ante el ministerio público. Durante la etapa de investigación el imputado estará obligado a comparecer ante el fiscal, cuando éste así lo dispusiere.
    Mientras el imputado se encuentre detenido o en prisión preventiva, el fiscal estará facultado para hacerlo traer a su presencia cuantas veces fuere necesario para los fines de la investigación, sin más trámite que dar aviso al juez y al defensor.



    Artículo 194.- Declaración voluntaria del imputado. Si el imputado se allanare a prestar declaración ante el fiscal y se tratare de su primera declaración, antes de comenzar el fiscal le comunicará detalladamente cuál es el hecho que se le atribuyere, con todas las circunstancias de tiempo, lugar y modo de comisión, en la medida conocida, incluyendo aquellas que fueren de importancia para su calificación jurídica, las disposiciones legales que resultaren aplicables y los antecedentes que la investigación arrojare en su contra. A continuación, el imputado podrá declarar cuanto tuviere por conveniente sobre el hecho que se le atribuyere.
    En todo caso, el imputado no podrá negarse a proporcionar al ministerio público su completa identidad, debiendo responder las preguntas que se le dirigieren con respecto a su identificación.
    En el registro que de la declaración se practicare de conformidad a las normas generales se hará constar, en su caso, la negativa del imputado a responder una o más preguntas.
    Artículo 195.- Métodos prohibidos. Queda absolutamente prohibido todo método de investigación o de interrogación que menoscabe o coarte la libertad del imputado para declarar. En consecuencia, no podrá ser sometido a ninguna clase de coacción, amenaza o promesa.
Sólo se admitirá la promesa de una ventaja que estuviere expresamente prevista en la ley penal o procesal penal.
    Se prohíbe, en consecuencia, todo método que afecte la memoria o la capacidad de comprensión y de dirección de los actos del imputado, en especial cualquier forma de maltrato, amenaza, violencia corporal o psíquica, tortura, engaño, o la administración de psicofármacos y la hipnosis.
    Las prohibiciones previstas en este artículo rigen aun para el evento de que el imputado consintiere en la utilización de alguno de los métodos vedados.
    Artículo 196.- Prolongación excesiva de la declaración. Si el examen del imputado se prolongare por mucho tiempo, o si se le hubiere dirigido un número de preguntas tan considerable que provocare su agotamiento, se concederá el descanso prudente y necesario para su recuperación.
    Se hará constar en el registro el tiempo invertido en el interrogatorio.
    Artículo 197.- Exámenes corporales. Si fuere necesario para constatar circunstancias relevantes para la investigación, podrán efectuarse exámenes corporales del imputado o del ofendido por el hecho punible, tales como pruebas de carácter biológico, extracciones de sangre u otros análogos, siempre que no fuere de temer menoscabo para la salud o dignidad del interesado.
    Si la persona que ha de ser objeto del examen, apercibida de sus derechos, consintiere en hacerlo, el fiscal o la policía ordenará que se practique sin más trámite. En caso de negarse, se solicitará la correspondiente autorización judicial, exponiéndose al juez las razones del rechazo.
    El juez de garantía autorizará la práctica de la diligencia siempre que se cumplieren las condiciones señaladas en el inciso primero.

    Artículo 198.- Exámenes médicos y pruebas relacionadas con los delitos previstos en los artículos 361 a 367 y en el artículo 375 del Código Penal. Tratándose de los delitos previstos en los artículos 361 a 367 y en el artículo 375 del Código Penal, los hospitales, clínicas y establecimientos de salud semejantes, sean públicos o privados, deberán practicar los reconocimientos, exámenes médicos y pruebas biológicas conducentes a acreditar el hecho punible y a identificar a los partícipes en su comisión, debiendo conservar los antecedentes y muestras correspondientes.
    Se levantará acta, en duplicado, del reconocimiento y de los exámenes realizados, la que será suscrita por el jefe del establecimiento o de la respectiva sección y por los profesionales que los hubieren practicado. Una copia será entregada a la persona que hubiere sido sometida al reconocimiento, o a quien la tuviere bajo su cuidado; la otra, así como las muestras obtenidas y los resultados de los análisis y exámenes practicados, se mantendrán en custodia y bajo estricta reserva en la dirección del hospital, clínica o establecimiento de salud, por un período no inferior a un año, para ser remitidos al ministerio público.
    Si los mencionados establecimientos no se encontraren acreditados ante el Servicio Médico Legal para determinar huellas genéticas, tomarán las muestras biológicas y obtendrán las evidencias necesarias, y procederán a remitirlas a la institución que corresponda para ese efecto, de acuerdo a la ley que crea el Sistema Nacional de Registros de ADN y su Reglamento.




NOTA:
      El Art. 24 de la LEY 19970, publicada el 06.10.2004, dispuso que la modificación de la presente norma comenzará a regir cuando sea dictado su Reglamento, el que fue aprobado por DTO 634, Justicia, publicado el 25.11.2008.
    Artículo 199.- Exámenes médicos y autopsias. En los delitos en que fuere necesaria la realización de exámenes médicos para la determinación del hecho punible, el fiscal podrá ordenar que éstos sean llevados a efecto por el Servicio Médico Legal o por cualquier otro servicio médico.
    Las autopsias que el fiscal dispusiere realizar como parte de la investigación de un hecho punible serán practicadas en las dependencias del Servicio Médico Legal, por el legista correspondiente; donde no lo hubiere, el fiscal designará el médico encargado y el lugar en que la autopsia debiere ser llevada a cabo.
    Para los efectos de su investigación, el fiscal podrá utilizar los exámenes practicados con anterioridad a su intervención, si le parecieren confiables.
    Artículo 199 bis. Exámenes y pruebas de ADN. Los exámenes y pruebas biológicas destinados a la determinación de huellas genéticas sólo podrán ser efectuados por profesionales y técnicos que se desempeñen en el Servicio Médico Legal, o en aquellas instituciones públicas o privadas que se encontraren acreditadas para tal efecto ante dicho Servicio.
    Las instituciones acreditadas constarán en una nómina que, en conformidad a lo dispuesto en el Reglamento, publicará el Servicio Médico Legal en el Diario Oficial.


NOTA:
      El Art. 24 de la LEY 19970, publicada el 06.10.2004, dispuso que la modificación de la presente norma comenzará a regir cuando sea dictado su Reglamento, el que fue aprobado por DTO 634, Justicia, publicado el 25.11.2008.
    Artículo 200.- Lesiones corporales. Toda persona a cuyo cargo se encontrare un hospital u otro establecimiento de salud semejante, fuere público o privado, dará en el acto cuenta al fiscal de la entrada de cualquier individuo que tuviere lesiones corporales de significación, indicando brevemente el estado del paciente y la exposición que hicieren la o las personas que lo hubieren conducido acerca del origen de dichas lesiones y del lugar y estado en que se le hubiere encontrado. La denuncia deberá consignar el estado del paciente, describir los signos externos de las lesiones e incluir las exposiciones que hicieren el afectado o las personas que lo hubieren conducido.
    En ausencia del jefe del establecimiento, dará cuenta el que lo subrogare en el momento del ingreso del lesionado.
    El incumplimiento de la obligación prevista en este artículo se castigará con la pena que prevé el artículo 494 del Código Penal.
    Artículo 201.- Hallazgo de un cadáver. Cuando hubiere motivo para sospechar que la muerte de una persona fuere el resultado de un hecho punible, el fiscal procederá, antes de la inhumación del cadáver o inmediatamente después de su exhumación, a practicar el reconocimiento e identificación del difunto y a ordenar la autopsia.
    El cadáver podrá entregarse a los parientes del difunto o a quienes invocaren título o motivo suficiente, previa autorización del fiscal, tan pronto la autopsia se hubiere practicado.
    Artículo 202.- Exhumación. En casos calificados y cuando considerare que la exhumación de un cadáver pudiere resultar de utilidad en la investigación de un hecho punible, el fiscal podrá solicitar autorización judicial para la práctica de dicha diligencia.
    El tribunal resolverá según lo estimare pertinente, previa citación del cónyuge o del conviviente civil, o de los parientes más cercanos del difunto.
    En todo caso, practicados el examen o la autopsia correspondientes se procederá a la inmediata sepultura del cadáver.

    Artículo 203.- Pruebas caligráficas. El fiscal podrá solicitar al imputado que escriba en su presencia algunas palabras o frases, a objeto de practicar las pericias caligráficas que considerare necesarias para la investigación. Si el imputado se negare a hacerlo, el fiscal podrá solicitar al juez de garantía la autorización correspondiente.
    Artículo 204.- Entrada y registro en lugares de libre acceso público. Carabineros de Chile y la Policía de Investigaciones podrán efectuar el registro de lugares y recintos de libre acceso público, en búsqueda del imputado contra el cual se hubiere librado orden de detención, o de rastros o huellas del hecho investigado o medios que pudieren servir a la comprobación del mismo.
    Artículo 205.- Entrada y registro en lugares cerrados. Cuando se presumiere que el imputado, o medios de comprobación del hecho que se investigare, se encontrare en un determinado edificio o lugar cerrado, se podrá entrar al mismo y proceder al registro, siempre que su propietario o encargado consintiere expresamente en la práctica de la diligencia.
    En este caso, el funcionario que practicare el registro deberá individualizarse y cuidará que la diligencia se realizare causando el menor daño y las menores molestias posibles a los ocupantes. Asimismo, entregará al propietario o encargado un certificado que acredite el hecho del registro, la individualización de los funcionarios que lo hubieren practicado y de aquél que lo hubiere ordenado.
    Si, por el contrario, el propietario o el encargado del edificio o lugar no permitiere la entrada y registro, la policía adoptará las medidas tendientes a evitar la posible fuga del imputado y el fiscal solicitará al juez la autorización para proceder a la diligencia. En todo caso, el fiscal hará saber al juez las razones que el propietario o encargado hubiere invocado para negar la entrada y registro.
    Artículo 206.- Entrada y registro en lugares cerrados sin autorización u orden. La policía podrá entrar en un lugar cerrado y registrarlo, sin el consentimiento expreso de su propietario o encargado ni autorización u orden previa, cuando las llamadas de auxilio de personas que se encontraren en el interior u otros signos evidentes indicaren que en el recinto se está cometiendo un delito, o que exista algún indicio de que se está procediendo a la destrucción de objetos o documentos, de cualquier clase, que pudiesen haber servido o haber estado destinados a la comisión de un hecho constitutivo de delito, o aquellos que de éste provinieren.
    De dicho procedimiento deberá darse comunicación al fiscal inmediatamente terminado y levantarse un acta circunstanciada que será enviada a éste dentro de las doce horas siguientes. Copia de dicha acta se entregará al propietario o encargado del lugar.
    Tratándose del delito de abigeato, la policía podrá ingresar a los predios cuando existan indicios o sospechas de que se está perpetrando dicho ilícito, siempre que las circunstancias hagan temer que la demora en obtener la autorización del propietario o del juez, en su caso, facilitará la concreción del mismo o la impunidad de sus hechores.






    Artículo 207.- Horario para el registro. El registro deberá hacerse en el tiempo que media entre las seis y las veintidós horas; pero podrá verificarse fuera de estas horas en lugares de libre acceso público y que se encontraren abiertos durante la noche. Asimismo, procederá en casos urgentes, cuando su ejecución no admitiere demora. En este último evento, la resolución que autorizare la entrada y el registro deberá señalar expresamente el motivo de la urgencia.
    Artículo 208.- Contenido de la orden de registro. La orden que autorizare la entrada y registro deberá señalar:
    a) El o los edificios o lugares que hubieren de ser registrados;
    b) El fiscal que lo hubiere solicitado;
    c) La autoridad encargada de practicar el registro, y
    d) El motivo del registro y, en su caso, del ingreso nocturno.
    La orden tendrá una vigencia máxima de diez días, después de los cuales caducará la autorización. Con todo, el juez que emitiere la orden podrá establecer un plazo de vigencia inferior.
    Artículo 209.- Entrada y registro en lugares especiales. Para proceder al examen y registro de lugares religiosos, edificios en que funcionare alguna autoridad pública o recintos militares, el fiscal deberá oficiar previamente a la autoridad o persona a cuyo cargo estuvieren, informando de la práctica de la actuación. Dicha comunicación deberá ser remitida con al menos 48 horas de anticipación y contendrá las señas de lo que hubiere de ser objeto del registro, a menos que fuere de temer que por dicho aviso pudiere frustrarse la diligencia. Además, en ella se indicará a las personas que lo acompañarán e invitará a la autoridad o persona a cargo del lugar, edificio o recinto a presenciar la actuación o a nombrar a alguna persona que asista.
    Si la diligencia implicare el examen de documentos reservados o de lugares en que se encontrare información o elementos de dicho carácter y cuyo conocimiento pudiere afectar la seguridad nacional, la autoridad o persona a cuyo cargo se encontrare el recinto informará de inmediato y fundadamente de este hecho al Ministro de Estado correspondiente, a través del conducto regular, quien, si lo estimare procedente, oficiará al fiscal manifestando su oposición a la práctica de la diligencia. Tratándose de entidades con autonomía constitucional, dicha comunicación deberá remitirse a la autoridad superior correspondiente.
    En este caso, si el fiscal estimare indispensable la realización de la actuación, remitirá los antecedentes al fiscal regional, quien, si compartiere esa apreciación, solicitará a la Corte Suprema que resuelva la controversia, decisión que se adoptará en cuenta. Mientras estuviere pendiente esa determinación, el fiscal dispondrá el sello y debido resguardo del lugar que debiere ser objeto de la diligencia.
    Regirá, en lo pertinente, lo dispuesto en el artículo 19, y, si la diligencia se llevare a cabo, se aplicará a la información o elementos que el fiscal resolviere incorporar a los antecedentes de la investigación lo dispuesto en el artículo 182.
    Artículo 210.- Entrada y registro en lugares que gozan de inviolabilidad diplomática. Para la entrada y registro de locales de embajadas, residencias de los agentes diplomáticos, sedes de organizaciones y organismos internacionales y de naves y aeronaves que, conforme al Derecho Internacional, gozaren de inviolabilidad, el juez pedirá su consentimiento al respectivo jefe de misión por oficio, en el cual le solicitará que conteste dentro de veinticuatro horas. Este será remitido por conducto del Ministerio de Relaciones Exteriores.
    Si el jefe de misión negare su consentimiento o no contestare en el término indicado, el juez lo comunicará al Ministerio de Relaciones Exteriores. Mientras el Ministro no contestare manifestando el resultado de las gestiones que practicare, el juez se abstendrá de ordenar la entrada en el lugar indicado. Sin perjuicio de ello, se podrán adoptar medidas de vigilancia, conforme a las reglas generales.
    En casos urgentes y graves, podrá el juez solicitar la autorización del jefe de misión directamente o por intermedio del fiscal, quien certificará el hecho de haberse concedido.
    Artículo 211.- Entrada y registro en locales consulares. Para la entrada y registro de los locales consulares o partes de ellos que se utilizaren exclusivamente para el trabajo de la oficina consular, se deberá recabar el consentimiento del jefe de la oficina consular o de una persona que él designare, o del jefe de la misión diplomática del mismo Estado.
    Regirá, en lo demás, lo dispuesto en el artículo precedente.
    Artículo 212. Procedimiento para el registro. La resolución que autorizare la entrada y el registro de un lugar cerrado se notificará al dueño o encargado, invitándolo a presenciar el acto, a menos que el juez de garantía autorizare la omisión de estos trámites sobre la base de antecedentes que hicieren temer que ello pudiere frustrar el éxito de la diligencia.
    Si no fuere habida alguna de las personas expresadas, la notificación se hará a cualquier persona mayor de edad que se hallare en el lugar o edificio, quien podrá, asimismo, presenciar la diligencia.
    Si no se hallare a nadie, se hará constar esta circunstancia en el acta de la diligencia.

    Artículo 213.- Medidas de vigilancia. Aun antes de que el juez de garantía dictare la orden de entrada y registro de que trata el artículo 208, el fiscal podrá disponer las medidas de vigilancia que estimare convenientes para evitar la fuga del imputado o la substracción de documentos o cosas que constituyeren el objeto de la diligencia.
    Artículo 214.- Realización de la entrada y registro. Practicada la notificación a que se refiere el artículo 212, se procederá a la entrada y registro. Si se opusiere resistencia al ingreso, o nadie respondiere a los llamados, se podrá emplear la fuerza pública. En estos casos, al terminar el registro se cuidará que los lugares queden cerrados, a objeto de evitar el ingreso de otras personas en los mismos. Todo ello se hará constar por escrito.
    En los registros se procurará no perjudicar ni molestar al interesado más de lo estrictamente necesario.
    El registro se practicará en un solo acto, pero podrá suspenderse cuando no fuere posible continuarlo, debiendo reanudarse apenas cesare el impedimento.
    Artículo 215.- Objetos y documentos no relacionados con el hecho investigado. Si durante la práctica de la diligencia de registro se descubriere objetos o documentos que permitieren sospechar la existencia de un hecho punible distinto del que constituyere la materia del procedimiento en que la orden respectiva se hubiere librado, podrán proceder a su incautación, debiendo dar aviso de inmediato al fiscal, quien los conservará.

    Artículo 216.- Constancia de la diligencia. De todo lo obrado durante la diligencia de registro deberá dejarse constancia escrita y circunstanciada. Los objetos y documentos que se incautaren serán puestos en custodia y sellados, entregándose un recibo detallado de los mismos al propietario o encargado del lugar.
    Si en el lugar o edificio no se descubriere nada sospechoso, se dará testimonio de ello al interesado, si lo solicitare.
    Artículo 217.- Incautación de objetos y documentos. Los objetos y documentos relacionados con el hecho investigado, los que pudieren ser objeto de la pena de comiso y aquellos que pudieren servir como medios de prueba, serán incautados, previa orden judicial librada a petición del fiscal, cuando la persona en cuyo poder se encontraren no los entregare voluntariamente, o si el requerimiento de entrega voluntaria pudiere poner en peligro el éxito de la investigación.
    Si los objetos y documentos se encontraren en poder de una persona distinta del imputado, en lugar de ordenar la incautación, o bien con anterioridad a ello, el juez podrá apercibirla para que los entregue. Regirán, en tal caso, los medios de coerción previstos para los testigos. Con todo, dicho apercibimiento no podrá ordenarse respecto de las personas a quienes la ley reconoce la facultad de no prestar declaración.
    Cuando existieren antecedentes que permitieren presumir suficientemente que los objetos y documentos se encuentran en un lugar de aquellos a que alude el artículo 205 se procederá de conformidad a lo allí prescrito.
    Artículo 218.- Retención e incautación de correspondencia. A petición del fiscal, el juez podrá autorizar, por resolución fundada, la retención de la correspondencia postal, telegráfica o de otra clase y los envíos dirigidos al imputado o remitidos por él, aun bajo nombre supuesto, o de aquéllos de los cuales, por razón de especiales circunstancias, se presumiere que emanan de él o de los que él pudiere ser el destinatario, cuando por motivos fundados fuere previsible su utilidad para la investigación. Del mismo modo, se podrá disponer la obtención de copias o respaldos de la correspondencia electrónica dirigida al imputado o emanada de éste.
    El fiscal deberá examinar la correspondencia o los envíos retenidos y conservará aquellos que tuvieren relación con el hecho objeto de la investigación. Para los efectos de su conservación se aplicará lo dispuesto en el artículo 188. La correspondencia o los envíos que no tuvieren relación con el hecho investigado serán devueltos o, en su caso, entregados a su destinatario, a algún miembro de su familia o a su mandatario o representante legal. La correspondencia que hubiere sido obtenida de servicios de comunicaciones será devuelta a ellos después de sellada, otorgando, en caso necesario, el certificado correspondiente.
 La Ley 21459, Art. 18 N° 1, D.O. 20.06.2022 agregó un Artículo 218 BIS en esta ubicación que depende del siguiente evento para entrar en vigencia: Las modificaciones a la presente norma según lo dispone el artículo segundo transitorio de la Ley 21459 comenzaran a regir transcurridos seis meses desde la publicación en el Diario Oficial de un reglamento dictado por el Ministerio de Transportes y Telecomunicaciones, suscrito además por el Ministro del Interior y Seguridad Pública.
Ver texto diferido
Ver modificatoria

    Artículo 218 ter.- Registros de llamadas y otros antecedentes de tráfico comunicacional. Cuando existan fundadas sospechas basadas en hechos determinados y ello sea útil para la investigación, el Ministerio Público podrá requerir a cualquier proveedor de servicios, previa autorización judicial, que entregue la información que tenga almacenada relativa al tráfico de llamadas telefónicas, de envíos de correspondencia o de tráfico de datos en internet de sus abonados, referida al período de tiempo determinado en la resolución judicial.
    Para efectos de este artículo se entenderá por datos relativos al tráfico todos aquellos referidos a una comunicación realizada por medio de un sistema informático o de telecomunicaciones, generados por este último en tanto elemento de la cadena de comunicación, y que indiquen el origen, el destino, la ruta, la hora, la fecha, el tamaño y la duración de la comunicación o el tipo de servicio subyacente.
    El Ministerio Público podrá requerir, en el marco de una investigación penal en curso y sin autorización judicial, a cualquier proveedor de servicios que ofrezca servicios en territorio chileno, que facilite los datos de suscriptor que posea sobre sus abonados, así como también la información referente a las direcciones IP utilizadas por éstos para facilitar la identificación de quienes corresponda en el marco de la investigación. Los proveedores de servicios deberán mantener el secreto de esta solicitud.
    Por datos de suscriptor se entenderá aquella información que posea un proveedor de servicios relacionada con sus abonados, excluidos los datos sobre tráfico y contenido, y que permita determinar su identidad, tales como la información del nombre del titular del servicio, número de identificación, domicilio, número de teléfono y correo electrónico. Las empresas concesionarias de servicios públicos de telecomunicaciones y proveedores de internet deberán mantener, con carácter reservado y adoptando las medidas de seguridad correspondientes, a disposición del Ministerio Público a efectos de una investigación penal, por un plazo de un año, una nómina y registro actualizado de sus rangos autorizados de direcciones IP y de los números IP de las conexiones que realicen sus clientes o usuarios, con sus correspondientes datos relativos al tráfico, así como los domicilios o residencias de sus clientes o usuarios.
    Los funcionarios públicos, los intervinientes en la investigación penal y los empleados de las empresas mencionadas en este artículo que intervengan en este tipo de requerimientos deberán guardar secreto acerca de ellos, salvo que se les cite a declarar.
    La entrega de los antecedentes deberá realizarse en el plazo que disponga la resolución judicial. Si el requerido estima que no puede cumplir con el plazo en atención al volumen y la naturaleza de la información solicitada o la información no existe o no la posee, deberá comunicar dicha circunstancia fundadamente al tribunal, dentro del término señalado en la resolución judicial respectiva.
    Si a pesar de las medidas señaladas en este artículo la información no es entregada, podrá ser requerida al representante legal de la institución u organización de que se trate, bajo apercibimiento de arresto.
    La infracción a la mantención de la nómina y registro actualizado de los antecedentes a que se refiere el inciso cuarto será castigada según las sanciones y el procedimiento previsto en los artículos 36 y 36 A de la ley N° 18.168, General de Telecomunicaciones. El incumplimiento de las obligaciones de mantener con carácter reservado y adoptar las medidas de seguridad correspondientes de los antecedentes señalados en dicho inciso, será sancionado con la pena prevista en la letra f) del artículo 36 B de la ley N° 18.168. Los registros así obtenidos quedarán bajo custodia del Ministerio Público, quien cuidará que los datos en cuestión no sean conocidos por terceras personas.
    Los registros sólo podrán ser utilizados para los efectos de la investigación en la que fueron solicitados, u otras seguidas por delitos que merezcan pena de crimen o sean propias del sistema de análisis criminal y focos investigativos, de acuerdo con lo establecido en el artículo 37 bis de la ley N° 19.640, que establece la ley orgánica constitucional del Ministerio Público, y no podrán ser utilizados para otros fines.
    El ejercicio de esta facultad se regulará mediante instrucciones generales dictadas por el Fiscal Nacional, conforme a lo establecido en el artículo 17 letra a) de la ley N° 19.640, con el objeto de asegurar su uso racional.


    Artículo 219.- Copias de comunicaciones o transmisiones. El juez de garantía podrá autorizar, a petición del fiscal, que cualquier empresa de comunicaciones facilite copias de las comunicaciones transmitidas o recibidas por ellas. Del mismo modo, podrá ordenar la entrega de las versiones que existieren de las transmisiones de radio, televisión u otros medios.
    Artículo 220.- Objetos y documentos no sometidos a incautación. No podrá disponerse la incautación, ni la entrega bajo el apercibimiento previsto en el inciso segundo del artículo 217:
    a) De las comunicaciones entre el imputado y las personas que pudieren abstenerse de declarar como testigos por razón de parentesco o en virtud de lo prescrito en el artículo 303;
    b) De las notas que hubieren tomado las personas mencionadas en la letra a) precedente, sobre comunicaciones confiadas por el imputado, o sobre cualquier circunstancia a la que se extendiere la facultad de abstenerse de prestar declaración, y c) De otros objetos o documentos, incluso los resultados de exámenes o diagnósticos relativos a la salud del imputado, a los cuales se extendiere naturalmente la facultad de abstenerse de prestar declaración.
    Las limitaciones previstas en este artículo sólo regirán cuando las comunicaciones, notas, objetos o documentos se encontraren en poder de las personas a quienes la ley reconoce la facultad de no prestar declaración; tratándose de las personas mencionadas en el artículo 303, la limitación se extenderá a las oficinas o establecimientos en los cuales ellas ejercieren su profesión o actividad.
    Asimismo, estas limitaciones no regirán cuando las personas facultadas para no prestar testimonio fueren imputadas por el hecho investigado o cuando se tratare de objetos y documentos que pudieren caer en comiso, por provenir de un hecho punible o haber servido, en general, a la comisión de un hecho punible.
    En caso de duda acerca de la procedencia de la incautación, el juez podrá ordenarla por resolución fundada. Los objetos y documentos así incautados serán puestos a disposición del juez, sin previo examen del fiscal o de la policía, quien decidirá, a la vista de ellos, acerca de la legalidad de la medida. Si el juez estimare que los objetos y documentos incautados se encuentran entre aquellos mencionados en este artículo, ordenará su inmediata devolución a la persona respectiva. En caso contrario, hará entrega de los mismos al fiscal, para los fines que éste estimare convenientes.
    Si en cualquier momento del procedimiento se constatare que los objetos y documentos incautados se encuentran entre aquellos comprendidos en este artículo, ellos no podrán ser valorados como medios de prueba en la etapa procesal correspondiente.
    Artículo 221.- Inventario y custodia. De toda diligencia de incautación se levantará inventario, conforme a las reglas generales. El encargado de la diligencia otorgará al imputado o a la persona que los hubiere tenido en su poder un recibo detallado de los objetos y documentos incautados.
    Los objetos y documentos incautados serán inventariados y sellados y se pondrán bajo custodia del ministerio público en los términos del artículo 188.

    I. Interceptación de comunicaciones


    Artículo 222.- Ámbito de aplicación. El juez de garantía, a petición del Ministerio Público, podrá ordenar la interceptación y grabación de las comunicaciones telefónicas o de otras formas de comunicación cuando existan fundadas sospechas basadas en hechos determinados de que una persona ha cometido o participado en la preparación o comisión, o que ella prepara actualmente la comisión o participación en un delito al que la ley le asigna pena de crimen, y la investigación de tales delitos lo haga imprescindible.
    La orden a que se refiere el inciso precedente sólo podrá afectar al imputado o a personas respecto de las cuales existieren fundadas sospechas basadas en hechos determinados, de que sirven de intermediarias de dichas comunicaciones y, asimismo, de aquellas que facilitaren sus medios de comunicación al imputado o sus intermediarios y la investigación de tales delitos lo hiciere imprescindible.
    No se podrán interceptar las comunicaciones entre el imputado y su abogado, a menos que el juez de garantía lo ordenare, por estimar fundadamente, sobre la base de hechos determinados de los que dejará constancia en la respectiva resolución, que el abogado pudiere tener responsabilidad penal en los hechos investigados.
    La orden que disponga la interceptación y grabación deberá consignar las circunstancias necesarias para individualizar o determinar al afectado por la medida y, de ser posible, los datos que permitan singularizar los medios de comunicación o telecomunicación a intervenir y grabar, tales como números de líneas telefónicas, direcciones IP, casillas de correos, entre otros. También señalará la autoridad o funcionario policial que se encargará de la diligencia de interceptación y grabación, la forma de la interceptación, su alcance y su duración.
    La interceptación no podrá exceder de sesenta días. El juez podrá prorrogar este plazo por períodos de hasta igual duración, para lo cual deberá examinar cada vez la concurrencia de los requisitos previstos en los incisos precedentes.
    Las empresas concesionarias de servicios públicos de telecomunicaciones y prestadores de servicios de internet deberán dar cumplimiento a esta medida, proporcionando a los funcionarios encargados de la diligencia las facilidades necesarias para que se lleve a cabo con la oportunidad con que se requiera. Con este objetivo los proveedores de tales servicios deberán mantener, en carácter reservado y bajo las medidas de seguridad correspondientes, a disposición del Ministerio Público, un listado actualizado de sus rangos autorizados de direcciones IP y un registro, no inferior a un año, de los números IP de las conexiones que realicen sus abonados. Transcurrido el plazo máximo de mantención de los datos señalados precedentemente, las empresas y prestadores de servicios deberán destruir en forma segura dicha información. La negativa o entorpecimiento a la práctica de la medida de interceptación y grabación será constitutiva del delito de desacato. Asimismo, los encargados de realizar la diligencia y los empleados de las empresas mencionadas en este inciso deberán guardar secreto acerca de la misma, salvo que se les citare como testigos al procedimiento.
    Si las sospechas tenidas en consideración para ordenar la medida se disiparen o hubiere transcurrido el plazo de duración fijado para la misma, ella deberá ser interrumpida inmediatamente.



    Artículo 223.- Registro de la interceptación. La interceptación de que trata el artículo precedente será registrada mediante su grabación magnetofónica u otros medios técnicos análogos que aseguraren la fidelidad del registro. La grabación será entregada directamente al ministerio público, quien la conservará bajo sello y cuidará que la misma no sea conocida por terceras personas.
    Cuando lo estimare conveniente, el ministerio público podrá disponer la transcripción escrita de la grabación, por un funcionario que actuará, en tal caso, como ministro de fe acerca de la fidelidad de aquélla. Sin perjuicio de ello, el ministerio público deberá conservar los originales de la grabación, en la forma prevista en el inciso precedente.
    La incorporación al juicio oral de los resultados obtenidos de la medida de interceptación se realizará de la manera que determinare el tribunal, en la oportunidad procesal respectiva. En todo caso, podrán ser citados como testigos los encargados de practicar la diligencia.
    Las comunicaciones que resulten impertinentes o irrelevantes para la investigación de los hechos de que se trate serán entregadas, en su oportunidad, a las personas afectadas con la medida. El Ministerio Público destruirá toda transcripción o copia de ellas.
    Lo prescrito en el inciso precedente no regirá respecto de aquellas grabaciones que contengan informaciones relevantes para otros procedimientos seguidos por hechos que puedan constituir un delito al que la ley le asigne pena de crimen, de las cuales se podrá hacer uso conforme a las normas precedentes.

    Artículo 224.- Notificación al afectado por la interceptación. La medida de interceptación será notificada al afectado por la misma con posterioridad a su realización, en cuanto el objeto de la investigación lo permitiere, y en la medida que ello no pusiere en peligro la vida o la integridad corporal de terceras personas. En lo demás regirá lo previsto en el artículo 182.

    Artículo 225.- Prohibición de utilización. Los resultados de la medida de interceptación telefónica o de otras formas de telecomunicaciones no podrán ser utilizados como medios de prueba en el procedimiento, cuando ella hubiere tenido lugar fuera de los supuestos previstos por la ley o cuando no se hubieren cumplido los requisitos previstos en el artículo 222 para la procedencia de la misma.


    Artículo 225 bis.- Registro remoto de equipos informáticos y ámbito de aplicación. A petición fundada del Ministerio Público, el juez de garantía podrá autorizar la utilización de programas computacionales que permitan acceder de manera remota y aprehender el contenido de un dispositivo, computador o sistema informático, sin conocimiento de su usuario, cuando existan fundadas sospechas basadas en hechos determinados, de que una persona ha cometido o participado en la preparación o comisión, o que el delito se esté cometiendo actualmente, o que se esté preparando la comisión o participación en una asociación delictiva o criminal.
    La medida será autorizada por un plazo máximo de 30 días. El juez de garantía podrá prorrogar este plazo por períodos de hasta igual duración, con un máximo de 60 días, para lo cual deberá examinar cada vez la concurrencia de los requisitos previstos en el inciso anterior.


    Artículo 225 ter.- Requisitos de la resolución que autoriza la medida. La resolución judicial que autorice el acceso remoto deberá especificar, a solicitud del fiscal:

    a) Los dispositivos, computadores o sistemas informáticos específicos objeto de la medida y las circunstancias necesarias para individualizar o determinar al afectado por la medida.

    b) El alcance de la medida, la forma en la que se procederá al acceso y aprehensión de contenidos relevantes para la causa y el programa computacional software mediante el cual se realizará acceso remoto.

    c) Los agentes autorizados para la ejecución de la medida.

    d) La autorización, en su caso, para la realización y conservación de copias de los contenidos para la causa.

    e) Las medidas técnicas específicas necesarias para preservar la integridad de los contenidos, así como para impedir el acceso y la supresión de dichos datos del sistema informático objeto de la medida.

    f) La duración precisa de la medida.


    Artículo 225 quáter.- Ampliación del registro. Cuando al ejecutarse el acceso remoto surjan motivos para creer que los contenidos buscados están almacenados en otro sistema informático o en una parte de él, el juez de garantía, a petición fundada del Ministerio Público, podrá autorizar la ampliación de los términos del acceso remoto.
    La resolución judicial que autorice la ampliación del registro deberá especificar los antecedentes señalados en el artículo anterior, que resulten pertinentes para el desarrollo de la ampliación.


    Artículo 225 quinquies.- Deber de colaboración. Los prestadores de servicios de telecomunicaciones, de acceso a una red de telecomunicaciones o de servicios de la sociedad de la información y los titulares o responsables del sistema informático o contenido objeto del acceso remoto, están obligados a colaborar con los funcionarios policiales encargados de ejecutar la medida. Asimismo, están obligados a facilitar la asistencia necesaria para que los contenidos aprehendidos puedan ser objeto de examen y visualización.
    Los sujetos requeridos para prestar la colaboración en este tipo de requerimientos deberán guardar secreto acerca de los mismos, salvo que se les cite a declarar. La ejecución de la técnica de investigación, en los términos de la resolución judicial que la autoriza, no podrá ser objeto de sanción penal o civil.

    Artículo 226.- Otros medios técnicos de investigación. Cuando el procedimiento tenga por objeto la investigación de un hecho punible al que la ley asigna pena de crimen, el juez de garantía podrá ordenar, a petición del Ministerio Público, el empleo de medios tecnológicos para captar, grabar y registrar subrepticiamente imágenes o sonidos en lugares cerrados o que no sean de libre acceso al público, cuando existan fundadas sospechas basadas en hechos determinados y graves que lo hagan imprescindible para el esclarecimiento de los hechos. Regirán, en lo pertinente, las disposiciones de los artículos 222 a 225.


    Párrafo 3° bis Diligencias especiales de investigación aplicables para casos de criminalidad organizada.


    I. Medidas intrusivas referidas a las comunicaciones, imágenes y sonidos, y al registro de equipos informáticos".

    Artículo 226 A.- Ámbito de aplicación. Las técnicas especiales de investigación previstas en este Párrafo serán aplicables en la investigación de hechos que involucren la participación en una asociación delictiva o criminal, de acuerdo con lo previsto en los artículos siguientes.
    Las medidas de retención e incautación de correspondencia y de obtención de copias de comunicaciones o transmisiones serán aplicables a la investigación según lo dispuesto en el artículo 218.
    Las medidas de interceptación y grabación de comunicaciones, de conversaciones o imágenes obtenidos en lugares cerrados o que no sean de libre acceso al público serán aplicables, previa autorización judicial, cuando existan fundadas sospechas, basadas en hechos determinados, de la intervención en una asociación delictiva o criminal y su uso sea imprescindible para el éxito de la investigación.
    El uso y autorización de las medidas intrusivas indicadas en los incisos anteriores se regirán por las reglas generales establecidas en el artículo 222.







    II. Agentes encubiertos, agentes reveladores e informantes



    Artículo 226 B.- Ámbito de aplicación. El Fiscal Regional competente podrá autorizar a funcionarios policiales determinados para que se desempeñen como agentes encubiertos o agentes reveladores cuando sea necesario para lograr el esclarecimiento de hechos que involucren la participación en una asociación delictiva o criminal, establecer la identidad e intervención de sus responsables, conocer los planes de la asociación, y prevenir la comisión de sus delitos o comprobar los que hubieren cometido.
    El Fiscal Regional deberá resolver la solicitud efectuada por el fiscal en un plazo máximo de 72 horas. En caso de negativa, el fiscal podrá solicitar nuevamente autorización para que funcionarios policiales se desempeñen como agentes encubiertos o agentes reveladores, aportando nuevos antecedentes.
    No será necesaria la autorización establecida en el inciso primero, en aquellos casos en que sea el Fiscal Nacional o el Fiscal Regional quien dirija personalmente la investigación, conforme a lo establecido en los artículos 18 y 19 de la ley N° 19.640.     
    Al autorizar la medida el Fiscal Regional deberá asegurarse que ella se limite a las acciones estrictamente necesarias para los objetivos de la investigación, que los agentes reveladores o infiltrados no induzcan a la perpetración de delitos, y que la seguridad de los agentes reveladores o infiltrados se encuentra debidamente resguardada.
    El acto que autorice la medida será mantenido en poder del Ministerio Público en dos registros distintos. Con todo, la información relativa a la verdadera identidad del agente se mantendrá únicamente en un registro.
    La autorización deberá consignar, además, la identidad supuesta con la que actuará en el caso concreto, si la tuviere. Asimismo, el acto que autorice deberá:

    a) Circunscribir el ámbito de actuación de dichos agentes en conformidad con los antecedentes y el delito o los delitos invocados en la solicitud correspondiente.

    b) Expresar la duración de la autorización, la que no podrá exceder de sesenta días. Ella será prorrogable por períodos iguales, y deberá cumplir los mismos requisitos establecidos para su otorgamiento.

    c) Establecer las medidas que deben adoptar para asegurar los objetivos establecidos en el inciso anterior, incluyendo aquellas previstas en el inciso cuarto del artículo 226 C.

    Si se cumplen las mismas circunstancias indicadas en el inciso primero, el Fiscal Regional podrá autorizar a cualquier persona para que se desempeñe como informante.
    Las autorizaciones establecidas en este artículo serán confidenciales y sólo podrán ser conocidas por terceros en los casos señalados en la ley.
    Cuando la ley autorice el conocimiento por parte de terceros, el Ministerio Público pondrá a su disposición el registro que no consigna la información verdadera sobre la identidad de los agentes e informantes. El acceso al registro completo deberá ser autorizado por el juez de garantía competente con audiencia del Ministerio Público y se otorgará la autorización únicamente si es estrictamente necesario, si no pone en peligro la seguridad personal del agente o informante y si existen todas las medidas necesarias para que la información no llegue a terceros. Teniendo en consideración los antecedentes concretos, el juez podrá autorizar el acceso al registro total o parcialmente.

    Artículo 226 C.- Agente encubierto. Agente encubierto es el funcionario policial que oculta su identidad oficial y se involucra o introduce en las asociaciones delictivas o criminales o agrupaciones u organizaciones a que se refiere el artículo anterior, con el objetivo de identificar a los participantes, reunir información y recoger antecedentes necesarios para la investigación.
    El agente encubierto podrá tener una identidad e historia ficticias. La Dirección Nacional del Servicio de Registro Civil e Identificación deberá otorgar los medios necesarios para su oportuna y debida materialización. Los funcionarios policiales que han actuado en una investigación con identidad falsa podrán mantener dicha identidad cuando testifiquen en el proceso que pueda derivarse de los hechos en que hayan intervenido y siempre que así se disponga mediante resolución judicial fundada.
    Asimismo, el Fiscal Regional podrá autorizar la apertura de una cuenta bancaria, la obtención de otras piezas de identidad relevantes tales como una licencia de conducir y la contratación de servicios básicos haciendo uso de la identidad ficticia. El uso de esta facultad se orientará exclusivamente a reforzar la credibilidad de la identidad e historia ficticias. Un reglamento expedido en conjunto por el Ministerio de Justicia y Derechos Humanos y el Ministerio del Interior y Seguridad Pública deberá establecer los procedimientos y condiciones de ejercicio de esta facultad.
    Sin perjuicio de las penas aplicables por la perpetración de otros delitos, el uso manifiestamente indebido de las facultades asociadas a la historia ficticia será sancionado con la pena de presidio menor en su grado mínimo.
    La información que vaya obteniendo el agente encubierto deberá ser puesta a la mayor brevedad posible en conocimiento de quien solicitó la autorización de la medida.

    Artículo 226 D.- Agente revelador. Agente revelador es el funcionario policial que simula requerir de otro la ejecución de una conducta delictiva con el objetivo de lograr la concreción de los propósitos delictivos de éste.
    El agente revelador podrá tener una identidad e historia ficticias. La Dirección Nacional del Servicio de Registro Civil e Identificación deberá otorgar los medios necesarios para la oportuna y debida materialización de aquellas. Los funcionarios policiales que hubiesen actuado en una investigación con identidad falsa podrán mantener dicha identidad cuando testifiquen en el proceso que pudiera derivarse de los hechos en que hubieran intervenido y siempre que así se acuerde mediante resolución judicial fundada.
    La información que obtenga el agente revelador deberá ser puesta a la mayor brevedad posible en conocimiento de quien solicitó la autorización de la medida.

    Artículo 226 E.- Informantes. Informante es quien suministra antecedentes sustanciales a los organismos policiales acerca de la preparación o de la comisión de un delito de asociación delictiva o criminal y requiere de protección.
    La autorización que conceda la calidad de informante deberá ser otorgada por el Fiscal Regional.
    Contando con autorización del Fiscal Regional, el Ministerio Público también podrá disponer que sea tratado como informante quien participe, con su conocimiento y bajo su control, de una operación encubierta o de una entrega vigilada.

    III. Entregas vigiladas



    Artículo 226 F.- Ámbito de aplicación. El Fiscal Regional podrá autorizar la entrega vigilada de objetos cuya fabricación, elaboración, distribución, transporte, comercialización, importación, exportación, posesión, o tenencia esté prohibida o restringida, o los objetos por las que se hayan sustituido total o parcialmente las anteriores mencionadas, de los instrumentos que hayan servido para la comisión de los delitos de que se trate, y de los efectos y ganancias de tales delitos, siempre que ello resulte útil para la investigación de la participación en una asociación delictiva o criminal, o para establecer la identidad e intervención de intervinientes distintos de quienes se encuentran en posesión de los bienes en cuestión.
    Se entenderá por entrega vigilada la técnica consistente en permitir que los objetos a los que se refiere el inciso anterior se trasladen, guarden, intercepten o circulen dentro del territorio nacional, salgan de él o entren en él, sin la interferencia de las policías o del Ministerio Público, pero bajo su conocimiento y vigilancia o control.
    Al autorizar la medida, el Fiscal Regional deberá asegurarse que ella se limite a las acciones estrictamente necesarias para los objetivos de la investigación, que los agentes estatales no induzcan a la perpetración de delitos, que el procedimiento no ponga en riesgo la integridad personal de terceros y que los bienes cuya entrega vigilada se autoriza puedan ser, en definitiva, sujetos a comiso.
    La resolución que autorice la medida deberá:

    a) Delimitar el objeto de la entrega vigilada, así como el tipo y cantidad de las especies de que se trate.

    b) Expresar la duración de la autorización, la que no podrá exceder de sesenta días, y será prorrogable por períodos iguales.

    c) Establecer las medidas que deben ser tomadas para asegurar los objetivos establecidos en el inciso anterior.

    Cuando los objetos se encuentren en zonas sujetas a la potestad aduanera, el Servicio Nacional de Aduanas observará las instrucciones que imparta el Ministerio Público para los efectos de aplicar esta técnica de investigación.
    Cuando la entrega vigilada o controlada deba practicarse total o parcialmente en territorio extranjero, ella se ajustará a lo dispuesto en los acuerdos o tratados internacionales ratificados por Chile y que se encuentren vigentes, si los hubiere.

    Artículo 226 G.- Suspensión de la entrega vigilada. Si las diligencias ponen en peligro la vida o integridad física de los funcionarios policiales o agentes encubiertos o reveladores que intervengan en la operación, la recolección de antecedentes relevantes para la investigación o el aseguramiento de los partícipes, el Ministerio Público podrá disponer la suspensión de la entrega vigilada y solicitar al juez de garantía que autorice la detención de los partícipes y la incautación de los instrumentos, objetos o efectos del delito.

    IV. Disposiciones comunes



    Artículo 226 H.- Exención de responsabilidad criminal. El agente encubierto, el agente revelador, el informante, así como los funcionarios que participen en una entrega vigilada u otra medida dispuesta de conformidad a este Párrafo, estarán exentos de responsabilidad criminal siempre que se trate de aquellos delitos en que deban incurrir o que no hayan podido impedir en cumplimiento de la resolución que autoriza la medida.

    Artículo 226 I.- Prohibición de la inducción a la perpetración de delitos. El agente encubierto, el agente revelador y los funcionarios que participen en una entrega vigilada o en otra medida dispuesta de conformidad a este Párrafo, no podrán inducir a la perpetración de delitos que, de otro modo, no habrían sido cometidos por éste.

    Artículo 226 J.- Secreto y acceso a la información de defensa. El Ministerio Público podrá disponer el secreto de determinadas actuaciones, registros o documentos respecto de uno o más intervinientes, cuando estime que existe riesgo para el éxito de la investigación o para la seguridad de los agentes encubiertos, agentes reveladores, informantes, testigos, peritos y, en general, de quienes hayan cooperado eficazmente en el procedimiento.
    Se aplicará lo dispuesto en el artículo 182. Con todo, el Ministerio Público podrá disponer que se mantenga el secreto hasta el cierre de la investigación. Además deberá adoptar medidas para garantizar que el término del secreto no ponga en riesgo la seguridad de las personas mencionadas en el inciso anterior.
    Tras el cierre de la investigación, el juez de garantía deberá procurar el acceso de la defensa a todos los medios de prueba pertinentes, y sólo lo restringirá en aquellos casos establecidos en el artículo 226 B, inciso final.
    El que de cualquier modo informe, difunda o divulgue información relativa a una investigación amparada por el secreto, incurrirá en la pena de presidio menor en su grado medio a máximo.

    Artículo 226 K.- Extralimitación en el uso de técnicas especiales. Los funcionarios policiales, agentes encubiertos y reveladores que ejecuten las medidas o actuaciones a que se refieren los artículos 226 B, 226 D y 226 F sin observar el objeto o límites impuestos por la autorización respectiva serán sancionados, además de las penas que corresponda por los delitos cometidos, con la pena de suspensión del empleo en su grado máximo y multa de quince a veinte unidades tributarias mensuales.
    La misma pena se aplicará al fiscal que al ejecutar técnicas especiales imparta órdenes que impliquen un abuso en su ejercicio, en atención a lo autorizado por el Fiscal Regional o en la resolución judicial.
    El juez de garantía declarará nulas las actuaciones que excedan manifiestamente el objeto de las técnicas especiales y las excluirá, de conformidad con el artículo 276.
    El agente policial o fiscal del Ministerio Público que perpetre el delito del artículo 269 ter del Código Penal con ocasión del uso de las técnicas especiales referidas en el inciso primero, será castigado con la pena de presidio menor en su grado máximo a presidio mayor en su grado mínimo e inhabilitación especial perpetua para el cargo.

    Artículo 226 L.- Utilización de medios de prueba. Los antecedentes o evidencia obtenidos mediante la aplicación de las facultades previstas en este Párrafo y que resulten irrelevantes para el procedimiento serán entregados o devueltos en su oportunidad a las personas respecto de quienes se solicitó la medida y se destruirá todo registro, transcripción o copia de ellos por el Ministerio Público.
    Lo prescrito en el inciso precedente no regirá respecto de aquellos antecedentes o evidencia que puedan ser útiles o relevantes para otros procedimientos seguidos por hechos en cuya investigación también sean aplicables las disposiciones de este Párrafo, delitos que merezcan pena de crimen o sean propias del sistema de análisis criminal y focos investigativos, de acuerdo con lo dispuesto en el artículo 37 bis de la ley N° 19.640.

    Artículo 226 M.- Rendición de cuentas. El Fiscal Nacional deberá dar cuenta, anualmente, sobre el número de medidas especiales utilizadas de conformidad con este Párrafo, con la ley N° 20.000 y con la ley N° 19.913 y sobre sus efectos, tanto a la Comisión de Seguridad Pública del Senado como a la Comisión de Seguridad Ciudadana de la Cámara de Diputados, en sesiones que tendrán el carácter de reservadas.

    V. De las medidas de protección para agentes encubiertos, reveladores e informantes


    Artículo 226 N.- Medidas especiales de protección. Sin perjuicio de las reglas generales sobre protección a los testigos contempladas en este Código, en cualquier etapa del procedimiento el Ministerio Público dispondrá, de oficio o a petición de parte, las medidas especiales de protección que resulten adecuadas cuando estime, por las circunstancias del caso, que existe riesgo o peligro grave para la vida o la integridad física de un informante, agente encubierto, agente revelador o de un testigo protegido, como asimismo de su cónyuge, conviviente civil, ascendientes, descendientes, hermanos u otras personas a quienes se hallen ligados por relaciones de afecto.
    Para proteger la identidad, domicilio, profesión y lugar de trabajo de los sujetos indicados en el inciso anterior, el fiscal podrá aplicar medidas tales como:

    a) Que en los registros de las diligencias que se practiquen no consten su nombre, apellidos, profesión u oficio, domicilio, lugar de trabajo, ni cualquier otro dato que pueda servir para su identificación. Podrá utilizar una clave u otro mecanismo de verificación para esos efectos.

    b) Que su domicilio, para efectos de notificaciones y citaciones, sea fijado en la sede de la fiscalía o del tribunal. El órgano interviniente deberá hacerlas llegar reservadamente a su destinatario.

    c) Que las diligencias que tengan lugar durante el curso de la investigación a las que deba comparecer como testigo, se realicen en un lugar distinto de aquél donde funciona la fiscalía y de cuya ubicación no se dejará constancia en el registro respectivo.

    Artículo 226 O.- Prohibición de revelación de información. Dispuesta la medida de protección de la identidad a que se refiere el artículo anterior, el tribunal, sin audiencia de los intervinientes, deberá decretar la prohibición de revelar, en cualquier forma, la identidad de los sujetos protegidos o los antecedentes que conduzcan a su identificación. Asimismo, deberá decretar la prohibición para que sean fotografiados, o se capte su imagen a través de cualquier otro medio.
    La infracción de estas prohibiciones será sancionada con la pena de reclusión menor en su grado medio a máximo, tratándose de quien proporcione la información. En caso de que la información fuere difundida por algún medio de comunicación social, se impondrá, además, a su director una multa de diez a cincuenta unidades tributarias mensuales.
    En ningún caso el tribunal podrá fundar la condena únicamente en declaraciones realizadas por agentes encubiertos, agentes reveladores e informantes, respecto de los cuales se haya decretado la prohibición de revelación de su identidad.

    Artículo 226 P.- Declaración en juicio. Las declaraciones de los agentes encubiertos, agentes reveladores o de testigos y peritos a los que se les otorgue la calidad de informantes podrán ser recibidas anticipadamente en conformidad con el artículo 191 cuando se estime necesario para su seguridad personal. En este caso, el juez de garantía podrá disponer que los testimonios de estas personas se presten por cualquier medio idóneo que impida su identificación física normal. Igual sistema de declaración protegida podrá disponerse por el tribunal de juicio oral en lo penal, en su caso.
    Sea que la declaración se preste de manera anticipada o en el desarrollo del juicio oral propiamente tal, el tribunal deberá comprobar en forma previa la identidad del testigo protegido, agente encubierto o revelador o del informante, en particular los antecedentes relativos a sus nombres y apellidos, edad, lugar de nacimiento, estado civil, profesión, industria o empleo y residencia o domicilio. Consignada en el registro tal comprobación, el tribunal podrá resolver que se excluya del debate cualquier referencia a la identidad que pueda poner en peligro su protección.
    En ningún caso las declaraciones de los testigos protegidos, agentes encubiertos o reveladores o de los informantes podrán ser recibidas e introducidas en el juicio sin que la defensa haya podido ejercer su derecho a contrainterrogarlo personalmente, con los resguardos contemplados en los incisos precedentes. Si la declaración se presta de forma anticipada, el juez de garantía podrá disponer el alzamiento del secreto establecido en el artículo 226 J y procurará el acceso de la defensa a todos los medios de prueba pertinentes. Sólo lo restringirá en aquellos casos establecidos en el artículo 226 B, inciso final.
    Dispuesta por el fiscal la protección de la identidad de los testigos o peritos en la etapa de investigación, el tribunal deberá mantenerla, sin perjuicio de los otros derechos que se confieren a los demás intervinientes.

    Artículo 226 Q.- Protección policial. De oficio o a petición del interesado, durante el desarrollo del juicio o incluso una vez que éste ha finalizado, si las circunstancias de peligro se mantienen el fiscal o el tribunal otorgarán protección policial a quien la necesite, de conformidad con lo prevenido en el artículo 308.

    Artículo 226 R.- Medidas de protección complementarias. Las medidas de protección antes descritas podrán ir acompañadas de otras medidas complementarias que se estimen idóneas en función del caso, si fuere necesario.

    Artículo 226 S.- Cambio de identidad. El tribunal podrá autorizar a los agentes encubiertos, reveladores e informantes a cambiar de identidad, con posterioridad al juicio, en caso de ser necesario para su seguridad.
    La Dirección Nacional del Servicio de Registro Civil e Identificación adoptará todos los resguardos necesarios para asegurar el carácter secreto de estas medidas.
    Todas las actuaciones judiciales y administrativas a que dé lugar esta medida serán secretas. El funcionario del Estado que viole este sigilo será sancionado con la pena de presidio menor en sus grados medio a máximo.
    Quienes hayan sido autorizados para cambiar de identidad sólo podrán usar sus nuevos nombres y apellidos en el futuro. El uso malicioso de su anterior identidad será sancionado con la pena de presidio menor en su grado mínimo.

    Artículo 226 T.- Violación del secreto de la investigación y de la identidad. La violación del secreto de la investigación y de la identidad de las personas a que se refieren los artículos precedentes será castigada con presidio menor en su grado máximo e inhabilitación absoluta perpetua para cargos u oficios públicos.

    Artículo 226 U.- Valoración de la prueba y condena. El tribunal valorará el testimonio de agentes encubiertos, agentes reveladores e informantes conforme a las reglas de la sana crítica.
    En ningún caso el tribunal podrá fundar la condena únicamente en declaraciones realizadas por agentes encubiertos, agentes reveladores, informantes y testigos protegidos respecto de los cuales se haya decretado la prohibición de revelación de su identidad.

    Artículo 226 V.- Protección de las víctimas. Es deber del Ministerio Público y de las policías otorgar protección a las víctimas de delitos o de amenazas emanadas de asociaciones delictivas o criminales. El fiscal podrá utilizar o solicitar, según sea el caso, la aplicación de las medidas previstas en este Párrafo, aun cuando la víctima no intervenga como testigo o informante.

    VI. Regla común al presente Párrafo



    Artículo 226 W.- Hallazgo casual con ocasión de diligencias especiales de investigación. Si con motivo de las diligencias especiales de investigación previstas en este Párrafo, y en el marco de la autorización concedida por el juez para su ejecución, ocurren hallazgos de objetos, documentos o antecedentes de los cuales no se tenía noticia, que permiten sospechar la existencia de un hecho punible distinto, dichos objetos, documentos o antecedentes podrán ser utilizados para la posterior persecución del delito descubierto, si éste tiene asignado una pena igual o superior a presidio menor en su grado máximo o una pena igual o superior a la del delito objeto de la investigación.
    Lo señalado en el inciso anterior no se aplicará a la interceptación de comunicaciones, las que se regirán por lo indicado en el inciso final del artículo 223.

    Artículo 226 X.- Regla especial referida a delitos terroristas. Cuando se hayan cometido o preparado la comisión de los delitos sancionados en la ley Nº 18.314, las diligencias especiales de investigación previstas en este Párrafo podrán ser utilizadas por el fiscal, sea que se trate de una persona, de una agrupación de dos o más personas o de una asociación delictiva o criminal.
    Párrafo 4º Registros de la investigación
    Artículo 227.- Registro de las actuaciones del ministerio público. El ministerio público deberá dejar constancia de las actuaciones que realizare, tan pronto tuvieren lugar, utilizando al efecto cualquier medio que permitiere garantizar la fidelidad e integridad de la información, así como el acceso a la misma de aquellos que de acuerdo a la ley tuvieren derecho a exigirlo.
    La constancia de cada actuación deberá consignar a lo menos la indicación de la fecha, hora y lugar de realización, de los funcionarios y demás personas que hubieren intervenido y una breve relación de sus resultados.
    Artículo 228.- Registro de las actuaciones policiales. La policía levantará un registro, en el que dejará constancia inmediata de las diligencias practicadas, con expresión del día, hora y lugar en que se hubieren realizado y de cualquier circunstancia que pudiere resultar de utilidad para la investigación. Se dejará constancia en el registro de las instrucciones recibidas del fiscal y del juez.
    El registro será firmado por el funcionario a cargo de la investigación y, en lo posible, por las personas que hubieren intervenido en los actos o proporcionado alguna información.
    En todo caso, estos registros no podrán reemplazar las declaraciones de la policía en el juicio oral.
    Párrafo 5º Formalización de la investigación
    Artículo 229.- Concepto de la formalización de la investigación. La formalización de la investigación es la comunicación que el fiscal efectúa al imputado, en presencia del juez de garantía, de que desarrolla actualmente una investigación en su contra respecto de uno o más delitos determinados.
    Artículo 230.- Oportunidad de la formalización de la investigación. El fiscal podrá formalizar la investigación cuando considerare oportuno formalizar el procedimiento por medio de la intervención judicial.
    Cuando el fiscal debiere requerir la intervención judicial para la práctica de determinadas diligencias de investigación, la recepción anticipada de prueba o la resolución sobre medidas cautelares, estará obligado a formalizar la investigación, a menos que lo hubiere realizado previamente. Exceptúanse los casos expresamente señalados en la ley.
    Artículo 231.- Solicitud de audiencia para la formalización de la investigación. Si el fiscal deseare formalizar la investigación respecto de un imputado que no se encontrare en el caso previsto en el artículo 132, solicitará al juez de garantía la realización de una audiencia en fecha próxima, mencionando la individualización del imputado, la indicación del delito que se le atribuyere, la fecha y lugar de su comisión y el grado de participación del imputado en el mismo.
    A esta audiencia se citará al imputado, a su defensor y a los demás intervinientes en el procedimiento.
    Artículo 232.- Audiencia de formalización de la investigación. En la audiencia, el juez ofrecerá la palabra al fiscal para que exponga verbalmente los cargos que presentare en contra del imputado y las solicitudes que efectuare al tribunal. Enseguida, el imputado podrá manifestar lo que estimare conveniente.
    A continuación el juez abrirá debate sobre las demás peticiones que los intervinientes plantearen.
    El imputado podrá reclamar ante las autoridades del ministerio público, según lo disponga la ley orgánica constitucional respectiva, de la formalización de la investigación realizada en su contra, cuando considerare que ésta hubiere sido arbitraria.
    Artículo 233.- Efectos de la formalización de la investigación. La formalización de la investigación producirá los siguientes efectos:
    a) Suspenderá el curso de la prescripción de la acción penal en conformidad a lo dispuesto en el artículo 96 del Código Penal;
    b) Comenzará a correr el plazo previsto en el artículo 247, y
    c) El ministerio público perderá la facultad de archivar provisionalmente el procedimiento.
    Artículo 234.- Plazo judicial para el cierre de la investigación. Cuando el juez de garantía, de oficio o a petición de alguno de los intervinientes y oyendo al ministerio público, lo considerare necesario con el fin de cautelar las garantías de los intervinientes y siempre que las características de la investigación lo permitieren, podrá fijar en la misma audiencia un plazo para el cierre de la investigación, al vencimiento del cual se producirán los efectos previstos en el artículo 247.
    Artículo 235.- Juicio inmediato. En la audiencia de formalización de la investigación, el fiscal podrá solicitar al juez que la causa pase directamente a juicio oral. Si el juez acogiere dicha solicitud, en la misma audiencia el fiscal deberá formular verbalmente su acusación y ofrecer prueba. También en la audiencia el querellante podrá adherirse a la acusación del fiscal o acusar particularmente y deberá indicar las pruebas de que pensare valerse en el juicio. El imputado podrá realizar las alegaciones que correspondieren y ofrecer, a su turno, prueba.
    Al término de la audiencia, el juez dictará auto de apertura del juicio oral. No obstante, podrá suspender la audiencia y postergar esta resolución, otorgando al imputado un plazo no menor de quince ni mayor de treinta días, dependiendo de la naturaleza del delito, para plantear sus solicitudes de prueba.
    Las resoluciones que el juez dictare en conformidad a lo dispuesto en este artículo no serán susceptibles de recurso alguno.
    Artículo 236.- Autorización para practicar diligencias sin conocimiento del afectado. Las diligencias de investigación que de conformidad al artículo 9º requirieren de autorización judicial previa podrán ser solicitadas por el fiscal aun antes de la formalización de la investigación. Si el fiscal requiriere que ellas se llevaren a cabo sin previa comunicación al afectado, el juez autorizará que se proceda en la forma solicitada cuando la gravedad de los hechos o la naturaleza de la diligencia de que se tratare permitiere presumir que dicha circunstancia resulta indispensable para su éxito.
    Si con posterioridad a la formalización de la investigación el fiscal solicitare proceder de la forma señalada en el inciso precedente, el juez lo autorizará cuando la reserva resultare estrictamente indispensable para la eficacia de la diligencia.
    Párrafo 6º Suspensión condicional del procedimiento
y acuerdos reparatorios
    Artículo 237.- Suspensión condicional del procedimiento. El fiscal, con el acuerdo del imputado, podrá solicitar al juez de garantía la suspensión condicional del procedimiento.
    El juez podrá requerir del ministerio público los antecedentes que estimare necesarios para resolver.
    La suspensión condicional del procedimiento podrá decretarse:
    a) Si la pena que pudiere imponerse al imputado, en el evento de dictarse sentencia condenatoria, no excediere de tres años de privación de libertad;
    b) Si el imputado no hubiere sido condenado anteriormente por crimen o simple delito, y
    c) Si el imputado no tuviere vigente una suspensión condicional del procedimiento, al momento de verificarse los hechos materia del nuevo proceso.
    La presencia del defensor del imputado en la audiencia en que se ventilare la solicitud de suspensión condicional del procedimiento constituirá un requisito de validez de la misma.
    Si el querellante o la víctima asistieren a la audiencia en que se ventile la solicitud de suspensión condicional del procedimiento, deberán ser oídos por el tribunal.
    Tratándose de imputados por delitos de homicidio, secuestro, robo con violencia o intimidación en las personas o fuerza en las cosas, sustracción de menores, aborto; por los contemplados en los artículos 361 a 366 bis y 367 del Código Penal; por los delitos señalados en los artículos 8º, 9º, 10, 13, 14 y 14 D de la ley Nº17.798; por los delitos o cuasidelitos contemplados en otros cuerpos legales que se cometan empleando alguna de las armas o elementos mencionados en las letras a), b), c), d) y e) del artículo 2º y en el artículo 3º de la citada ley Nº17.798, y por conducción en estado de ebriedad causando la muerte o lesiones graves o gravísimas, el fiscal deberá someter su decisión de solicitar la suspensión condicional del procedimiento al Fiscal Regional.
    Al decretar la suspensión condicional del procedimiento, el juez de garantía establecerá las condiciones a las que deberá someterse el imputado, por el plazo que determine, el que no podrá ser inferior a un año ni superior a tres. Durante dicho período no se reanudará el curso de la prescripción de la acción penal. Asimismo, durante el término por el que se prolongare la suspensión condicional del procedimiento se suspenderá el plazo previsto en el artículo 247.
    La resolución que se pronunciare acerca de la suspensión condicional del procedimiento será apelable por el imputado, por la víctima, por el ministerio público y por el querellante.
    La suspensión condicional del procedimiento no impedirá de modo alguno el derecho a perseguir por la vía civil las responsabilidades pecuniarias derivadas del mismo hecho.





    Artículo 238.- Condiciones por cumplir decretada la suspensión condicional del procedimiento. El juez de garantía dispondrá, según correspondiere, que durante el período de suspensión, el imputado esté sujeto al cumplimiento de una o más de las siguientes condiciones:
    a) Residir o no residir en un lugar determinado;
    b) Abstenerse de frecuentar determinados lugares o
personas;
    c) Someterse a un tratamiento médico, psicológico o
de otra naturaleza;
    d) Tener o ejercer un trabajo, oficio, profesión o
empleo, o asistir a algún programa educacional o de
capacitación;
    e) Pagar una determinada suma, a título de
indemnización de perjuicios, a favor de la víctima o
garantizar debidamente su pago. Se podrá autorizar el
pago en cuotas o dentro de un determinado plazo, el que
en ningún caso podrá exceder el período de suspensión
del procedimiento;
    f) Acudir periódicamente ante el ministerio público
y, en su caso, acreditar el cumplimiento de las demás
condiciones impuestas;
    g) Fijar domicilio e informar al ministerio público
de cualquier cambio del mismo, y
    h) Otra condición que resulte adecuada en
consideración con las circunstancias del caso concreto
de que se tratare y fuere propuesta, fundadamente, por
el Ministerio Público.
    Durante el período de suspensión y oyendo en una
audiencia a todos los intervinientes que concurrieren a
ella, el juez podrá modificar una o más de las
condiciones impuestas.

    Artículo 239.- Revocación de la suspensión condicional. Cuando el imputado incumpliere, sin justificación, grave o reiteradamente las condiciones impuestas, o fuere objeto de una nueva formalización de la investigación por hechos distintos, el juez, a petición del fiscal o la víctima, revocará la suspensión condicional del procedimiento, y éste continuará de acuerdo a las reglas generales.
    Será apelable la resolución que se dictare en conformidad al inciso precedente.
    Artículo 240.- Efectos de la suspensión condicional del procedimiento. La suspensión condicional del procedimiento no extingue las acciones civiles de la víctima o de terceros. Sin embargo, si la víctima recibiere pagos en virtud de lo previsto en el artículo 238, letra e), ellos se imputarán a la indemnización de perjuicios que le pudiere corresponder.
    Transcurrido el plazo que el tribunal hubiere fijado de conformidad al artículo 237, inciso quinto, sin que la suspensión fuere revocada, se extinguirá la acción penal, debiendo el tribunal dictar de oficio o a petición de parte el sobreseimiento definitivo.
    Artículo 241.- Procedencia de los acuerdos reparatorios. El imputado y la víctima podrán convenir acuerdos reparatorios, los que el juez de garantía aprobará, en audiencia a la que citará a los intervinientes para escuchar sus planteamientos, si verificare que los concurrentes al acuerdo hubieren prestado su consentimiento en forma libre y con pleno conocimiento de sus derechos.
    Los acuerdos reparatorios sólo podrán referirse a hechos investigados que afectaren bienes jurídicos disponibles de carácter patrimonial, consistieren en lesiones menos graves o constituyeren delitos culposos.
    Sin perjuicio de lo señalado en los incisos precedentes, los acuerdos reparatorios procederán también respecto de los delitos de los artículos 144 inciso primero, 146, 161-A, 161 B, 231, inciso segundo del 247, 284, 296, 297, 494 Nº 4 y 494 Nº 5, todos del Código Penal. Asimismo, procederán también respecto de los delitos contemplados en el decreto con fuerza de ley Nº 3, de 2006, que fija el texto refundido, coordinado y sistematizado de la ley Nº 19.039, de Propiedad Industrial, y en la ley Nº 17.336, de Propiedad Intelectual.
    En consecuencia, de oficio o a petición del ministerio público, el juez negará aprobación a los acuerdos reparatorios convenidos en procedimientos que versaren sobre hechos diversos de los previstos en los incisos segundo y tercero, o si el consentimiento de los que lo hubieren celebrado no apareciere libremente prestado, o si existiere un interés público prevalente en la continuación de la persecución penal. Se entenderá especialmente que concurre este interés si el imputado hubiere incurrido reiteradamente en hechos como los que se investigaren en el caso particular.

    Artículo 242.- Efectos penales del acuerdo reparatorio. Una vez cumplidas las obligaciones contraídas por el imputado en el acuerdo reparatorio o garantizadas debidamente a satisfacción de la víctima, el tribunal dictará sobreseimiento definitivo, total o parcial, en la causa, con lo que se extinguirá, total o parcialmente, la responsabilidad penal del imputado que lo hubiere celebrado.
    Cuando el imputado incumpliere de forma injustificada, grave o reiterada las obligaciones contraídas, la víctima podrá solicitar que el juez resuelva el cumplimiento de las obligaciones de conformidad al artículo siguiente o que se deje sin efecto el acuerdo reparatorio y se oficie al Ministerio Público a fin de reiniciar la investigación penal. En este último caso, el asunto no será susceptible de un nuevo acuerdo reparatorio.

    Artículo 243.- Efectos civiles del acuerdo reparatorio. Ejecutoriada la resolución judicial que aprobare el acuerdo reparatorio, podrá solicitarse su cumplimiento ante el juez de garantía con arreglo a lo establecido en los artículos 233 y siguientes del Código de Procedimiento Civil.
    El acuerdo reparatorio no podrá ser dejado sin efecto por ninguna acción civil.
    Artículo 244.- Efectos subjetivos del acuerdo reparatorio. Si en la causa existiere pluralidad de imputados o víctimas, el procedimiento continuará respecto de quienes no hubieren concurrido al acuerdo.
    Artículo 245.- Oportunidad para pedir y decretar la suspensión condicional del procedimiento o los acuerdos reparatorios. La suspensión condicional del procedimiento y el acuerdo reparatorio podrán solicitarse y decretarse en cualquier momento posterior a la formalización de la investigación. Si no se planteare en esa misma audiencia la solicitud respectiva, el juez citará a una audiencia, a la que podrán comparecer todos los intervinientes en el procedimiento.
    Una vez declarado el cierre de la investigación, la suspensión condicional del procedimiento y el acuerdo reparatorio sólo podrán ser decretados durante la audiencia de preparación del juicio oral.
    Sin perjuicio de lo señalado precedentemente, podrán, excepcionalmente, solicitarse y decretarse la suspensión condicional del procedimiento y los acuerdos reparatorios, aun cuando hubiere finalizado la audiencia de preparación del juicio oral y hasta antes del envío del auto de apertura al tribunal de juicio oral en lo penal. La solicitud se resolverá de conformidad a lo establecido en el artículo 280 bis.

    Artículo 246.- Registro. El ministerio público llevará un registro en el cual dejará constancia de los casos en que se decretare la suspensión condicional del procedimiento o se aprobare un acuerdo reparatorio.
    El registro tendrá por objeto verificar que el imputado cumpla las condiciones que el juez impusiere al disponer la suspensión condicional del procedimiento, o reúna los requisitos necesarios para acogerse, en su caso, a una nueva suspensión condicional o acuerdo reparatorio.
    El registro será reservado, sin perjuicio del derecho de la víctima de conocer la información relativa al imputado.
    Párrafo 7º Conclusión de la investigación
    Artículo 247.- Plazo para declarar el cierre de la investigación. Transcurrido el plazo de dos años desde la fecha en que la investigación hubiere sido formalizada, el fiscal deberá proceder a cerrarla.
    Si el fiscal no declarare cerrada la investigación en el plazo señalado, el imputado o el querellante podrán solicitar al juez que aperciba al fiscal para que proceda a tal cierre.
    Para estos efectos, el juez citará a los intervinientes a una audiencia y si el fiscal no compareciere, el juez otorgará un plazo máximo de dos días para que éste se pronuncie, dando cuenta de ello al fiscal regional. Transcurrido tal plazo sin que el fiscal se pronuncie o si, compareciendo, se negare a declarar cerrada la investigación, el juez decretará el sobreseimiento definitivo de la causa, informando de ello al fiscal regional a fin de que éste aplique las sanciones disciplinarias correspondientes. Esta resolución será apelable.
    Si el fiscal se allanare a la solicitud de cierre de la investigación, deberá formular en la audiencia la declaración en tal sentido y tendrá el plazo de diez días para deducir acusación.
    Transcurrido este plazo sin que se hubiere deducido acusación, el juez fijará un plazo máximo de dos días para que el fiscal deduzca la acusación, dando cuenta de inmediato de ello al fiscal regional. Transcurrido dicho plazo, el juez, de oficio o a petición de cualquiera de los intervinientes, sin que se hubiere deducido la acusación, en audiencia citada al efecto dictará sobreseimiento definitivo. En este caso, informará de ello al fiscal regional a fin de que éste aplique las sanciones disciplinarias correspondientes.
    El plazo de dos años previsto en este artículo se suspenderá en los casos siguientes:

    a) cuando se dispusiere la suspensión condicional
del procedimiento;

    b) cuando se decretare sobreseimiento temporal de
conformidad a lo previsto en el artículo 252, y

    c) desde que se alcanzare un acuerdo reparatorio
hasta el cumplimiento de las obligaciones contraídas por
el imputado a favor de la víctima o hasta que hubiere
debidamente garantizado su cumplimiento a satisfacción
de esta última.



    Artículo 248- Cierre de la investigación. Practicadas las diligencias necesarias para la averiguación del hecho punible y sus autores, cómplices o encubridores, el fiscal declarará cerrada la investigación y podrá, dentro de los diez días siguientes:
    a) Solicitar el sobreseimiento definitivo o temporal de la causa;
    b) Formular acusación, cuando estimare que la investigación proporciona fundamento serio para el enjuiciamiento del imputado contra quien se hubiere formalizado la misma, o
    c) Comunicar la decisión del ministerio público de no perseverar en el procedimiento, por no haberse reunido durante la investigación los antecedentes suficientes para fundar una acusación.
    La comunicación de la decisión contemplada en la letra c) precedente dejará sin efecto la formalización de la investigación, dará lugar a que el juez revoque las medidas cautelares que se hubieren decretado, y la prescripción de la acción penal continuará corriendo como si nunca se hubiere interrumpido.
    Artículo 249.- Citación a audiencia. Cuando decidiere solicitar el sobreseimiento definitivo o temporal, o comunicar la decisión a que se refiere la letra c) del artículo anterior, el fiscal deberá formular su requerimiento al juez de garantía, quien citará a todos los intervinientes a una audiencia.
    Artículo 250.- Sobreseimiento definitivo. El juez de garantía decretará el sobreseimiento definitivo:
    a) Cuando el hecho investigado no fuere constitutivo de delito;
    b) Cuando apareciere claramente establecida la inocencia del imputado;
    c) Cuando el imputado estuviere exento de responsabilidad criminal en conformidad al artículo 10 del Código Penal o en virtud de otra disposición legal;
    d) Cuando se hubiere extinguido la responsabilidad penal del imputado por algunos de los motivos establecidos en la ley;
    e) Cuando sobreviniere un hecho que, con arreglo a la ley, pusiere fin a dicha responsabilidad, y
    f) Cuando el hecho de que se tratare hubiere sido materia de un procedimiento penal en el que hubiere recaído sentencia firme respecto del imputado.
    El juez no podrá dictar sobreseimiento definitivo respecto de los delitos que, conforme a los tratados internacionales ratificados por Chile y que se encuentren vigentes, sean imprescriptibles o no puedan ser amnistiados, salvo en los casos de los números 1° y 2° del artículo 93 del Código Penal.

    Artículo 251.- Efectos del sobreseimiento definitivo. El sobreseimiento definitivo pone término al procedimiento y tiene la autoridad de cosa juzgada.
    Artículo 252.- Sobreseimiento temporal. El juez de garantía decretará el sobreseimiento temporal en los siguientes casos:
    a) Cuando para el juzgamiento criminal se
requiriere la resolución previa de una cuestión civil,
de acuerdo con lo dispuesto en el artículo 171;
    b) Cuando el imputado no compareciere al
procedimiento y fuere declarado rebelde, de acuerdo con
lo dispuesto en los artículos 99 y siguientes, y
    c) Cuando, después de cometido el delito, el
imputado cayere en enajenación mental, de acuerdo con lo
dispuesto en el Título VII del Libro Cuarto.
    El tribunal de juicio oral en lo penal dictará
sobreseimiento temporal cuando el acusado no hubiere
comparecido a la audiencia del juicio oral y hubiere
sido declarado rebelde de conformidad a lo dispuesto en
los artículos 100 y 101 de este Código.

    Artículo 253.- Recursos. El sobreseimiento sólo será impugnable por la vía del recurso de apelación ante la Corte de Apelaciones respectiva.
    Artículo 254.- Reapertura del procedimiento al cesar la causal de sobreseimiento temporal. A solicitud del fiscal o de cualquiera de los restantes intervinientes, el juez podrá decretar la reapertura del procedimiento cuando cesare la causa que hubiere motivado el sobreseimiento temporal.
    Artículo 255.- Sobreseimiento total y parcial. El sobreseimiento será total cuando se refiriere a todos los delitos y a todos los imputados; y parcial cuando se refiriere a algún delito o a algún imputado, de los varios a que se hubiere extendido la investigación y que hubieren sido objeto de formalización de acuerdo al artículo 229.
    Si el sobreseimiento fuere parcial, se continuará el procedimiento respecto de aquellos delitos o de aquellos imputados a que no se extendiere aquél.
    Artículo 256.- Facultades del juez respecto del sobreseimiento. El juez de garantía, al término de la audiencia a que se refiere el artículo 249, se pronunciará sobre la solicitud de sobreseimiento planteada por el fiscal. Podrá acogerla, sustituirla, decretar un sobreseimiento distinto del requerido o rechazarla, si no la considerare procedente. En este último caso, dejará a salvo las atribuciones del ministerio público contempladas en las letras b) y c) del artículo 248.
    Artículo 257. Reapertura de la investigación. Dentro de los diez días siguientes al cierre de la investigación, los intervinientes podrán reiterar la solicitud de diligencias precisas de investigación que oportunamente hubieren formulado durante la investigación y que el Ministerio Público hubiere rechazado o respecto de las cuales no se hubiere pronunciado.
    Si el juez de garantía acogiere la solicitud, ordenará al fiscal reabrir la investigación y proceder al cumplimiento de las diligencias, en el plazo que le fijará. Podrá el fiscal, en dicho evento y por una sola vez, solicitar ampliación del mismo plazo.
    El juez no decretará ni renovará aquellas diligencias que en su oportunidad se hubieren ordenado a petición de los intervinientes y no se hubieren cumplido por negligencia o hecho imputable a los mismos, ni tampoco las que fueren manifiestamente impertinentes, las que tuvieren por objeto acreditar hechos públicos y notorios ni, en general, todas aquellas que hubieren sido solicitadas con fines puramente dilatorios.
    Vencido el plazo o su ampliación, o aun antes de ello si se hubieren cumplido las diligencias, el fiscal cerrará nuevamente la investigación y procederá en la forma señalada en el artículo 248.

    Artículo 258.- Forzamiento de la acusación. Si el querellante particular se opusiere a la solicitud de sobreseimiento formulada por el fiscal, el juez dispondrá que los antecedentes sean remitidos al fiscal regional, a objeto que éste revise la decisión del fiscal a cargo de la causa.
    Si el fiscal regional, dentro de los tres días siguientes, decidiere que el ministerio público formulará acusación, dispondrá simultáneamente si el caso habrá de continuar a cargo del fiscal que hasta el momento lo hubiere conducido, o si designará uno distinto. En dicho evento, la acusación del ministerio público deberá ser formulada dentro de los diez días siguientes, de conformidad a las reglas generales.
    Por el contrario, si el fiscal regional, dentro del plazo de tres días de recibidos los antecedentes, ratificare la decisión del fiscal a cargo del caso, el juez podrá disponer que la acusación correspondiente sea formulada por el querellante, quien la habrá de sostener en lo sucesivo en los mismos términos que este Código lo establece para el ministerio público, o bien procederá a decretar el sobreseimiento correspondiente.
    En caso de que el fiscal hubiere comunicado la decisión a que se refiere la letra c) del artículo 248, el querellante podrá solicitar al juez que lo faculte para ejercer los derechos a que se refiere el inciso anterior.
    La resolución que negare lugar a una de las solicitudes que el querellante formulare de conformidad a este artículo será inapelable, sin perjuicio de los recursos que procedieren en contra de aquella que pusiere término al procedimiento.
    Título II
    Preparación del juicio oral

    Párrafo 1º Acusación
    Artículo 259.- Contenido de la acusación. La acusación deberá contener en forma clara y precisa:
    a) La individualización de el o los acusados y de su defensor;
    b) La relación circunstanciada de el o los hechos atribuidos y de su calificación jurídica;
    c) La relación de las circunstancias modificatorias de la responsabilidad penal que concurrieren, aun subsidiariamente de la petición principal;
    d) La participación que se atribuyere al acusado;
    e) La expresión de los preceptos legales aplicables;
    f) El señalamiento de los medios de prueba de que el ministerio público pensare valerse en el juicio;
    g) La pena cuya aplicación se solicitare, y h) En su caso, la solicitud de que se proceda de acuerdo al procedimiento abreviado.
    Si, de conformidad a lo establecido en la letra f) de este artículo, el fiscal ofreciere rendir prueba de testigos, deberá presentar una lista, individualizándolos con nombre, apellidos, profesión y domicilio o residencia, salvo en el caso previsto en el inciso segundo del artículo 307, y señalando, además, los puntos sobre los que habrán de recaer sus declaraciones. En el mismo escrito deberá individualizar, de igual modo, al perito o los peritos cuya comparecencia solicitare, indicando sus títulos o calidades.
    Si el fiscal solicita la aplicación del comiso de ganancias o, de ser procedente, del comiso por valor equivalente de efectos o instrumentos del delito, deberá indicar su monto aproximado y expresar con claridad y precisión los fundamentos de su solicitud, y señalará los medios de prueba de que piensa valerse y dando, en su caso, cumplimiento a lo dispuesto en el inciso precedente.
    La acusación sólo podrá referirse a hechos y personas incluidos en la formalización de la investigación, aunque se efectuare una distinta calificación jurídica. Con todo, en la acusación podrá solicitarse el comiso de ganancias respecto de terceros en los casos previstos por la ley.

    Párrafo 2º Audiencia de preparación del juicio oral
    Artículo 260.- Citación a la audiencia. Presentada la acusación, el juez de garantía ordenará su notificación a todos los intervinientes y citará, dentro de las veinticuatro horas siguientes, a la audiencia de preparación del juicio oral, la que deberá tener lugar en un plazo no inferior a veinticinco ni superior a treinta y cinco días. Al acusado se le entregará la copia de la acusación, en la que se dejará constancia, además, del hecho de encontrarse a su disposición, en el tribunal, los antecedentes acumulados durante la investigación.
    Artículo 261.- Actuación del querellante. Hasta quince días antes de la fecha fijada para la realización de la audiencia de preparación del juicio oral, el querellante, por escrito, podrá:
    a) Adherir a la acusación del ministerio público o acusar particularmente. En este segundo caso, podrá plantear una distinta calificación de los hechos, otras formas de participación del acusado, solicitar otra pena o ampliar la acusación del fiscal, extendiéndola a hechos o a imputados distintos, siempre que hubieren sido objeto de la formalización de la investigación;
    b) Señalar los vicios formales de que adoleciere el escrito de acusación, requiriendo su corrección;
    c) Ofrecer la prueba que estimare necesaria para sustentar su acusación, lo que deberá hacerse en los mismos términos previstos en el artículo 259, y d) Deducir demanda civil, cuando procediere.
    Artículo 262.- Plazo de notificación. Las actuaciones del querellante, las acusaciones particulares, adhesiones y la demanda civil deberán ser notificadas al acusado, a más tardar, diez días antes de la realización de la audiencia de preparación del juicio oral.
    Artículo 263.- Facultades del acusado. Hasta la víspera del inicio de la audiencia de preparación del juicio oral, por escrito, o al inicio de dicha audiencia, en forma verbal, el acusado podrá:
    a) Señalar los vicios formales de que adoleciere el escrito de acusación, requiriendo su corrección;
    b) Deducir excepciones de previo y especial pronunciamiento, y
    c) Exponer los argumentos de defensa que considere necesarios y señalar los medios de prueba cuyo examen en el juicio oral solicitare, en los mismos términos previstos en el artículo 259.
    Artículo 264.- Excepciones de previo y especial pronunciamiento. El acusado podrá oponer como excepciones de previo y especial pronunciamiento las siguientes:
    a) Incompetencia del juez de garantía;
    b) Litis pendencia;
    c) Cosa juzgada;
    d) Falta de autorización para proceder criminalmente, cuando la Constitución o la ley lo exigieren, y
    e) Extinción de la responsabilidad penal.
    Artículo 265.- Excepciones en el juicio oral. No obstante lo dispuesto en el artículo 263, si las excepciones previstas en las letras c) y e) del artículo anterior no fueren deducidas para ser discutidas en la audiencia de preparación del juicio oral, ellas podrán ser planteadas en el juicio oral.
    Párrafo 3º Desarrollo de la audiencia de
preparación del juicio oral
    Artículo 266.- Oralidad e inmediación. La audiencia de preparación del juicio oral será dirigida por el juez de garantía, quien la presenciará en su integridad, se desarrollará oralmente y durante su realización no se admitirá la presentación de escritos.
    Artículo 267.- Resumen de las presentaciones de los intervinientes. Al inicio de la audiencia, el juez de garantía hará una exposición sintética de las presentaciones que hubieren realizado los intervinientes.
    Artículo 268.- Defensa oral del imputado. Si el imputado no hubiere ejercido por escrito las facultades previstas en el artículo 263, el juez le otorgará la oportunidad de efectuarlo verbalmente.
    Artículo 269.- Comparecencia del fiscal y del defensor. La presencia del fiscal y del defensor del imputado durante la audiencia constituye un requisito de validez de la misma.
    Si en la audiencia se ventilare la aprobación de convenciones probatorias, procedimiento abreviado, suspensión condicional del procedimiento o un acuerdo reparatorio, o cualquier otra actuación en que la ley exigiere expresamente la participación del imputado, su presencia constituirá un requisito de validez de aquella.
    La inasistencia o el abandono injustificado de la audiencia por parte del fiscal deberá ser subsanada de inmediato por el tribunal, el que, además, pondrá este hecho en conocimiento del fiscal regional respectivo para que determine la responsabilidad del fiscal ausente, de conformidad a lo que disponga la ley orgánica constitucional del Ministerio Público. Si no compareciere el defensor, el tribunal declarará el abandono de la defensa, designará un defensor de oficio al imputado y dispondrá la suspensión de la audiencia por un plazo que no excediere de cinco días, a objeto de permitir que el defensor designado se interiorice del caso.
    Inciso Suprimido.




    Artículo 270.- Corrección de vicios formales en la audiencia de preparación del juicio oral. Cuando el juez considerare que la acusación del fiscal, la del querellante o la demanda civil adolecen de vicios formales, ordenará que los mismos sean subsanados, sin suspender la audiencia, si ello fuere posible.
    En caso contrario, ordenará la suspensión de la misma por el período necesario para la corrección del procedimiento, el que en ningún caso podrá exceder de cinco días. Transcurrido este plazo, si la acusación del querellante o la demanda civil no hubieren sido rectificadas, se tendrán por no presentadas. Si no lo hubiere sido la acusación del fiscal, a petición de éste, el juez podrá conceder una prórroga hasta por otros cinco días, sin perjuicio de lo cual informará al fiscal regional.
    Si el ministerio público no subsanare oportunamente los vicios, el juez procederá a decretar el sobreseimiento definitivo de la causa, a menos que existiere querellante particular, que hubiere deducido acusación o se hubiere adherido a la del fiscal. En este caso, el procedimiento continuará sólo con el querellante y el ministerio público no podrá volver a intervenir en el mismo.
    La falta de oportuna corrección de los vicios de su acusación importará, para todos los efectos, una grave infracción a los deberes del fiscal.
    Artículo 271.- Resolución de excepciones en la audiencia de preparación del juicio oral. Si el imputado hubiere planteado excepciones de previo y especial pronunciamiento, el juez abrirá debate sobre la cuestión. Asimismo, de estimarlo pertinente, el juez podrá permitir durante la audiencia la presentación de los antecedentes que estimare relevantes para la decisión de las excepciones planteadas.
    El juez resolverá de inmediato las excepciones de incompetencia, litis pendencia y falta de autorización para proceder criminalmente, si hubieren sido deducidas. La resolución que recayere respecto de dichas excepciones será apelable.
    Tratándose de las restantes excepciones previstas en el artículo 264, el juez podrá acoger una o más de las que se hubieren deducido y decretar el sobreseimiento definitivo, siempre que el fundamento de la decisión se encontrare suficientemente justificado en los antecedentes de la investigación. En caso contrario, dejará la resolución de la cuestión planteada para la audiencia del juicio oral. Esta última decisión será inapelable.
    Artículo 272.- Debate acerca de las pruebas ofrecidas por las partes. Durante la audiencia de preparación del juicio oral cada parte podrá formular las solicitudes, observaciones y planteamientos que estimare relevantes con relación a las pruebas ofrecidas por las demás, para los fines previstos en los incisos segundo y tercero del artículo 276.
    Artículo 273.- Conciliación sobre la responsabilidad civil en la audiencia de preparación del juicio oral. El juez deberá llamar al querellante y al imputado a conciliación sobre las acciones civiles que hubiere deducido el primero y proponerles bases de arreglo. Regirán a este respecto los artículos 263 y 267 del Código de Procedimiento Civil.
    Si no se produjere conciliación, el juez resolverá en la misma audiencia las solicitudes de medidas cautelares reales que la víctima hubiere formulado al deducir su demanda civil.
    Artículo 274.- Unión y separación de acusaciones. Cuando el ministerio público formulare diversas acusaciones que el juez considerare conveniente someter a un mismo juicio oral, y siempre que ello no perjudicare el derecho a defensa, podrá unirlas y decretar la apertura de un solo juicio oral, si ellas estuvieren vinculadas por referirse a un mismo hecho, a un mismo imputado o porque debieren ser examinadas unas mismas pruebas.
    El juez de garantía podrá dictar autos de apertura del juicio oral separados, para distintos hechos o diferentes imputados que estuvieren comprendidos en una misma acusación, cuando, de ser conocida en un solo juicio oral, pudiere provocar graves dificultades en la organización o el desarrollo del juicio o detrimento al derecho de defensa, y siempre que ello no implicare el riesgo de provocar decisiones contradictorias.
    Artículo 275.- Convenciones probatorias. Durante la audiencia, el fiscal, el querellante, si lo hubiere, y el imputado podrán solicitar en conjunto al juez de garantía que de por acreditados ciertos hechos, que no podrán ser discutidos en el juicio oral. El juez de garantía podrá formular proposiciones a los intervinientes sobre la materia.
    Si la solicitud no mereciere reparos, por conformarse a las alegaciones que hubieren hecho los intervinientes, el juez de garantía indicará en el auto de apertura del juicio oral los hechos que se dieren por acreditados, a los cuales deberá estarse durante el juicio oral.
    Artículo 276.- Exclusión de pruebas para el juicio oral. El juez de garantía, luego de examinar las pruebas ofrecidas y escuchar a los intervinientes que hubieren comparecido a la audiencia, ordenará fundadamente que se excluyan de ser rendidas en el juicio oral aquellas que fueren manifiestamente impertinentes y las que tuvieren por objeto acreditar hechos públicos y notorios.
    Si estimare que la aprobación en los mismos términos en que hubieren sido ofrecidas las pruebas testimonial y documental produciría efectos puramente dilatorios en el juicio oral, dispondrá también que el respectivo interviniente reduzca el número de testigos o de documentos, cuando mediante ellos deseare acreditar unos mismos hechos o circunstancias que no guardaren pertinencia sustancial con la materia que se someterá a conocimiento del tribunal de juicio oral en lo penal.
    Del mismo modo, el juez excluirá las pruebas que provinieren de actuaciones o diligencias que hubieren sido declaradas nulas y aquellas que hubieren sido obtenidas con inobservancia de garantías fundamentales.
    Las demás pruebas que se hubieren ofrecido serán admitidas por el juez de garantía al dictar el auto de apertura del juicio oral.
    Artículo 277.- Auto de apertura del juicio oral. Al término de la audiencia, el juez de garantía dictará el auto de apertura del juicio oral. Esta resolución deberá indicar:
    a) El tribunal competente para conocer el juicio
oral;
    b) La o las acusaciones que deberán ser objeto del
juicio y las correcciones formales que se hubieren
realizado en ellas;
    c) La demanda civil;
    d) Los hechos que se dieren por acreditados, en
conformidad con lo dispuesto en el artículo 275;
    e) Las pruebas que deberán rendirse en el juicio
oral, de acuerdo a lo previsto en el artículo anterior,
y
    f) La individualización de quienes debieren ser
citados a la audiencia del juicio oral, con mención de
los testigos a los que debiere pagarse anticipadamente
sus gastos de traslado y habitación y los montos
respectivos.
    El auto de apertura del juicio oral sólo será
susceptible del recurso de apelación, cuando lo
interpusiere el ministerio público por la exclusión de
pruebas decretada por el juez de garantía de acuerdo a
lo previsto en el inciso tercero del artículo
precedente. Este recurso será concedido en ambos
efectos. Lo dispuesto en este inciso se entenderá sin
perjuicio de la procedencia, en su caso, del recurso de
nulidad en contra de la sentencia definitiva que se
dictare en el juicio oral, conforme a las reglas
generales.
    Si se excluyeren, por resolución firme, pruebas de
cargo que el Ministerio Público considere esenciales
para sustentar su acusación en el juicio oral
respectivo, el fiscal podrá solicitar el sobreseimiento
definitivo de la causa ante el juez competente, el que
la decretará en audiencia convocada al efecto.

    Artículo 278.- Nuevo plazo para presentar prueba. Cuando, al término de la audiencia, el juez de garantía comprobare que el acusado no hubiere ofrecido oportunamente prueba por causas que no le fueren imputables, podrá suspender la audiencia hasta por un plazo de diez días.
    Artículo 279.- Devolución de los documentos de la investigación. El tribunal devolverá a los intervinientes los documentos que hubieren acompañado durante el procedimiento.
    Artículo 280.- Prueba anticipada. Durante la audiencia de preparación del juicio oral también se podrá solicitar la prueba testimonial anticipada conforme a lo previsto en el artículo 191.
    Si con posterioridad a la realización de la audiencia de preparación del juicio oral, sobreviniere, respecto de los testigos, alguna de las circunstancias señaladas en el inciso segundo del artículo 191, cualquiera de los intervinientes podrá solicitar al juez de garantía, en audiencia especial citada al efecto, la rendición de prueba anticipada.
    Asimismo, se podrá solicitar la declaración de peritos en conformidad con las normas del Párrafo 6º del Título III del Libro Segundo, cuando fuere previsible que la persona de cuya declaración se tratare se encontrará en la imposibilidad de concurrir al juicio oral, por alguna de las razones contempladas en el inciso segundo del artículo 191.
    Para los efectos de lo establecido en los incisos anteriores, el juez de garantía citará a una audiencia especial para la recepción de la prueba anticipada.





NOTA
      El Art. primero transitorio de la ley 21.057, establece que la modificación introducida al presente artículo comenzará a regir de manera gradual, en plazos contados desde la publicación del Reglamento: Primera etapa: seis meses después, respecto de las regiones XV, I, II, VII, XI y XII.  Segunda etapa: dieciocho meses después, respecto de las regiones III, IV, VIII, IX y XIV. Tercera etapa:  treinta meses después, comprendiendo las regiones V, VI, X y Metropolitana.
NOTA 1
      El N° 6 del artículo 7 de la ley 21523, publicada el 31.12.2022, reemplazó en el inciso segundo del presente artículo la frase "la situación señalada en el artículo 191 bis" por: "las situaciones señaladas en los artículos 191 bis y 191 ter"; sin embargo, la frase "o se tratare de la situación señalada en el artículo 191 bis" ya había sido eliminada por el N° 4 del artículo 32 de la ley 21057, cuyo N° 3 además derogó el artículo 191 bis, por lo que no se ha podido incorporar en este texto actualizado.

    Artículo 280 bis.- Audiencia intermedia. Una vez fallado el recurso de apelación contra el auto de apertura del juicio oral o habiendo transcurrido el plazo para interponerlo, y antes de su envío al tribunal de juicio oral en lo penal competente, en conjunto con la solicitud de aplicación del procedimiento abreviado, la suspensión condicional del procedimiento, acuerdos reparatorios o el arribo de convenciones probatorias, se solicitará al juez de garantía, por una única vez, la realización de una nueva audiencia, a efectos de resolver la solicitud.
    La solicitud de nueva audiencia se realizará de común acuerdo entre los intervinientes que correspondan, de conformidad a lo previsto en el artículo 237, si la solicitud se tratare de la aplicación de una suspensión condicional del procedimiento; en el artículo 241, si se tratare de la aplicación de un acuerdo reparatorio; en el artículo 275, si se tratare de convenciones probatorias; o en el artículo 406, si se tratare de la aplicación de un procedimiento abreviado.
    La solicitud suspenderá el plazo de remisión del auto de apertura al tribunal de juicio oral en lo penal competente.
    El juez de garantía citará a la audiencia al fiscal, al imputado, al defensor, a la víctima y al querellante si lo hubiere, dentro del plazo de cinco días contados desde la solicitud.
    Finalizada la audiencia, el juez de garantía procederá conforme a las reglas generales. En el caso de arribarse a convenciones probatorias, el tribunal procederá a la dictación de un nuevo auto de apertura del juicio oral.

    Título III
    Juicio oral

    Párrafo 1º Actuaciones previas al juicio oral




    Artículo 281.- Fecha, lugar, integración y citaciones. El juez de garantía hará llegar el auto de apertura del juicio oral al tribunal competente, no antes de las veinticuatro horas ni después de las setenta y dos horas siguientes al momento en que quedare firme.
    También pondrá a disposición del tribunal de juicio oral en lo penal las personas sometidas a prisión preventiva o a otras medidas cautelares personales.
    Una vez distribuida la causa, cuando procediere, el juez presidente de la sala respectiva procederá de inmediato a decretar la fecha para la celebración de la audiencia del mismo, la que deberá tener lugar no antes de quince ni después de sesenta días desde la notificación del auto de apertura del juicio oral.
    Señalará, asimismo, la localidad en la cual se constituirá y funcionará el tribunal de juicio oral en lo penal, si se tratare de alguno de los casos previstos en el artículo 21 A del Código Orgánico de Tribunales.
    En su resolución, el juez presidente indicará también el nombre de los jueces que integrarán la sala. Con la aprobación del juez presidente del comité de jueces, convocará a un número de jueces mayor de tres para que la integren, cuando existieren circunstancias que permitieren presumir que con el número ordinario no se podrá dar cumplimiento a lo exigido en el artículo 284.
    Ordenará, por último, que se cite a la audiencia de todos quienes debieren concurrir a ella. El acusado deberá ser citado con, a lo menos, siete días de anticipación a la realización de la audiencia, bajo los apercibimientos previstos en los artículos 33 y 141, inciso cuarto.


    Párrafo 2º Principios del juicio oral
    Artículo 282.- Continuidad del juicio oral. La audiencia del juicio oral se desarrollará en forma continua y podrá prolongarse en sesiones sucesivas, hasta su conclusión. Constituirán, para estos efectos, sesiones sucesivas, aquellas que tuvieren lugar en el día siguiente o subsiguiente de funcionamiento ordinario del tribunal.
    Artículo 283.- Suspensión de la audiencia o del juicio oral. El tribunal podrá suspender la audiencia hasta por dos veces solamente por razones de absoluta necesidad y por el tiempo mínimo necesario de acuerdo con el motivo de la suspensión. Al reanudarla, efectuará un breve resumen de los actos realizados hasta ese momento.
    El juicio se suspenderá por las causas señaladas en el artículo 252. Con todo, el juicio seguirá adelante cuando la declaración de rebeldía se produjere respecto del imputado a quien se le hubiere otorgado la posibilidad de prestar declaración en el juicio oral, siempre que el tribunal estimare que su ulterior presencia no resulta indispensable para la prosecución del juicio o cuando sólo faltare la dictación de la sentencia.
    La suspensión de la audiencia o la interrupción del juicio oral por un período que excediere de diez días impedirá su continuación. En tal caso, el tribunal deberá decretar la nulidad de lo obrado en él y ordenar su reinicio.
    En aquellos casos en que, debido al número de imputados, o de querellantes, o de la prueba ofrecida, el juicio oral se extendiera por más de seis meses, el tribunal podrá suspender la audiencia hasta por tres veces adicionales a las dos señaladas en el inciso primero; y si en las mismas circunstancias el juicio oral se extendiera por más de un año, el tribunal podrá suspender la audiencia hasta por seis veces adicionales a las dos señaladas en el inciso primero. El plazo total de estas suspensiones no podrá extenderse por más de treinta días en el primer caso, ni de sesenta en el segundo.
    Cuando fuere necesario suspender la audiencia, el tribunal comunicará verbalmente la fecha y hora de su continuación, lo que se tendrá como suficiente citación.

    Artículo 284.- Presencia ininterrumpida de los jueces y del ministerio público en el juicio oral. La audiencia del juicio oral se realizará con la presencia ininterrumpida de los jueces que integraren el tribunal y del fiscal, sin perjuicio de lo dispuesto en el artículo 258.
    Lo dispuesto en el inciso final del artículo 76 respecto de la inhabilidad se aplicará también a los casos en que, iniciada la audiencia, faltare un integrante del tribunal de juicio oral en lo penal.
    Cualquier infracción de lo dispuesto en este artículo implicará la nulidad del juicio oral y de la sentencia que se dictare en él.
    Artículo 285.- Presencia del acusado en el juicio oral. El acusado deberá estar presente durante toda la audiencia.
    El tribunal podrá autorizar la salida de la sala del acusado cuando éste lo solicitare, ordenando su permanencia en una sala próxima.
    Asimismo, el tribunal podrá disponer que el acusado abandonare la sala de audiencia, cuando su comportamiento perturbare el orden.
    En ambos casos, el tribunal adoptará las medidas necesarias para asegurar la oportuna comparecencia del acusado.
    El presidente de la sala deberá informar al acusado de lo ocurrido en su ausencia, en cuanto éste reingresare a la sala de audiencia.
    Artículo 286.- Presencia del defensor en el juicio oral. La presencia del defensor del acusado durante toda la audiencia del juicio oral será un requisito de validez del mismo, de acuerdo a lo previsto en el artículo 103.
    La no comparecencia del defensor a la audiencia constituirá abandono de la defensa y obligará al tribunal a la designación de un defensor penal público, de acuerdo con lo dispuesto en el inciso cuarto del artículo 106.
    No se podrá suspender la audiencia por la falta de comparecencia del defensor elegido por el acusado. En tal caso, se designará de inmediato un defensor penal público al que se concederá un período prudente para interiorizarse del caso.

    Artículo 287.- Sanciones al fiscal que no asistiere o abandonare la audiencia injustificadamente. A la inasistencia o abandono injustificado del fiscal a la audiencia del juicio oral o a alguna de sus sesiones, si se desarrollare en varias, se aplicará lo previsto en el inciso segundo del artículo 269.

    Artículo 288.- Ausencia del querellante o de su apoderado en el juicio oral. La no comparecencia del querellante o de su apoderado a la audiencia, o el abandono de la misma sin autorización del tribunal, dará lugar a la declaración de abandono establecida en la letra c) del artículo 120.
    Artículo 289.- Publicidad de la audiencia del juicio oral. La audiencia del juicio oral será pública, pero el tribunal podrá disponer, a petición de parte y por resolución fundada, una o más de las siguientes medidas, cuando considerare que ellas resultan necesarias para proteger la intimidad, el honor o la seguridad de cualquier persona que debiere tomar parte en el juicio o para evitar la divulgación de un secreto protegido por la ley:
    a) Impedir el acceso u ordenar la salida de personas determinadas de la sala donde se efectuare la audiencia;
    b) Impedir el acceso del público en general u ordenar su salida para la práctica de pruebas específicas, y
    c) Prohibir al fiscal, a los demás intervinientes y a sus abogados que entreguen información o formulen declaraciones a los medios de comunicación social durante el desarrollo del juicio.
    Los medios de comunicación social podrán fotografiar, filmar o transmitir alguna parte de la audiencia que el tribunal determinare, salvo que las partes se opusieren a ello. Si sólo alguno de los intervinientes se opusiere, el tribunal resolverá.
    Artículo 290.- Incidentes en la audiencia del juicio oral. Los incidentes promovidos en el transcurso de la audiencia del juicio oral se resolverán inmediatamente por el tribunal. Las decisiones que recayeren sobre estos incidentes no serán susceptibles de recurso alguno.
    Artículo 291.- Oralidad. La audiencia del juicio se desarrollará en forma oral, tanto en lo relativo a las alegaciones y argumentaciones de las partes como a las declaraciones del acusado, a la recepción de las pruebas y, en general, a toda intervención de quienes participaren en ella. Las resoluciones serán dictadas y fundamentadas verbalmente por el tribunal y se entenderán notificadas desde el momento de su pronunciamiento, debiendo constar en el registro del juicio.
    El tribunal no admitirá la presentación de argumentaciones o peticiones por escrito durante la audiencia del juicio oral.
    Sin embargo, quienes no pudieren hablar o no lo supieren hacer en el idioma castellano, intervendrán por escrito o por medio de intérpretes.
    El acusado sordo o que no pudiere entender el idioma castellano será asistido de un intérprete que le comunicará el contenido de los actos del juicio.
    Párrafo 3º Dirección y disciplina
    Artículo 292.- Facultades del juez presidente de la sala en la audiencia del juicio oral. El juez presidente de la sala dirigirá el debate, ordenará la rendición de las pruebas, exigirá el cumplimiento de las solemnidades que correspondieren y moderará la discusión. Podrá impedir que las alegaciones se desvíen hacia aspectos no pertinentes o inadmisibles, pero sin coartar el ejercicio de la acusación ni el derecho a defensa.
    También podrá limitar el tiempo del uso de la palabra a las partes que debieren intervenir durante el juicio, fijando límites máximos igualitarios para todas ellas o interrumpiendo a quien hiciere uso manifiestamente abusivo de su facultad.
    Además, ejercerá las facultades disciplinarias destinadas a mantener el orden y decoro durante el debate, y, en general, a garantizar la eficaz realización del mismo.
    En uso de estas facultades, el presidente de la sala podrá ordenar la limitación del acceso de público a un número determinado de personas. También podrá impedir el acceso u ordenar la salida de aquellas personas que se presentaren en condiciones incompatibles con la seriedad de la audiencia.
    Artículo 293.- Deberes de los asistentes a la audiencia del juicio oral. Quienes asistieren a la audiencia deberán guardar respeto y silencio mientras no estuvieren autorizados para exponer o debieren responder a las preguntas que se les formularen. No podrán llevar armas ni ningún elemento que pudiere perturbar el orden de la audiencia. No podrán adoptar un comportamiento intimidatorio, provocativo o contrario al decoro.
    Artículo 294.- Sanciones. Quienes infringieren las medidas sobre publicidad previstas en el artículo 289 o lo dispuesto en el artículo 293 podrán ser sancionados de conformidad con los artículos 530 ó 532 del Código Orgánico de Tribunales, según correspondiere.
    Sin perjuicio de lo anterior, el tribunal podrá expulsar a los infractores de la sala.
    En caso de que el expulsado fuere el fiscal o el defensor, deberá procederse a su reemplazo antes de continuar el juicio. Si lo fuere el querellante, se procederá en su ausencia y si lo fuere su abogado, deberá reemplazarlo.
    Párrafo 4º Disposiciones generales sobre la prueba
    Artículo 295.- Libertad de prueba. Todos los hechos y circunstancias pertinentes para la adecuada solución del caso sometido a enjuiciamiento podrán ser probados por cualquier medio producido e incorporado en conformidad a la ley.
    Artículo 296.- Oportunidad para la recepción de la prueba. La prueba que hubiere de servir de base a la sentencia deberá rendirse durante la audiencia del juicio oral, salvas las excepciones expresamente previstas en la ley. En estos últimos casos, la prueba deberá ser incorporada en la forma establecida en el Párrafo 9º de este Título.
    Artículo 297.- Valoración de la prueba. Los tribunales apreciarán la prueba con libertad, pero no podrán contradecir los principios de la lógica, las máximas de la experiencia y los conocimientos científicamente afianzados.
    El tribunal deberá hacerse cargo en su fundamentación de toda la prueba producida, incluso de aquélla que hubiere desestimado, indicando en tal caso las razones que hubiere tenido en cuenta para hacerlo.
    La valoración de la prueba en la sentencia requerirá el señalamiento del o de los medios de prueba mediante los cuales se dieren por acreditados cada uno de los hechos y circunstancias que se dieren por probados. Esta fundamentación deberá permitir la reproducción del razonamiento utilizado para alcanzar las conclusiones a que llegare la sentencia.
    Párrafo 5º Testigos
    Artículo 298.- Deber de comparecer y declarar. Toda persona que no se encontrare legalmente exceptuada tendrá la obligación de concurrir al llamamiento judicial practicado con el fin de prestar declaración testimonial; de declarar la verdad sobre lo que se le preguntare y de no ocultar hechos, circunstancias o elementos acerca del contenido de su declaración.
    Para la citación de los testigos regirán las normas previstas en el Párrafo 4º del Título II del Libro Primero.
    En casos urgentes, los testigos podrán ser citados por cualquier medio, haciéndose constar el motivo de la urgencia. Con todo, en estos casos no procederá la aplicación de los apercibimientos previstos en el artículo 33 sino una vez practicada la citación con las formalidades legales.
    Artículo 299.- Renuencia a comparecer o a declarar. Si el testigo legalmente citado no compareciere sin justa causa, se procederá conforme a lo dispuesto en el inciso tercero del artículo 33. Además, podrá imponérsele el pago de las costas provocadas por su inasistencia.
    El testigo que se negare sin justa causa a declarar, será sancionado con las penas que establece el inciso segundo del artículo 240 del Código de Procedimiento Civil.
    Artículo 300.- Excepciones a la obligación de comparecencia. No estarán obligados a concurrir al llamamiento judicial de que tratan los artículos precedentes, y podrán declarar en la forma señalada en el artículo 301:
    a) El Presidente de la República y los ex Presidentes; los Ministros de Estado; los Senadores y Diputados; los miembros de la Corte Suprema; los integrantes del Tribunal Constitucional; el Contralor General de la República y el Fiscal Nacional;
    b) Los Comandantes en Jefe de las Fuerzas Armadas, el General Director de Carabineros de Chile y el Director General de la Policía de Investigaciones de Chile;
    c) Los chilenos o extranjeros que gozaren en el país de inmunidad diplomática, en conformidad a los tratados vigentes sobre la materia, y
    d) Los que, por enfermedad grave u otro impedimento calificado por el tribunal, se hallaren en imposibilidad de hacerlo.
    Con todo, si las personas enumeradas en las letras a), b) y d) renunciaren a su derecho a no comparecer, deberán prestar declaración conforme a las reglas generales. También deberán hacerlo si, habiendo efectuado el llamamiento un tribunal de juicio oral en lo penal, la unanimidad de los miembros de la sala, por razones fundadas, estimare necesaria su concurrencia ante el tribunal.
    Artículo 301.- Declaración de personas exceptuadas. Las personas comprendidas en las letras a), b) y d) del artículo anterior serán interrogadas en el lugar en que ejercieren sus funciones o en su domicilio. A tal efecto, propondrán oportunamente la fecha y el lugar correspondientes. Si así no lo hicieren, los fijará el tribunal. En caso de inasistencia del testigo, se aplicarán las normas generales. A la audiencia ante el tribunal tendrán siempre derecho a asistir los intervinientes. El tribunal podrá calificar las preguntas que se dirigieren al testigo, teniendo en cuenta su pertinencia con los hechos y la investidura o estado del deponente.
    Las personas comprendidas en la letra c) del artículo precedente declararán por informe, si consintieren a ello voluntariamente. Al efecto se les dirigirá un oficio respetuoso, por medio del ministerio respectivo.
    Artículo 302.- Facultad de no declarar por motivos personales. No estarán obligados a declarar el cónyuge o el conviviente del imputado, sus ascendientes o descendientes, sus parientes colaterales hasta el segundo grado de consanguinidad o afinidad, su pupilo o su guardador, su adoptante o adoptado.
    Si se tratare de personas que, por su inmadurez o por insuficiencia o alteración de sus facultades mentales, no comprendieren el significado de la facultad de abstenerse, se requerirá la decisión del representante legal o, en su caso, de un curador
designado al efecto. Si el representante interviniere en el procedimiento, se designará un curador, quien deberá resguardar los intereses del testigo. La sola circunstancia de que el testigo fuere menor de edad no configurará necesariamente alguna de las situaciones previstas en la primera parte de este inciso.
    Las personas comprendidas en este artículo deberán ser informadas acerca de su facultad de abstenerse, antes de comenzar cada declaración. El testigo podrá retractar en cualquier momento el consentimiento que hubiere dado para prestar su declaración. Tratándose de las personas mencionadas en el inciso segundo de este artículo, la declaración se llevará siempre a cabo en presencia del representante legal o curador.

    Artículo 303.- Facultad de abstenerse de declarar por razones de secreto. Tampoco estarán obligadas a declarar aquellas personas que, por su estado, profesión o función legal, como el abogado, médico o confesor, tuvieren el deber de guardar el secreto que se les hubiere confiado, pero únicamente en lo que se refiriere a dicho secreto.
    Las personas comprendidas en el inciso anterior no podrán invocar la facultad allí reconocida cuando se las relevare del deber de guardar secreto por aquel que lo hubiere confiado.
    Artículo 304.- Deber de comparecencia en ambos casos. Los testigos comprendidos en los dos artículos precedentes deberán comparecer a la presencia judicial y explicar los motivos de los cuales surgiere la facultad de abstenerse que invocaren. El tribunal podrá considerar como suficiente el juramento o promesa que los mencionados testigos prestaren acerca de la veracidad del hecho fundante de la facultad invocada.
    Los testigos comprendidos en los dos artículos precedentes estarán obligados a declarar respecto de los demás imputados con quienes no estuvieren vinculados de alguna de las maneras allí descritas, a menos que su declaración pudiere comprometer a aquéllos con quienes existiere dicha relación.
    Artículo 305.- Principio de no autoincriminación. Todo testigo tendrá el derecho de negarse a responder aquellas preguntas cuya respuesta pudiere acarrearle peligro de persecución penal por un delito.
    El testigo tendrá el mismo derecho cuando, por su declaración, pudiere incriminar a alguno de los parientes mencionados en el artículo 302, inciso primero.
    Artículo 306.- Juramento o promesa. Todo testigo, antes de comenzar su declaración, prestará juramento o promesa de decir verdad sobre lo que se le preguntare, sin ocultar ni añadir nada de lo que pudiere conducir al esclarecimiento de los hechos.
    No se tomará juramento o promesa a los testigos menores de dieciocho años, ni a aquellos de quienes el tribunal sospechare que pudieren haber tomado parte en los hechos investigados. Se hará constar en el registro la omisión del juramento o promesa y las causas de ello.
    El tribunal, si lo estimare necesario, instruirá al testigo acerca del sentido del juramento o promesa y de su obligación de ser veraz, así como de las penas con las cuales la ley castiga el delito de falso testimonio en causa criminal.
    Artículo 307.- Individualización del testigo. La declaración del testigo comenzará por el señalamiento de los antecedentes relativos a su persona, en especial sus nombres y apellidos, edad, lugar de nacimiento, estado, profesión, industria o empleo y residencia o domicilio, todo ello sin perjuicio de las excepciones contenidas en leyes especiales.
    Si existiere motivo para temer que la indicación pública de su domicilio pudiere implicar peligro para el testigo u otra persona, el presidente de la sala o el juez, en su caso, podrá autorizar al testigo a no responder a dicha pregunta durante la audiencia.
    Si el testigo hiciere uso del derecho previsto en el inciso precedente, quedará prohibida la divulgación, en cualquier forma, de su identidad o de antecedentes que condujeren a ella. El tribunal deberá decretar esta prohibición. La infracción a esta norma será sancionada con la pena de reclusión mayor en su grado mínimo, tratándose de quien proporcionare la información. En caso que la información fuere difundida por algún medio de comunicación social, además se impondrá a su director una multa de diez a cincuenta ingresos mínimos mensuales.

    Artículo 308.- Protección a los testigos. El tribunal, en casos graves y calificados, o para evitar toda consecuencia negativa que puedan sufrir los testigos con ocasión de su interacción en un juicio oral, podrá, por solicitud de cualquiera de las partes o del propio testigo, disponer medidas especiales destinadas a proteger la seguridad de este último, las que podrán consistir, entre otras, en autorizarlo para deponer vía sistema de vídeo conferencia, separado del resto de la sala de audiencias mediante algún sistema de obstrucción visual, o por otros mecanismos que impidan el contacto directo del testigo con los intervinientes o el público. Dichas medidas durarán el tiempo razonable que el tribunal dispusiere y podrán ser renovadas cuantas veces fuere necesario.
    De igual forma, el ministerio público, de oficio o a petición del interesado, adoptará las medidas que fueren procedentes para conferir al testigo, antes o después de prestadas sus declaraciones, la debida protección.
    Se entenderá que constituye un caso grave y calificado, especialmente cuando existan malos tratos de obra o amenazas en los términos del artículo 296 del Código Penal. Para adoptar esta decisión, el tribunal podrá oír de manera reservada al testigo, sin participación de los intervinientes en el juicio.



    Artículo 309.- Declaración de testigos. En el procedimiento penal no existirán testigos inhábiles. Sin perjuicio de ello, los intervinientes podrán dirigir al testigo, preguntas tendientes a demostrar su credibilidad o falta de ella, la existencia de vínculos con alguno de los intervinientes que afectaren o pudieren afectar su imparcialidad, o algún otro defecto de idoneidad.
    Todo testigo dará razón circunstanciada de los hechos sobre los cuales declarare, expresando si los hubiere presenciado, si los dedujere de antecedentes que le fueren conocidos o si los hubiere oído referir a otras personas.
    Artículo 310.- Testigos menores de edad. El testigo menor de edad sólo será interrogado por el presidente de la sala, debiendo los intervinientes dirigir las preguntas por su intermedio, teniendo éste el deber de impedir que se formulen preguntas que puedan causar sufrimiento o afectación grave de la dignidad del niño, niña o adolescente, a efectos de resguardar su interés superior.



NOTA
      El Art. primero transitorio de la ley 21.057, establece que la modificación introducida al presente artículo comenzará a regir de manera gradual, en plazos contados desde la publicación del Reglamento: Primera etapa: seis meses después, respecto de las regiones XV, I, II, VII, XI y XII.  Segunda etapa: dieciocho meses después, respecto de las regiones III, IV, VIII, IX y XIV.  Tercera etapa:  treinta meses después, comprendiendo las regiones V, VI, X y Metropolitana.
    Artículo 311.- Testigos sordos o mudos. Si el testigo fuere sordo, las preguntas le serán dirigidas por escrito; y si fuere mudo, dará por escrito sus contestaciones.
    Si no fuere posible proceder de esa manera, la declaración del testigo será recibida por intermedio de una o más personas que pudieren entenderse con él por signos o que comprendieren a los sordomudos. Estas personas prestarán previamente el juramento o promesa prescritos en el artículo 306.
    Artículo 312.- Derechos del testigo. El testigo que careciere de medios suficientes o viviere solamente de su remuneración, tendrá derecho a que la persona que lo presentare le indemnice la pérdida que le ocasionare su comparecencia para prestar declaración y le pague, anticipadamente, los gastos de traslado y habitación, si procediere.
    Se entenderá renunciado este derecho si no se ejerciere en el plazo de veinte días, contado desde la fecha en que se prestare la declaración.
    En caso de desacuerdo, estos gastos serán regulados por el tribunal a simple requerimiento del interesado, sin forma de juicio y sin ulterior recurso.
    Tratándose de testigos presentados por el ministerio público, o por intervinientes que gozaren de privilegio de pobreza, la indemnización será pagada anticipadamente por el Fisco y con este fin, tales intervinientes deberán expresar en sus escritos de acusación o contestación el nombre de los testigos a quien debiere efectuarse el pago y el monto aproximado a que el mismo alcanzará.
    Lo prescrito en este artículo se entenderá sin perjuicio de la resolución que recayere acerca de las costas de la causa.

    Artículo 313.- Efectos de la comparecencia respecto de otras obligaciones similares. La comparecencia del testigo a la audiencia a la que debiere concurrir, constituirá siempre suficiente justificación cuando su presencia fuere requerida simultáneamente para dar cumplimiento a obligaciones laborales, educativas o de otra naturaleza y no le ocasionará consecuencias jurídicas adversas bajo circunstancia alguna.
    Párrafo 6º Informe de peritos
    Artículo 314. - Procedencia del informe de peritos. El ministerio público y los demás intervinientes podrán presentar informes elaborados por peritos de su confianza y solicitar en la audiencia de preparación del juicio oral que éstos fueren citados a declarar a dicho juicio, acompañando los comprobantes que acreditaren la idoneidad profesional del perito.
    Procederá el informe de peritos en los casos determinados por la ley y siempre que para apreciar algún hecho o circunstancia relevante para la causa fueren necesarios o convenientes conocimientos especiales de una ciencia, arte u oficio.
    Los informes deberán emitirse con imparcialidad, ateniéndose a los principios de la ciencia o reglas del arte u oficio que profesare el perito.

    Artículo 315.- Contenido del informe de peritos. Sin perjuicio del deber de los peritos de concurrir a declarar ante el tribunal acerca de su informe, éste deberá entregarse por escrito y contener:
    a) La descripción de la persona o cosa que fuere
objeto de él, del estado y modo en que se hallare;
    b) La relación circunstanciada de todas las
operaciones practicadas y su resultado, y
    c) Las conclusiones que, en vista de tales datos,
formularen los peritos conforme a los principios de su
ciencia o reglas de su arte u oficio.
    No obstante, de manera excepcional, las pericias
consistentes en análisis de alcoholemia, de ADN y
aquellas que recayeren sobre sustancias estupefacientes
o psicotrópicas, podrán ser incorporadas al juicio oral
mediante la sola presentación del informe respectivo.
Sin embargo, si alguna de las partes lo solicitare
fundadamente, la comparecencia del perito no podrá ser
substituida por la presentación del informe.

    Artículo 316.- Admisibilidad del informe y remuneración de los peritos. El juez de garantía admitirá los informes y citará a los peritos cuando, además de los requisitos generales para la admisibilidad de las solicitudes de prueba, considerare que los peritos y sus informes otorgan suficientes garantías de seriedad y profesionalismo. Con todo, el juez de garantía podrá limitar el número de informes o de peritos, cuando unos u otros resultaren excesivos o udieren entorpecer la realización del juicio.
    Los honorarios y demás gastos derivados de la intervención de los peritos mencionados en este artículo corresponderán a la parte que los presentare.
    Excepcionalmente, el juez de garantía podrá relevar a la parte, total o parcialmente, del pago de la remuneración del perito, cuando considerare que ella no cuenta con medios suficientes para solventarlo o cuando, tratándose del imputado, la no realización de la diligencia pudiere importar un notorio desequilibrio en sus posibilidades de defensa. En este último caso, el juez de garantía regulará prudencialmente la remuneración del perito, teniendo presente los honorarios habituales en la plaza y el total o la parte de la remuneración que no fuere asumida por el solicitante será de cargo fiscal.

    Artículo 317.- Incapacidad para ser perito. No podrán desempeñar las funciones de peritos las personas a quienes la ley reconociere la facultad de abstenerse de prestar declaración testimonial.
    Artículo 318.- Improcedencia de inhabilitación de los peritos. Los peritos no podrán ser inhabilitados. No obstante, durante la audiencia del juicio oral podrán dirigírseles preguntas orientadas a determinar su imparcialidad e idoneidad, así como el rigor técnico o científico de sus conclusiones. Las partes o el tribunal podrán requerir al perito información acerca de su remuneración y la adecuación de ésta a los montos usuales para el tipo de trabajo realizado.
    Artículo 319.- Declaración de peritos. La declaración de los peritos en la audiencia del juicio oral se regirá por las normas previstas en el artículo 329 y, supletoriamente, por las establecidas para los testigos.
Si el perito se negare a prestar declaración, se le aplicará lo dispuesto para los testigos en el artículo 299 inciso segundo.
    Artículo 320.- Instrucciones necesarias para el trabajo de los peritos. Durante la etapa de investigación o en la audiencia de preparación del juicio oral, los intervinientes podrán solicitar del juez de garantía que dicte las instrucciones necesarias para que sus peritos puedan acceder a examinar los objetos, documentos o lugares a que se refiriere su pericia o para cualquier otro fin pertinente. El juez de garantía accederá a la solicitud, a menos que, presentada durante la etapa de investigación, considerare necesario postergarla para proteger el éxito de ésta.
    Artículo 321.- Auxiliares del ministerio público como peritos. El ministerio público podrá presentar como peritos a los miembros de los organismos técnicos que le prestaren auxilio en su función investigadora, ya sea que pertenecieren a la policía, al propio ministerio público o a otros organismos estatales especializados en tales funciones.
    Artículo 322.- Terceros involucrados en el procedimiento. En caso necesario, los peritos y otros terceros que debieren intervenir en el procedimiento para efectos probatorios podrán pedir al ministerio público que adopte medidas tendientes a que se les brinde la protección prevista para los testigos.
    Párrafo 7º Otros medios de prueba
    Artículo 323.- Medios de prueba no regulados expresamente. Podrán admitirse como pruebas películas cinematográficas, fotografías, fonografías, videograbaciones y otros sistemas de reproducción de la imagen o del sonido, versiones taquigráficas y, en general, cualquier medio apto para producir fe.
    El tribunal determinará la forma de su incorporación al procedimiento, adecuándola, en lo posible, al medio de prueba más análogo.
    Párrafo 8º Prueba de las acciones civiles
    Artículo 324.- Prueba de las acciones civiles. La prueba de las acciones civiles en el procedimiento criminal se sujetará a las normas civiles en cuanto a la determinación de la parte que debiere probar y a las disposiciones de este Código en cuanto a su procedencia, oportunidad, forma de rendirla y apreciación de su fuerza probatoria.
    Lo previsto en este artículo se aplicará también a las cuestiones civiles a que se refiere el inciso primero del artículo 173 del Código Orgánico de Tribunales.
    Párrafo 9º Desarrollo del juicio oral
    Artículo 325.- Apertura del juicio oral. El día y hora fijados, el tribunal se constituirá con la asistencia del fiscal, del acusado, de su defensor y de los demás intervinientes. Asimismo, verificará la disponibilidad de los testigos, peritos, intérpretes y demás personas que hubieren sido citadas a la audiencia y declarará iniciado el juicio.
    El presidente de la sala señalará las acusaciones que deberán ser objeto del juicio contenidas en el auto de apertura del juicio oral, advertirá al acusado que deberá estar atento a lo que oirá y dispondrá que los peritos y los testigos hagan abandono de la sala de la audiencia.
    Seguidamente concederá la palabra al fiscal, para que exponga su acusación, al querellante para que sostenga la acusación, así como la demanda civil si la hubiere interpuesto.

    Artículo 326.- Defensa y declaración del acusado. Realizadas las exposiciones previstas en el artículo anterior, se le indicará al acusado que tiene la posibilidad de ejercer su defensa en conformidad a lo dispuesto en el artículo 8º.
    Al efecto, se ofrecerá la palabra al abogado defensor, quien podrá exponer los argumentos en que fundare su defensa.
    Asimismo, el acusado podrá prestar declaración. En tal caso, el juez presidente de la sala le permitirá que manifieste libremente lo que creyere conveniente respecto de la o de las acusaciones formuladas. Luego, podrá ser interrogado directamente por el fiscal, el querellante y el defensor, en ese mismo orden. Finalmente, el o los jueces podrán formularle preguntas destinadas a aclarar sus dichos.
    En cualquier estado del juicio, el acusado podrá solicitar ser oído, con el fin de aclarar o complementar sus dichos.
    Artículo 327.- Comunicación entre el acusado y su defensor. El acusado podrá comunicarse libremente con su defensor durante el juicio, siempre que ello no perturbare el orden de la audiencia. No obstante, no podrá hacerlo mientras prestare declaración.
    Artículo 328.- Orden de recepción de las pruebas en la audiencia del juicio oral. Cada parte determinará el orden en que rendirá su prueba, correspondiendo recibir primero la ofrecida para acreditar los hechos y peticiones de la acusación y de la demanda civil y luego la prueba ofrecida por el acusado respecto de todas las acciones que hubieren sido deducidas en su contra.
    Artículo 329.- Peritos y testigos en la audiencia del juicio oral. Durante la audiencia, los peritos y testigos deberán ser interrogados personalmente. Su declaración personal no podrá ser sustituida por la lectura de los registros en que constaren anteriores declaraciones o de otros documentos que las contuvieren, sin perjuicio de lo dispuesto en los artículos 331 y 332.
    El juez presidente de la sala identificará al perito o testigo y ordenará que preste juramento o promesa de decir la verdad.
    La declaración de los testigos se sujetará al interrogatorio de las partes. Los peritos deberán exponer brevemente el contenido y las conclusiones de su informe, y a continuación se autorizará que sean interrogados por las partes. Los interrogatorios serán realizados en primer lugar por la parte que hubiere ofrecido la respectiva prueba y luego por las restantes. Si en el juicio intervinieren como acusadores el ministerio público y el querellante particular, o el mismo se realizare contra dos o más acusados, se concederá sucesivamente la palabra a todos los acusadores o a todos los acusados, según corresponda.
    Finalmente, los miembros del tribunal podrán formular preguntas al testigo o perito con el fin de aclarar sus dichos.
    A solicitud de alguna de las partes, el tribunal podrá autorizar un nuevo interrogatorio de los testigos o peritos que ya hubieren declarado en la audiencia.
    Antes de declarar, los peritos y los testigos no podrán comunicarse entre sí, ni ver, oír ni ser informados de lo que ocurriere en la audiencia.
    Los testigos y peritos que, por algún motivo grave y difícil de superar no pudieren comparecer a declarar a la audiencia del juicio, podrán hacerlo a través de videoconferencia o a través de cualquier otro medio tecnológico apto para su interrogatorio y contrainterrogatorio. La parte que los presente justificará su petición en una audiencia previa que será especialmente citada al efecto, debiendo aquéllos comparecer ante el tribunal con competencia en materia penal más cercano al lugar donde se encuentren.
    Excepcionalmente, en el caso de fallecimiento o incapacidad sobreviniente del perito para comparecer, las pericias podrán introducirse mediante la exposición que realice otro perito de la misma especialidad y que forme parte de la misma institución del fallecido o incapacitado. Esta solicitud se tramitará conforme a lo dispuesto en el artículo 283.


    Artículo 330.- Métodos de interrogación. En sus interrogatorios, las partes que hubieren presentado a un testigo o perito no podrán formular sus preguntas de tal manera que ellas sugirieren la respuesta.
    En relación a la víctima, no se podrán realizar interrogaciones ni contrainterrogatorios que humillen, causen sufrimiento, intimiden o lesionen su dignidad.
    Durante el contrainterrogatorio, las partes podrán confrontar al perito o testigo con su propios dichos u otras versiones de los hechos presentadas en el juicio.
    En ningún caso se admitirán preguntas engañosas, aquéllas destinadas a coaccionar o a acosar ilegítimamente al testigo o perito, ni las que fueren formuladas en términos poco claros para ellos.
    Estas normas se aplicarán al imputado cuando se allanare a prestar declaración.

    Artículo 331.- Reproducción de declaraciones anteriores en la audiencia del juicio oral. Podrá reproducirse o darse lectura a los registros en que constaren anteriores declaraciones de testigos, peritos o imputados, en los siguientes casos:
    a) Cuando se tratare de declaraciones de testigos o peritos que hubieren fallecido o caído en incapacidad física o mental, o estuvieren ausentes del país, o cuya residencia se ignorare o que por cualquier motivo difícil de superar no pudieren declarar en el juicio, siempre que ellas hubieren sido recibidas por el juez de garantía en una audiencia de prueba formal, en conformidad con lo dispuesto en los artículos 191, 192 y 280;
    b) Cuando constaren en registros o dictámenes que todas las partes acordaren en incorporar, con aquiescencia del tribunal;
    c) Cuando la no comparecencia de los testigos, peritos o coimputados fuere imputable al acusado;
    d) Cuando se tratare de declaraciones realizadas por coimputados rebeldes, prestadas ante el juez de garantía, y
    e) Cuando las hipótesis previstas en la letra a) sobrevengan con posterioridad a lo previsto en el artículo 280 y se trate de testigos, o de peritos privados cuya declaración sea considerada esencial por el tribunal, podrá incorporarse la respectiva declaración o pericia mediante la lectura de la misma, previa solicitud fundada de alguno de los intervinientes.
    f) Cuando existan antecedentes fundados sobre la retractación de la víctima, los que serán valorados por el tribunal de acuerdo a lo dispuesto en el artículo 297, teniendo en especial consideración los informes psicológicos acompañados y los antecedentes relativos a la evaluación del riesgo en que se encuentra.



    Artículo 332.- Lectura para apoyo de memoria en la audiencia del juicio oral. Sólo una vez que el acusado o el testigo hubieren prestado declaración, se podrá leer en el interrogatorio parte o partes de sus declaraciones anteriores prestadas ante el fiscal, el abogado asistente del fiscal, en su caso, o el juez de garantía, cuando fuere necesario para ayudar la memoria del respectivo acusado o testigo, para demostrar o superar contradicciones o para solicitar las aclaraciones pertinentes.
    Con los mismos objetivos, se podrá leer durante la declaración de un perito partes del informe que él hubiere elaborado.



    Artículo 333.- Lectura o exhibición de documentos, objetos y otros medios. Los documentos serán leídos y exhibidos en el debate, con indicación de su origen. Los objetos que constituyeren evidencia deberán ser exhibidos y podrán ser examinados por las partes. Las grabaciones, los elementos de prueba audiovisuales, computacionales o cualquier otro de carácter electrónico apto para producir fe, se reproducirán en la audiencia por cualquier medio idóneo para su percepción por los asistentes. El tribunal podrá autorizar, con acuerdo de las partes, la lectura o reproducción parcial o resumida de los medios de prueba mencionados, cuando ello pareciere conveniente y se asegurare el conocimiento de su contenido. Todos estos medios podrán ser exhibidos al acusado, a los peritos o testigos durante sus declaraciones, para que los reconocieren o se refirieren a su conocimiento de ellos.
    Artículo 334.- Prohibición de lectura de registros y documentos. Salvo en los casos previstos en los artículos 331 y 332, no se podrá incorporar o invocar como medios de prueba ni dar lectura durante el juicio oral, a los registros y demás documentos que dieren cuenta de diligencias o actuaciones realizadas por la policía o el ministerio público.
    Ni aun en los casos señalados se podrá incorporar como medio de prueba o dar lectura a actas o documentos que dieren cuenta de actuaciones o diligencias declaradas nulas, o en cuya obtención se hubieren vulnerado garantías fundamentales.
    Artículo 335.- Antecedentes de la suspensión condicional del procedimiento, acuerdos reparatorios y procedimiento abreviado. No se podrá invocar, dar lectura ni incorporar como medio de prueba al juicio oral ningún antecedente que dijere relación con la proposición, discusión, aceptación, procedencia, rechazo o revocación de una suspensión condicional del procedimiento, de un acuerdo reparatorio o de la tramitación de un procedimiento abreviado.
    Artículo 336.- Prueba no solicitada oportunamente. A petición de alguna de las partes, el tribunal podrá ordenar la recepción de pruebas que ella no hubiere ofrecido oportunamente, cuando justificare no haber sabido de su existencia sino hasta ese momento.
    Si con ocasión de la rendición de una prueba surgiere una controversia relacionada exclusivamente con su veracidad, autenticidad o integridad, el tribunal podrá autorizar la presentación de nuevas pruebas destinadas a esclarecer esos puntos, aunque ellas no hubieren sido ofrecidas oportunamente y siempre que no hubiere sido posible prever su necesidad.
    Artículo 337.- Constitución del tribunal en lugar distinto de la sala de audiencias. Cuando lo considerare necesario para la adecuada apreciación de determinadas circunstancias relevantes del caso, el tribunal podrá constituirse en un lugar distinto de la sala de audiencias, manteniendo todas las formalidades propias del juicio.
    Artículo 338.- Alegato final y clausura de la audiencia del juicio oral. Concluida la recepción de las pruebas, el juez presidente de la sala otorgará sucesivamente la palabra al fiscal, al acusador particular, al actor civil y al defensor, para que expongan sus conclusiones. El tribunal tomará en consideración la extensión del juicio para determinar el tiempo que concederá al efecto.
    Seguidamente, se otorgará al fiscal, al acusador particular, al actor civil y al defensor la posibilidad de replicar. Las respectivas réplicas sólo podrán referirse a las conclusiones planteadas por las demás partes.
    Por último, se otorgará al acusado la palabra, para que manifestare lo que estimare conveniente. A continuación se declarará cerrado el debate.

    Párrafo 10° Sentencia definitiva
    Artículo 339.- Deliberación. Inmediatamente después de clausurado el debate, los miembros del tribunal que hubieren asistido a él pasarán a deliberar en privado.
    Artículo 340.- Convicción del tribunal. Nadie podrá ser condenado por delito sino cuando el tribunal que lo juzgare adquiriere, más allá de toda duda razonable, la convicción de que realmente se hubiere cometido el hecho punible objeto de la acusación y que en él hubiere correspondido al acusado una participación culpable y penada por la ley.
    El tribunal formará su convicción sobre la base de la prueba producida durante el juicio oral.
    No se podrá condenar a una persona con el solo mérito de su propia declaración.
    Artículo 341.- Sentencia y acusación. La sentencia condenatoria no podrá exceder el contenido de la acusación. En consecuencia, no se podrá condenar por hechos o circunstancias no contenidos en ella.
    Con todo, el tribunal podrá dar al hecho una calificación jurídica distinta de aquella contenida en la acusación o apreciar la concurrencia de causales modificatorias agravantes de la responsabilidad penal no incluidas en ella, siempre que hubiere advertido a los intervinientes durante la audiencia.
    Si durante la deliberación uno o más jueces consideraren la posibilidad de otorgar a los hechos una calificación distinta de la establecida en la acusación, que no hubiere sido objeto de discusión durante la audiencia, deberán reabrirla, a objeto de permitir a las partes debatir sobre ella.
    Artículo 342.- Contenido de la sentencia. La sentencia definitiva contendrá:
    a) La mención del tribunal y la fecha de su dictación; la identificación del acusado y la de el o los acusadores;
    b) La enunciación breve de los hechos y circunstancias que hubieren sido objeto de la acusación; en su caso, los daños cuya reparación reclamare en la demanda civil y su pretensión reparatoria, y las defensas del acusado;
    c) La exposición clara, lógica y completa de cada uno de los hechos y circunstancias que se dieren por probados, fueren ellos favorables o desfavorables al acusado, y de la valoración de los medios de prueba que fundamentaren dichas conclusiones de acuerdo con lo dispuesto en el artículo 297;
    d) Las razones legales o doctrinales que sirvieren para calificar jurídicamente cada uno de los hechos y sus circunstancias y para fundar el fallo;
    e) La resolución que condenare o absolviere a cada uno de los acusados por cada uno de los delitos que la acusación les hubiere atribuido; la que se pronunciare sobre la responsabilidad civil de los mismos y fijare el monto de las indemnizaciones a que hubiere lugar;
    f) El pronunciamiento sobre las costas de la causa, y
    g) La firma de los jueces que la hubieren dictado.
    La sentencia será siempre redactada por uno de los miembros del tribunal colegiado, designado por éste, en tanto la disidencia o prevención será redactada por su autor. La sentencia señalará el nombre de su redactor y el del que lo sea de la disidencia o prevención.
    Artículo 343.- Decisión sobre absolución o condena. Una vez concluida la deliberación privada de los jueces, de conformidad a lo previsto en el artículo 339, la sentencia definitiva que recayere en el juicio oral deberá ser pronunciada en la audiencia respectiva, comunicándose la decisión relativa a la absolución o condena del acusado por cada uno de los delitos que se le imputaren, indicando respecto de cada uno de ellos los fundamentos principales tomados en consideración para llegar a dichas conclusiones.
    Excepcionalmente, cuando la audiencia del juicio se hubiere prolongado por más de dos días y la complejidad del caso no permitiere pronunciar la decisión inmediatamente, el tribunal podrá prolongar su deliberación hasta por veinticuatro horas, hecho que será dado a conocer a los intervinientes en la misma audiencia, fijándose de inmediato la oportunidad en que la decisión les será comunicada.
    La omisión del pronunciamiento de la decisión de conformidad a lo previsto en los incisos precedentes producirá la nulidad del juicio, el que deberá repetirse en el más breve plazo posible.
    En el caso de condena, el tribunal deberá resolver sobre las circunstancias modificatorias de responsabilidad penal en la misma oportunidad prevista en el inciso primero. No obstante, tratándose de circunstancias ajenas al hecho punible, y los demás factores relevantes para la determinación y cumplimiento de la pena, el tribunal abrirá debate sobre tales circunstancias y factores, inmediatamente después de pronunciada la decisión a que se refiere el inciso primero y en la misma audiencia. Para dichos efectos, el tribunal recibirá los antecedentes que hagan valer los intervinientes para fundamentar sus peticiones, dejando su resolución para la audiencia de lectura de sentencia.

    Artículo 344. Plazo para redacción de la sentencia. Al pronunciarse sobre la absolución o condena, el tribunal podrá diferir la redacción del fallo y, en su caso, la determinación de la pena hasta por un plazo de cinco días, fijando la fecha de la audiencia en que tendrá lugar su lectura. No obstante, si el juicio hubiere durado más de cinco días, el tribunal dispondrá, para la fijación de la fecha de la audiencia para su comunicación, de un día adicional por cada dos de exceso de duración del juicio. En ambos casos, si el vencimiento del plazo para la redacción del fallo coincidiere con un día domingo o festivo, el plazo se diferirá hasta el día siguiente que no sea domingo o festivo. El transcurso de estos plazos sin que hubiere tenido lugar la audiencia citada, constituirá falta grave que deberá ser sancionada disciplinariamente. Sin perjuicio de ello, se deberá citar a una nueva audiencia de lectura de la sentencia, la que en caso alguno podrá tener lugar después del segundo día contado desde la fecha fijada para la primera. Transcurrido este plazo adicional sin que se comunicare la sentencia se producirá la nulidad del juicio, a menos que la decisión hubiere sido la de absolución del acusado. Si, siendo varios los acusados, se hubiere absuelto a alguno de ellos, la repetición del juicio sólo comprenderá a quienes hubieren sido condenados.
    El vencimiento del plazo adicional mencionado en el inciso precedente sin que se diere a conocer el fallo, sea que se produjere o no la nulidad del juicio, constituirá respecto de los jueces que integraren el tribunal una nueva infracción que deberá ser sancionada disciplinariamente.

    Artículo 345.- DEROGADO

    Artículo 346.- Audiencia de comunicación de la sentencia. Una vez redactada la sentencia, de conformidad a lo previsto en el artículo 342, se procederá a darla a conocer en la audiencia fijada al efecto, oportunidad a contar de la cual se entenderá notificada a todas las partes, aun cuando no asistieren a la misma.

    Artículo 347.- Decisión absolutoria y medidas cautelares personales. Comunicada a las partes la decisión absolutoria prevista en el artículo 343, el tribunal dispondrá, en forma inmediata, el alzamiento de las medidas cautelares personales que se hubieren decretado en contra del acusado y ordenará se tome nota de este alzamiento en todo índice o registro público y policial en el que figuraren. También se ordenará la cancelación de las garantías de comparecencia que se hubieren otorgado.

    Artículo 348.- Sentencia condenatoria. La sentencia condenatoria fijará todas las penas principales y accesorias que corresponda imponer, con indicación específica de cada una de ellas, y se pronunciará sobre la eventual aplicación de alguna de las penas sustitutivas a la privación o restricción de libertad previstas en la ley.
    La sentencia que condenare a una pena temporal deberá expresar con toda precisión el día desde el cual empezará ésta a contarse y fijará el tiempo de detención, prisión preventiva y privación de libertad impuesta en conformidad a la letra a) del artículo 155 que deberá servir de abono para su cumplimiento. Para estos efectos, se abonará a la pena impuesta un día por cada día completo, o fracción igual o superior a doce horas, de dichas medidas cautelares que hubiere cumplido el condenado.
    La sentencia condenatoria dispondrá también el comiso de los instrumentos o efectos del delito o su restitución, cuando fuere procedente. En cuanto al comiso de las ganancias del delito o del valor equivalente de efectos o instrumentos del delito, si éstas o aquél ascienden a un monto superior a 400 unidades tributarias mensuales, se estará a lo dispuesto en el artículo siguiente. De lo contrario, el tribunal lo impondrá en la misma sentencia condenatoria si fuere procedente.
    Cuando se hubiere declarado falso, en todo o en parte, un instrumento público, el tribunal, junto con su devolución, ordenará que se lo reconstituya, cancele o modifique de acuerdo con la sentencia.
    Cuando se pronunciare la decisión de condena, el tribunal podrá disponer, a petición de alguno de los intervinientes, la revisión de las medidas cautelares personales, atendiendo al tiempo transcurrido y a la pena probable.





    Artículo 348 bis.- Comiso de ganancias y comiso por valor equivalente. En caso de haberse solicitado la aplicación del comiso de ganancias o de valor equivalente por un monto superior a 400 unidades tributarias mensuales, o si la aplicación del comiso afecta a terceros, en la sentencia condenatoria se citará a una audiencia especial.
    Si el comiso sólo afecta a personas que han sido condenadas, la audiencia tendrá lugar dentro de décimo día a contar de la fecha en que la sentencia quede ejecutoriada. Si el comiso afecta a terceros, la audiencia no podrá tener lugar antes de treinta ni después de sesenta días contados desde la fecha en que la sentencia quede ejecutoriada. En ambos casos, se debe notificar la resolución a los afectados.
    La resolución y la audiencia respectiva se sujetarán a lo dispuesto en los artículos 415 quinquies, 415 sexies y 415 septies.
    El tribunal pronunciará su decisión de imposición del comiso o rechazo de la solicitud. En el primer caso determinará el monto por el cual se lo impone. De haber bienes asegurados para hacerlo efectivo, los deberá identificar.

    Artículo 349.- Pronunciamiento sobre la demanda civil.Tanto en el caso de absolución como en el de condena deberá el tribunal pronunciarse acerca de la demanda civil válidamente interpuesta.
    Artículo 350.- DEROGADO

    Artículo 351.- Reiteración de crímenes o simples delitos de una misma especie. En los casos de reiteración de crímenes o simples delitos de una misma especie se impondrá la pena correspondiente a las diversas infracciones, estimadas como un solo delito, aumentándola en uno o dos grados.
    Si, por la naturaleza de las diversas infracciones, éstas no pudieren estimarse como un solo delito, el tribunal aplicará la pena señalada a aquella que, considerada aisladamente, con las circunstancias del caso, tuviere asignada una pena mayor, aumentándola en uno o dos grados, según fuere el número de los delitos.
    Podrá, con todo, aplicarse las penas en la forma establecida en el artículo 74 del Código Penal si, de seguirse este procedimiento, hubiere de corresponder al condenado una pena menor.
    Para los efectos de este artículo, se considerará delitos de una misma especie aquellos que afectaren al mismo bien jurídico.
    Libro Tercero
    Recursos

    Título I
    Disposiciones generales
    Artículo 352.- Facultad de recurrir. Podrán recurrir en contra de las resoluciones judiciales el ministerio público y los demás intervinientes agraviados por ellas, sólo por los medios y en los casos expresamente establecidos en la ley.
    Artículo 353.- Aumento de los plazos. Si el juicio oral hubiere sido conocido por un tribunal que se hubiese constituido y funcionado en una localidad situada fuera de su lugar de asiento, los plazos legales establecidos para la interposición de los recursos se aumentarán conforme a la tabla de emplazamiento prevista en el artículo 259 del Código de Procedimiento Civil.
    Artículo 354.- Renuncia y desistimiento de los recursos. Los recursos podrán renunciarse expresamente, una vez notificada la resolución contra la cual procedieren.
    Quienes hubieren interpuesto un recurso podrán desistirse de él antes de su resolución. En todo caso, los efectos del desistimiento no se extenderán a los demás recurrentes o a los adherentes al recurso.
    El defensor no podrá renunciar a la interposición de un recurso, ni desistirse de los recursos interpuestos, sin mandato expreso del imputado.
    Artículo 355.- Efecto de la interposición de recursos. La interposición de un recurso no suspenderá la ejecución de la decisión, salvo que se impugnare una sentencia definitiva condenatoria o que la ley dispusiere expresamente lo contrario.
    Artículo 356.- Prohibición de suspender la vista de la causa por falta de integración del tribunal. No podrá suspenderse la vista de un recurso penal por falta de jueces que pudieren integrar la sala. Si fuere necesario, se interrumpirá la vista de recursos civiles para que se integren a la sala jueces no inhabilitados. En consecuencia, la audiencia sólo se suspenderá si no se alcanzare, con los jueces que conformaren ese día el tribunal, el mínimo de miembros no inhabilitados que debieren intervenir en ella.
    Artículo 357.- Suspensión de la vista de la causa por otras causales. La vista de los recursos penales no podrá suspenderse por las causales previstas en los numerales 1, 5, 6 y 7 del artículo 165 del Código de Procedimiento Civil.
    Al confeccionar la tabla o disponer la agregación extraordinaria de recursos o determinar la continuación para el día siguiente de un pleito, la Corte adoptará las medidas necesarias para que la sala que correspondiere no viere alterada su labor.
    Si en la causa hubiere personas privadas de libertad, sólo se suspenderá la vista de la causa por muerte del abogado del recurrente, del cónyuge o del conviviente civil o de alguno de sus ascendientes o descendientes, ocurrida dentro de los ocho días anteriores al designado para la vista del recurso.
    En los demás casos la vista sólo podrá suspenderse si lo solicitare el recurrente o todos los intervinientes facultados para concurrir a ella, de común acuerdo. Este derecho podrá ejercerse una sola vez por el recurrente o por todos los intervinientes, por medio de un escrito que deberá presentarse hasta las doce horas del día hábil anterior a la audiencia correspondiente, a menos que la agregación de la causa se hubiere efectuado con menos de setenta y dos horas antes de la vista, caso en el cual la suspensión podrá solicitarse hasta antes de que comenzare la audiencia.

    Artículo 358.- Reglas generales de vista de los recursos. La vista de la causa se efectuará en una audiencia pública.
    La falta de comparecencia de uno o más recurrentes a la audiencia dará lugar a que se declare el abandono del recurso respecto de los ausentes. La incomparecencia de uno o más de los recurridos permitirá proceder en su ausencia.
    La audiencia se iniciará con el anuncio, tras el cual, sin mediar relación, se otorgará la palabra a el o los recurrentes para que expongan los fundamentos del recurso, así como las peticiones concretas que formularen. Luego se permitirá intervenir a los recurridos y finalmente se volverá a ofrecer la palabra a todas las partes con el fin de que formulen aclaraciones respecto de los hechos o de los argumentos vertidos en el debate.
    En cualquier momento del debate, cualquier miembro del tribunal podrá formular preguntas a los representantes de las partes o pedirles que profundicen su argumentación o la refieran a algún aspecto específico de la cuestión debatida.
    Concluido el debate, el tribunal pronunciará sentencia de inmediato o, si no fuere posible, en un día y hora que dará a conocer a los intervinientes en la misma audiencia. La sentencia será redactada por el miembro del tribunal colegiado que éste designare y el voto disidente o la prevención, por su autor.
    Artículo 359.- Prueba en los recursos. En el recurso de nulidad podrá producirse prueba sobre las circunstancias que constituyeren la causal invocada, siempre que se hubiere ofrecido en el escrito de interposición del recurso.
    Esta prueba se recibirá en la audiencia conforme con las reglas que rigen su recepción en el juicio oral. En caso alguno la circunstancia de que no pudiere rendirse la prueba dará lugar a la suspensión de la audiencia.
    Artículo 360.- Decisiones sobre los recursos. El tribunal que conociere de un recurso sólo podrá pronunciarse sobre las solicitudes formuladas por los recurrentes, quedándole vedado extender el efecto de su decisión a cuestiones no planteadas por ellos o más allá de los límites de lo solicitado, salvo en los casos previstos en este artículo y en el artículo 379 inciso segundo.
    Si sólo uno de varios imputados por el mismo delito entablare el recurso contra la resolución, la decisión favorable que se dictare aprovechará a los demás, a menos que los fundamentos fueren exclusivamente personales del recurrente, debiendo el tribunal declararlo así expresamente.
    Si la resolución judicial hubiere sido objeto de recurso por un solo interviniente, la Corte no podrá reformarla en perjuicio del recurrente.
    Artículo 361.- Aplicación supletoria. Los recursos se regirán por las normas de este Libro. Supletoriamente, serán aplicables las reglas del Título III del Libro Segundo de este Código.

    Título II
    Recurso de reposición
    Artículo 362.- Reposición de las resoluciones dictadas fuera de audiencias. De las sentencias interlocutorias, de los autos y de los decretos dictados fuera de audiencias, podrá pedirse reposición al tribunal que los hubiere pronunciado. El recurso deberá interponerse dentro de tercero día y deberá ser fundado.
    El tribunal se pronunciará de plano, pero podrá oír a los demás intervinientes si se hubiere deducido en un asunto cuya complejidad así lo aconsejare.
    Cuando la reposición se interpusiere respecto de una resolución que también fuere susceptible de apelación y no se dedujere a la vez este recurso para el caso de que la reposición fuere denegada, se entenderá que la parte renuncia a la apelación.
    La reposición no tendrá efecto suspensivo, salvo cuando contra la misma resolución procediere también la apelación en este efecto.
    Artículo 363.- Reposición en las audiencias orales. La reposición de las resoluciones pronunciadas durante audiencias orales deberá promoverse tan pronto se dictaren y sólo serán admisibles cuando no hubieren sido precedidas de debate. La tramitación se efectuará verbalmente, de inmediato, y de la misma manera se pronunciará el fallo.
    Título III
    Recurso de apelación
    Artículo 364.- Resoluciones inapelables. Serán inapelables las resoluciones dictadas por un tribunal de juicio oral en lo penal.
    Artículo 365.- Tribunal ante el que se entabla el recurso de apelación. El recurso de apelación deberá entablarse ante el mismo juez que hubiere dictado la resolución y éste lo concederá o lo denegará.
    Artículo 366.- Plazo para interponer el recurso de apelación. El recurso de apelación deberá entablarse dentro de los cinco días siguientes a la notificación de la resolución impugnada.
    Artículo 367.- Forma de interposición del recurso de apelación. El recurso de apelación deberá ser interpuesto por escrito, con indicación de sus fundamentos y de las peticiones concretas que se formularen.
    Artículo 368.- Efectos del recurso de apelación. La apelación se concederá en el solo efecto devolutivo, a menos que la ley señalare expresamente lo contrario.
    Artículo 369.- Recurso de hecho. Denegado el recurso de apelación, concedido siendo improcedente u otorgado con efectos no ajustados a derecho, los intervinientes podrán ocurrir de hecho, dentro de tercero día, ante el tribunal de alzada, con el fin de que resuelva si hubiere lugar o no al recurso y cuáles debieren ser sus efectos.
    Presentado el recurso, el tribunal de alzada solicitará, cuando correspondiere, los antecedentes señalados en el artículo 371 y luego fallará en cuenta. Si acogiere el recurso por haberse denegado la apelación, retendrá tales antecedentes o los recabará, si no los hubiese pedido, para pronunciarse sobre la apelación.
    Artículo 370.- Resoluciones apelables. Las resoluciones dictadas por el juez de garantía serán apelables en los siguientes casos:
    a) Cuando pusieren término al procedimiento, hicieren imposible su prosecución o la suspendieren por más de treinta días, y
    b) Cuando la ley lo señalare expresamente.
    Artículo 371.- Antecedentes a remitir concedido el recurso de apelación. Concedido el recurso, el juez remitirá al tribunal de alzada copia fiel de la resolución y de todos los antecedentes que fueren pertinentes para un acabado pronunciamiento sobre el recurso.
    Título IV
    Recurso de Nulidad
    Artículo 372.- Del recurso de nulidad. El recurso de nulidad se concede para invalidar el juicio oral total o parcialmente junto con la sentencia definitiva, o sólo esta última, según corresponda, por las causales expresamente señaladas en la ley.
    Deberá interponerse, por escrito, dentro de los diez días siguientes a la notificación de la sentencia definitiva, ante el tribunal que hubiere conocido del juicio oral.

    Artículo 373.- Causales del recurso. Procederá la declaración de nulidad total o sólo la parcial del juicio oral y de la sentencia, si el vicio hubiere generado efectos que son divisibles y subsanables por separado sólo respecto de determinados delitos o recurrentes:
    a) Cuando, en la cualquier etapa del procedimiento
o en el pronunciamiento de la sentencia, se hubieren
infringido sustancialmente derechos o garantías
asegurados por la Constitución o por los tratados
internacionales ratificados por Chile que se encuentren
vigentes, y
    b) Cuando, en el pronunciamiento de la sentencia,
se hubiere hecho una errónea aplicación del derecho que
hubiere influido sustancialmente en lo dispositivo del
fallo.


    Artículo 374.- Motivos absolutos de nulidad. El juicio oral y la sentencia, o parte de éstos, serán siempre anulados:
    a) Cuando la sentencia hubiere sido pronunciada por un tribunal incompetente, o no integrado por los jueces designados por la ley; cuando hubiere sido pronunciada por un juez de garantía o con la concurrencia de un juez de tribunal de juicio oral en lo penal legalmente implicado, o cuya recusación estuviere pendiente o hubiere sido declarada por tribunal competente; y cuando hubiere sido acordada por un menor número de votos o pronunciada por menor número de jueces que el requerido por la ley, o con concurrencia de jueces que no hubieren asistido al juicio;
    b) Cuando la audiencia del juicio oral hubiere tenido lugar en ausencia de alguna de las personas cuya presencia continuada exigen, bajo sanción de nulidad, los artículos 284 y 286;
    c) Cuando al defensor se le hubiere impedido ejercer las facultades que la ley le otorga;
    d) Cuando en el juicio oral hubieren sido violadas las disposiciones establecidas por la ley sobre publicidad y continuidad del juicio;
    e) Cuando, en la sentencia, se hubiere omitido alguno de los requisitos previstos en el artículo 342, letras c), d) o e);
    f) Cuando la sentencia se hubiere dictado con infracción de lo prescrito en el artículo 341, y g) Cuando la sentencia hubiere sido dictada en oposición a otra sentencia criminal pasada en autoridad de cosa juzgada.

    Artículo 375.- Defectos no esenciales. No causan nulidad los errores de la sentencia recurrida que no influyeren en su parte dispositiva, sin perjuicio de lo cual la Corte podrá corregir los que advirtiere durante el conocimiento del recurso.
    Artículo 376.- Tribunal competente para conocer del recurso. El conocimiento del recurso que se fundare en la causal prevista en el artículo 373, letra a), corresponderá a la Corte Suprema.
    La respectiva Corte de Apelaciones conocerá de los recursos que se fundaren en las causales señaladas en el artículo 373, letra b), y en el artículo 374.
    No obstante lo dispuesto en el inciso precedente, cuando el recurso se fundare en la causal prevista en el artículo 373, letra b), y respecto de la materia de derecho objeto del mismo existieren distintas interpretaciones sostenidas en diversos fallos emanados de los tribunales superiores, corresponderá pronunciarse a la Corte Suprema.
    Del mismo modo, si un recurso se fundare en distintas causales y por aplicación de las reglas contempladas en los incisos precedentes correspondiere el conocimiento de al menos una de ellas a la Corte Suprema, ésta se pronunciará sobre todas. Lo mismo sucederá si se dedujeren distintos recursos de nulidad contra la sentencia y entre las causales que los fundaren hubiere una respecto de la cual correspondiere pronunciarse a la Corte Suprema.
    Artículo 377.- Preparación del recurso. Si la infracción invocada como motivo del recurso se refiriere a una ley que regulare el procedimiento, el recurso sólo será admisible cuando quien lo entablare hubiere reclamado oportunamente del vicio o defecto.
    No será necesaria la reclamación del inciso anterior cuando se tratare de alguna de las causales del artículo 374; cuando la ley no admitiere recurso alguno contra la resolución que contuviere el vicio o defecto, cuando éste hubiere tenido lugar en el pronunciamiento mismo de la sentencia que se tratare de anular, ni cuando dicho vicio o defecto hubiere llegado al conocimiento de la parte después de pronunciada la sentencia.
    Artículo 378.- Requisitos del escrito de interposición. En el escrito en que se interpusiere el recurso de nulidad se consignarán los fundamentos del mismo y las peticiones concretas que se sometieren al fallo del tribunal.
    El recurso podrá fundarse en varias causales, caso en el cual se indicará si se invocan conjunta o subsidiariamente. Cada motivo de nulidad deberá ser fundado separadamente.
    Cuando el recurso se fundare en la causal prevista en el artículo 373, letra b), y el recurrente sostuviere que, por aplicación del inciso tercero del artículo 376, su conocimiento correspondiere a la Corte Suprema, deberá, además, indicar en forma precisa los fallos en que se hubiere sostenido las distintas interpretaciones que invocare y acompañar copia de las sentencias o de las publicaciones que se hubieren efectuado del texto íntegro de las mismas.
    Artículo 379.- Efectos de la interposición del recurso. La interposición del recurso de nulidad suspende los efectos de la sentencia condenatoria recurrida. En lo demás, se aplicará lo dispuesto en el artículo 355.
    Interpuesto el recurso, no podrán invocarse nuevas causales. Con todo, la Corte, de oficio, podrá acoger el recurso que se hubiere deducido en favor del imputado por un motivo distinto del invocado por el recurrente, siempre que aquél fuere alguno de los señalados en el artículo 374.
    Artículo 380.- Admisibilidad del recurso en el tribunal a quo. Interpuesto el recurso, el tribunal a quo se pronunciará sobre su admisibilidad.
    La inadmisibilidad sólo podrá fundarse en haberse deducido el recurso en contra de resolución que no fuere impugnable por este medio o en haberse deducido fuera de plazo.
    La resolución que declarare la inadmisibilidad será susceptible de reposición dentro de tercero día.
    Artículo 381.- Antecedentes a remitir concedido el recurso. Concedido el recurso, el tribunal remitirá a la Corte copia de la sentencia definitiva, del registro de la audiencia del juicio oral o de las actuaciones determinadas de ella que se impugnaren, y del escrito en que se hubiere interpuesto el recurso.
    Artículo 382.- Actuaciones previas al conocimiento del recurso. Ingresado el recurso a la Corte, se abrirá un plazo de cinco días para que las demás partes solicitaren que se le declare inadmisible, se adhirieren a él o le formularen observaciones por escrito.
    La adhesión al recurso deberá cumplir con todos los requisitos necesarios para interponerlo y su admisibilidad se resolverá de plano por la Corte.
    Hasta antes de la audiencia en que se conociere el recurso, el acusado podrá solicitar la designación de un defensor penal público con domicilio en la ciudad asiento de la Corte, para que asuma su representación, cuando el juicio oral se hubiere desarrollado en una ciudad distinta.
    Artículo 383.- Admisibilidad del recurso en el tribunal ad quem. Transcurrido el plazo previsto en el artículo anterior, el tribunal ad quem se pronunciará en cuenta acerca de la admisibilidad del recurso.
    Lo declarará inadmisible si concurrieren las razones contempladas en el artículo 380, el escrito de interposición careciere de fundamentos de hecho y de derecho o de peticiones concretas, o el recurso no se hubiere preparado oportunamente.
    Sin embargo, si el recurso se hubiere deducido para ante la Corte Suprema, ella no se pronunciará sobre su admisibilidad, sino que ordenará que sea remitido junto con sus antecedentes a la Corte de Apelaciones respectiva para que, si lo estima admisible, entre a conocerlo y fallarlo, en los siguientes casos:
    a) Si el recurso se fundare en la causal prevista en el artículo 373, letra a), y la Corte Suprema estimare que, de ser efectivos los hechos invocados como fundamento, serían constitutivos de alguna de las causales señaladas en el artículo 374;
    b) Si, respecto del recurso fundado en la causal del artículo 373, letra b), la Corte Suprema estimare que no existen distintas interpretaciones sobre la materia de derecho objeto del mismo o, aun existiendo, no fueren determinantes para la decisión de la causa, y c) Si en alguno de los casos previstos en el inciso final del artículo 376, la Corte Suprema estimare que concurre respecto de los motivos de nulidad invocados alguna de las situaciones previstas en las letras a) y b) de este artículo.
    Artículo 384.- Fallo del recurso. La Corte deberá fallar el recurso dentro de los veinte días siguientes a la fecha en que hubiere terminado de conocer de él.
    En la sentencia, el tribunal deberá exponer los fundamentos que sirvieren de base a su decisión; pronunciarse sobre las cuestiones controvertidas, salvo que acogiere el recurso, en cuyo caso podrá limitarse a la causal o causales que le hubieren sido suficientes, y declarar si es nulo o no total o parcialmente el juicio oral y la sentencia definitiva reclamados, o si solamente es nula dicha sentencia, en los casos que se indican en el artículo siguiente.
    El fallo del recurso se dará a conocer en la audiencia indicada al efecto, con la lectura de su parte resolutiva o de una breve síntesis de la misma.


    Artículo 385.- Nulidad de la sentencia. La Corte podrá invalidar sólo la sentencia y dictar, sin nueva audiencia pero separadamente, la sentencia de reemplazo que se conformare a la ley, si la causal de nulidad no se refiriere a formalidades del juicio ni a los hechos y circunstancias que se hubieren dado por probados, sino se debiere a que el fallo hubiere calificado de delito un hecho que la ley no considerare tal, aplicado una pena cuando no procediere aplicar pena alguna, o impuesto una superior a la que legalmente correspondiere.
    La sentencia de reemplazo reproducirá las consideraciones de hecho, los fundamentos de derecho y las decisiones de la resolución anulada, que no se refieran a los puntos que hubieren sido objeto del recurso o que fueren incompatibles con la resolución recaída en él, tal como se hubieren dado por establecidos en el fallo recurrido.

    Artículo 386.- Nulidad del juicio oral y de la sentencia. Salvo los casos mencionados en el artículo 385, si la Corte acogiere el recurso anulará total o parcialmente la sentencia y el juicio oral, determinará el estado en que hubiere de quedar el procedimiento y ordenará la remisión de los autos al tribunal no inhabilitado que correspondiere, para que éste disponga la realización de un nuevo juicio oral.
    En caso de que se declare la nulidad parcial del juicio oral y la sentencia, existiendo pluralidad de delitos o de imputados, la Corte deberá precisar a qué prueba, a qué hechos y a qué imputados afecta la declaración de nulidad parcial del juicio oral y la sentencia.
    No será obstáculo para que se ordene efectuar un nuevo juicio oral la circunstancia de haberse dado lugar al recurso por un vicio o defecto cometido en el pronunciamiento mismo de la sentencia.

    Artículo 387.- Improcedencia de recursos. La resolución que fallare un recurso de nulidad no será susceptible de recurso alguno, sin perjuicio de la revisión de la sentencia condenatoria firme de que se trata en este Código.
    Tampoco será susceptible de recurso alguno la sentencia que se dictare en el nuevo juicio que se realizare como consecuencia de la resolución que hubiere acogido el recurso de nulidad. No obstante, si la sentencia fuere condenatoria y la que se hubiere anulado hubiese sido absolutoria, procederá el recurso de nulidad en favor del acusado, conforme a las reglas generales.
    Libro Cuarto
    Procedimientos especiales y ejecución

    Título I
    Procedimiento simplificado
    Artículo 388.- Ámbito de aplicación. El conocimiento y fallo de las faltas se sujetará al procedimiento previsto en este Título.
    El procedimiento se aplicará, además, respecto de los hechos constitutivos de simple delito para los cuales el ministerio público requiriere la imposición de una pena que no excediere de presidio o reclusión menores en su grado mínimo.

    Artículo 389.- Normas supletorias. El procedimiento simplificado se regirá por las normas de este Título y, en lo que éste no proveyere, supletoriamente por las del Libro Segundo de este Código, en cuanto se adecuen a su brevedad y simpleza.
    Artículo 390.- Requerimiento. Recibida por el fiscal la denuncia de un hecho constitutivo de alguno de los delitos a que se refiere el artículo 388, solicitará del juez de garantía competente la citación inmediata a audiencia, a menos que fueren insuficientes los antecedentes aportados, se encontrare extinguida la responsabilidad penal del imputado o el fiscal decidiere hacer aplicación de la facultad que le concede el artículo 170. De igual manera, cuando los antecedentes lo ameritaren y hasta la deducción de la acusación, el fiscal podrá dejar sin efecto la formalización de la investigación que ya hubiere realizado de acuerdo con lo previsto en el artículo 230, y proceder conforme a las reglas de este Título.
    Asimismo, si el fiscal formulare acusación y la pena requerida no excediere de presidio o reclusión menores en su grado mínimo, la acusación se tendrá como requerimiento, debiendo el juez disponer la continuación del procedimiento de conformidad a las normas de este Título.
    Tratándose de las faltas indicadas en los artículos 494, Nº 5, y 496, Nº 11, del Código Penal, sólo podrán efectuar el requerimiento precedente las personas a quienes correspondiere la titularidad de la acción conforme a lo dispuesto en los artículos 54 y 55.
    Si la falta contemplada en el artículo 494 bis del Código Penal se cometiere en un establecimiento de comercio, para la determinación del valor de las cosas hurtadas se considerará el precio de venta, salvo que los antecedentes que se reúnan permitan formarse una convicción diferente.

    Artículo 391.- Contenido del requerimiento. El requerimiento deberá contener:
    a) La individualización del imputado;
    b) Una relación sucinta del hecho que se le atribuyere, con indicación del tiempo y lugar de
comisión y demás circunstancias relevantes;
    c) La cita de la disposición legal infringida;
    d) La exposición de los antecedentes o elementos que fundamentaren la imputación;
    e) La pena solicitada por el requirente, y
    f) La individualización y firma del requirente.

    Si el fiscal solicita la aplicación del comiso de ganancias o del comiso por valor equivalente de bienes o instrumentos, deberá indicar su monto aproximado y expresar con claridad y precisión los fundamentos de su solicitud, exponiendo los antecedentes o elementos en los que ella se basa.


    Artículo 392.- Procedimiento monitorio. Se aplicará el procedimiento monitorio a la tramitación de las faltas respecto de las cuales el fiscal pidiere sólo pena de multa. En el requerimiento señalado en el artículo precedente el fiscal indicará el monto de la multa que solicitare imponer.
    Si el juez estimare suficientemente fundado el requerimiento y la proposición relativa a la multa, deberá acogerlos inmediatamente, dictando una resolución que así lo declare. Dicha resolución contendrá, además, las siguientes indicaciones:
    a) La instrucción acerca del derecho del imputado de reclamar en contra del requerimiento y de la imposición de la sanción, dentro de los quince días siguientes a su notificación, así como de los efectos de la interposición del reclamo;
    b) La instrucción acerca de la posibilidad de que dispone el imputado en orden a aceptar el requerimiento y la multa impuesta, así como de los efectos de la aceptación, y
    c) El señalamiento del monto de la multa y de la forma en que la misma debiere enterarse en arcas fiscales, así como del hecho que, si la multa fuere pagada dentro de los quince días siguientes a la notificación al imputado de la resolución prevista en este inciso, ella será rebajada en 25%, expresándose el monto a enterar en dicho caso.
    Si el imputado pagare dicha multa o transcurriere el plazo de quince días desde la notificación de la resolución que la impusiere, sin que el imputado reclamare sobre su procedencia o monto, se entenderá que acepta su imposición. En dicho evento la resolución se tendrá, para todos los efectos legales, como sentencia ejecutoriada.
    Por el contrario, si, dentro del mismo plazo de quince días, el imputado manifestare, de cualquier modo fehaciente, su falta de conformidad con la imposición de la multa o su monto, se proseguirá con el procedimiento en la forma prevista en los artículos siguientes. Lo mismo sucederá si el juez no considerare suficientemente fundado el requerimiento o la multa propuesta por el fiscal.

    Artículo 393.- Citación a audiencia. Recibido el requerimiento, el tribunal ordenará su notificación al imputado y citará a todos los intervinientes a la audiencia a que se refiere el artículo 394, la que no podrá tener lugar antes de veinte ni después de cuarenta días contados desde la fecha de la resolución. El imputado deberá ser citado con, a lo menos, diez días de anticipación a la fecha de la audiencia. La citación del imputado se hará bajo el apercibimiento señalado en el artículo 33 y a la misma se acompañarán copias del requerimiento y de la querella, en su caso.
    En el procedimiento simplificado no procederá la interposición de demandas civiles, salvo aquella que tuviere por objeto la restitución de la cosa o su valor.
    La resolución que dispusiere la citación ordenará que las partes comparezcan a la audiencia, con todos sus medios de prueba. Si alguna de ellas requiriere de la citación de testigos o peritos por medio del tribunal, deberán formular la respectiva solicitud con una anticipación no inferior a cinco días a la fecha de la audiencia.

    Artículo 393 bis. Procedimiento simplificado en caso de falta o simple delito flagrante. Tratándose de una persona sorprendida in fraganti cometiendo una falta o un simple delito de aquéllos a que da lugar este procedimiento, el fiscal podrá disponer que el imputado sea puesto a disposición del juez de garantía, para el efecto de comunicarle en la audiencia de control de la detención, de forma verbal, el requerimiento a que se refiere el artículo 391, y proceder de inmediato conforme a lo dispuesto en este Título.

    Artículo 394.- Primeras actuaciones de la audiencia. Al inicio de la audiencia, el tribunal efectuará una breve relación del requerimiento y de la querella, en su caso. Cuando se encontrare presente la víctima, el juez instruirá a ésta y al imputado sobre la posibilidad de poner término al procedimiento de conformidad a lo previsto en el artículo 241, si ello procediere atendida la naturaleza del hecho punible materia del requerimiento. Asimismo, el fiscal podrá proponer la suspensión condicional del procedimiento, si se cumplieren los requisitos del artículo 237.
    Artículo 395.- Resolución inmediata. Una vez efectuado lo prescrito en el artículo anterior, el tribunal preguntará al imputado si admite responsabilidad en los hechos contenidos en el requerimiento o si, por el contrario, solicitará la realización de la audiencia. Para los efectos de lo dispuesto en el presente inciso, en caso de que del imputado admitiere su responsabilidad, el fiscal podrá modificar la pena requerida y solicitar una pena inferior en un grado al mínimo de los señalados por la ley y en el caso de la multa, podrá solicitar una inferior al mínimo legal.
    Con todo, la regla señalada en el inciso anterior sobre la facultad del fiscal para modificar la pena, sólo será aplicable en la primera audiencia a la que se haya citado al imputado, o en la nueva audiencia a la que se le deba citar, cuando su no comparecencia se encuentre debidamente justificada.
    Si el imputado compareciere a una nueva audiencia, en razón de su inasistencia injustificada a la primera audiencia a la que se haya citado, su admisión de responsabilidad podrá ser considerada por el fiscal como suficiente para estimar que concurre la circunstancia atenuante del artículo 11, Nº 9, del Código Penal, sin perjuicio de las demás reglas que fueren aplicables para la determinación de la pena.
    Si el imputado admitiere su responsabilidad en el hecho, el tribunal dictará sentencia inmediatamente. En estos casos, el juez no podrá imponer una pena superior a la solicitada en el requerimiento, permitiéndose la incorporación de antecedentes que sirvieren para la determinación de la pena.



    Artículo 395 bis. Preparación del juicio simplificado. Si el imputado no admitiere responsabilidad, el juez procederá en la misma audiencia e inmediatamente a la preparación del juicio simplificado, salvo que esta audiencia coincida con la del artículo 132, en cuyo caso la preparación del juicio podrá realizarse a más tardar dentro de quinto día.


    Artículo 396.- Realización del juicio. El juicio simplificado deberá tener lugar en la misma audiencia en que se proceda con su preparación, si ello fuere posible, o a más tardar dentro de trigésimo día.
    El juicio simplificado comenzará dándose lectura al requerimiento del fiscal y a la querella, si la hubiere. En seguida, se oirá a los comparecientes y se recibirá la prueba, tras lo cual se preguntará al imputado si tuviere algo que agregar. Con su nueva declaración o sin ella, el juez pronunciará su decisión de absolución o condena, y fijará una nueva audiencia, para dentro de los cinco días próximos, para dar a conocer el texto escrito de la sentencia. Sin perjuicio de lo anterior, si el vencimiento del plazo para la redacción del fallo coincidiere con un día domingo o festivo, el plazo se diferirá hasta el día siguiente que no sea domingo o festivo.
    La audiencia no podrá suspenderse, ni aun por falta de comparecencia de alguna de las partes o por no haberse rendido prueba en la misma.
    Sin embargo, si no hubiere comparecido algún testigo o perito cuya citación judicial hubiere sido solicitada de conformidad a lo dispuesto en el inciso tercero del artículo 393 y el tribunal considerare su declaración como indispensable para la adecuada resolución de la causa, dispondrá lo necesario para asegurar su comparecencia. La suspensión no podrá en caso alguno exceder de cinco días, transcurridos los cuales deberá proseguirse conforme a las reglas generales, aun a falta del testigo o perito.
    En caso que el imputado requerido, válidamente emplazado, no asista injustificadamente a la audiencia de juicio por segunda ocasión, el tribunal deberá recibir, siempre que considere que ello no vulnera el derecho a defensa del imputado, la prueba testimonial y pericial del Ministerio Público, de la defensa y del querellante, en carácter de prueba anticipada, conforme a lo previsto en el artículo 191 de este Código, sin que sea necesaria su comparecencia posterior al juicio.
    Si se solicita en el requerimiento el comiso de ganancias o el comiso por valor equivalente de bienes o instrumentos por un monto igual o inferior a 400 unidades tributarias mensuales, el juez se pronunciará acerca de su procedencia en la sentencia. Si el monto es superior o si el comiso afecta a terceros, se estará a lo dispuesto en el artículo 348 bis.




    Artículo 397.- Reiteración de faltas. En caso de reiteración de faltas de una misma especie se aplicará, en lo que correspondiere, las reglas contenidas en el artículo 351.
    Artículo 398. Suspensión de la imposición de condena por falta. Cuando resulte mérito para condenar por la falta imputada, pero concurrieren antecedentes favorables que no hicieren aconsejable la imposición de la pena al imputado, el juez podrá dictar la sentencia y disponer en ella la suspensión de la pena y sus efectos por un plazo de seis meses. En tal caso, no procederá acumular esta suspensión con alguna de las penas sustitutivas contempladas en la ley N° 18.216.
    Transcurrido el plazo previsto en el inciso anterior sin que el imputado hubiere sido objeto de nuevo requerimiento o de una formalización de la investigación, el tribunal dejará sin efecto la sentencia y, en su reemplazo, decretará el sobreseimiento definitivo de la causa.
    Esta suspensión no afecta la responsabilidad civil derivada del delito.


    Artículo 399.- Recursos. Contra la sentencia definitiva sólo podrá interponerse el recurso de nulidad previsto en el Título IV del Libro Tercero. El fiscal requirente y el querellante, en su caso, sólo podrán recurrir si hubieren concurrido al juicio.
    Título II
    Procedimiento por delito de acción privada
    Artículo 400.- Inicio del procedimiento. El procedimiento comenzará sólo con la interposición de la querella por la persona habilitada para promover la acción penal, ante el juez de garantía competente. Este escrito deberá cumplir con los requisitos de los artículos 113 y 261, en lo que no fuere contrario a lo dispuesto en este Título.
    El querellante deberá acompañar una copia de la querella por cada querellado a quien la misma debiere ser notificada.
    En la misma querella se podrá solicitar al juez la realización de determinadas diligencias destinadas a precisar los hechos que configuran el delito de acción privada. Ejecutadas las diligencias, el tribunal citará a las partes a la audiencia a que se refiere el artículo 403.
    Artículo 401.- Desistimiento de la querella. Si el querellante se desistiere de la querella se decretará sobreseimiento definitivo en la causa y el querellante será condenado al pago de las costas, salvo que el desistimiento obedeciere a un acuerdo con el querellado.
    Con todo, una vez iniciado el juicio no se dará lugar al desistimiento de la acción privada, si el querellado se opusiere a él.
    Artículo 402- Abandono de la acción. La inasistencia del querellante a la audiencia del juicio, así como su inactividad en el procedimiento por más de treinta días, entendiendo por tal la falta de realización de diligencias útiles para dar curso al proceso que fueren de cargo del querellante, producirán el abandono de la acción privada. En tal caso el tribunal deberá, de oficio o a petición de parte, decretar el sobreseimiento definitivo de la causa.
    Lo mismo se observará si, habiendo muerto o caído en incapacidad el querellante, sus herederos o representante legal no concurrieren a sostener la acción dentro del término de noventa días.
    Artículo 403.- Comparecencia de las partes a la audiencia en los delitos de acción privada. El querellante y querellado podrán comparecer a la audiencia en forma personal o representados por mandatario con facultades suficientes para transigir. Sin perjuicio de ello, deberán concurrir en forma personal, cuando el tribunal así lo ordenare.
    Artículo 404.- Conciliación. Al inicio de la audiencia, el juez instará a las partes a buscar un acuerdo que ponga término a la causa. Tratándose de los delitos de calumnia o de injuria, otorgará al querellado la posibilidad de dar explicaciones satisfactorias de su conducta.
    Artículo 405.- Normas supletorias. En lo que no proveyere este título, el procedimiento por delito de acción privada se regirá por las normas del Título I del Libro Cuarto, con excepción del artículo 398.
    Título III
    Procedimiento abreviado
    Artículo 406.- Presupuestos del procedimiento abreviado. Se aplicará el procedimiento abreviado para conocer y fallar, los hechos respecto de los cuales el fiscal requiriere la imposición de una pena privativa de libertad no superior a cinco años de presidio o reclusión menores en su grado máximo ; no superior a diez años de presidio o reclusión mayores en su grado mínimo, tratándose de los ilícitos comprendidos en los párrafos 1 a 4 bis del título IX del Libro Segundo del Código Penal y en el artículo 456 bis A del mismo Código, con excepción de las figuras sancionadas en los artículos 448, inciso primero, y 448 quinquies de ese cuerpo legal, o bien cualesquiera otras penas de distinta naturaleza, cualquiera fuere su entidad o monto, ya fueren ellas únicas, conjuntas o alternativas.
    También se aplicará cuando el fiscal requiriere la imposición de una pena privativa de libertad no superior a diez años de presidio o reclusión mayores en su grado mínimo, tratándose de los ilícitos previstos en la ley N° 17.798, sobre control de armas.
    Para ello, será necesario que el imputado, en conocimiento de los hechos materia de la acusación y de los antecedentes de la investigación que la fundaren, los acepte expresamente y manifieste su conformidad con la aplicación de este procedimiento.
    La existencia de varios acusados o la atribución de varios delitos a un mismo acusado no impedirá la aplicación de las reglas del procedimiento abreviado a aquellos acusados o delitos respecto de los cuales concurrieren los presupuestos señalados en este artículo.




    Artículo 407. Oportunidad para solicitar el procedimiento abreviado. Una vez formalizada la investigación, la tramitación de la causa conforme a las reglas del procedimiento abreviado podrá ser acordada en cualquier etapa del procedimiento, hasta la audiencia de preparación del juicio oral.

    Sin perjuicio de lo señalado en el inciso precedente, podrá solicitarse el procedimiento abreviado, aun cuando hubiere finalizado la audiencia de preparación del juicio oral y hasta antes del envío del auto de apertura al tribunal de juicio oral en lo penal. La solicitud se resolverá de conformidad a lo establecido en el artículo 280 bis.

    Si no se hubiere deducido aún acusación, el fiscal y el querellante, en su caso, las formularán verbalmente en la audiencia que el tribunal convocare para resolver la solicitud de procedimiento abreviado, a la que deberá citar a todos los intervinientes. Deducidas verbalmente las acusaciones, se procederá en lo demás en conformidad a las reglas de este Título.

      Si se hubiere deducido acusación, el fiscal y el acusador particular podrán modificarla según las reglas generales, así como la pena requerida, con el fin de permitir la tramitación del caso conforme a las reglas de este Título. Para estos efectos, la aceptación de los hechos a que se refiere el inciso segundo del artículo 406 podrá ser considerada por el fiscal como suficiente para estimar que concurre la circunstancia atenuante del artículo 11, Nº 9, del Código Penal, sin perjuicio de las demás reglas que fueren aplicables para la determinación de la pena.

    Sin perjuicio de lo establecido en los incisos anteriores, respecto de los delitos señalados en el artículo 449 del Código Penal, si el imputado acepta expresamente los hechos y los antecedentes de la investigación en que se fundare un procedimiento abreviado, el fiscal o el querellante, según sea el caso, podrá solicitar una pena inferior en un grado al mínimo de los señalados por la ley, debiendo considerar previamente lo establecido en las reglas 1a o 2a de ese artículo.

    Si el procedimiento abreviado no fuere admitido por el juez de garantía, se tendrán por no formuladas las acusaciones verbales realizadas por el fiscal y el querellante, lo mismo que las modificaciones que, en su caso, éstos hubieren realizado a sus respectivos libelos, y se continuará de acuerdo a las disposiciones del Libro Segundo de este Código.



    Artículo 408.- Oposición del querellante al procedimiento abreviado. El querellante sólo podrá oponerse al procedimiento abreviado cuando en su acusación particular hubiere efectuado una calificación jurídica de los hechos, atribuido una forma de participación o señalado circunstancias modificatorias de la responsabilidad penal diferentes de las consignadas por el fiscal en su acusación y, como consecuencia de ello, la pena solicitada excediere el límite señalado en el artículo 406.
    Artículo 409.- Intervención previa del juez de garantía. Antes de resolver la solicitud del fiscal, el juez de garantía consultará al acusado a fin de asegurarse que éste ha prestado su conformidad al procedimiento abreviado en forma libre y voluntaria, que conociere su derecho a exigir un juicio oral, que entendiere los términos del acuerdo y las consecuencias que éste pudiere significarle y, especialmente, que no hubiere sido objeto de coacciones ni presiones indebidas por parte del fiscal o de terceros.
    Artículo 410.- Resolución sobre la solicitud de procedimiento abreviado. El juez aceptará la solicitud del fiscal y del imputado cuando los antecedentes de la investigación fueren suficientes para proceder de conformidad a las normas de este Título, la pena solicitada por el fiscal se conformare a lo previsto en el inciso primero del artículo 406 y verificare que el acuerdo hubiere sido prestado por el acusado con conocimiento de sus derechos, libre y voluntariamente.
    Cuando no lo estimare así, o cuando considerare fundada la oposición del querellante, rechazará la solicitud de procedimiento abreviado y dictará el auto de apertura del juicio oral. En este caso, se tendrán por no formuladas la aceptación de los hechos por parte del acusado y la aceptación de los antecedentes a que se refiere el inciso segundo del artículo 406, como tampoco las modificaciones de la acusación o de la acusación particular efectuadas para posibilitar la tramitación abreviada del procedimiento. Asimismo, el juez dispondrá que todos los antecedentes relativos al planteamiento, discusión y resolución de la solicitud de proceder de conformidad al procedimiento abreviado sean eliminadas del registro.
    Artículo 411.- Trámite en el procedimiento abreviado. Acordado el procedimiento abreviado, el juez abrirá el debate, otorgará la palabra al fiscal, quien efectuará una exposición resumida de la acusación y de las actuaciones y diligencias de la investigación que la fundamentaren. A continuación, se dará la palabra a los demás intervinientes. En todo caso, la exposición final corresponderá siempre al acusado.
    Si el fiscal solicita la aplicación del comiso de ganancias o del comiso por valor equivalente de bienes e instrumentos, deberá indicar su monto aproximado y expresar con claridad y precisión los fundamentos de su solicitud.

    Artículo 411 bis.- Sanciones al fiscal que no asistiere o abandonare la audiencia injustificadamente. A la inasistencia o abandono injustificado del fiscal a la audiencia del procedimiento abreviado o a alguna de sus sesiones, si se desarrollare en varias, se aplicará lo previsto en el inciso segundo del artículo 269.


    Artículo 412.- Fallo en el procedimiento abreviado. Terminado el debate, el juez dictará sentencia. En caso de ser condenatoria, no podrá imponer una pena superior ni más desfavorable a la requerida por el fiscal o el querellante, en su caso.
    La sentencia condenatoria no podrá emitirse exclusivamente sobre la base de la aceptación de los hechos por parte del imputado.
    En ningún caso el procedimiento abreviado obstará a la concesión de alguna de las penas sustitutivas consideradas en la ley, cuando correspondiere.
    La sentencia no se pronunciará sobre la demanda civil que hubiere sido interpuesta.


    Artículo 413.- Contenido de la sentencia en el procedimiento abreviado. La sentencia dictada en el procedimiento abreviado contendrá:
    a) La mención del tribunal, la fecha de su dictación y la identificación de los intervinientes;
    b) La enunciación breve de los hechos y circunstancias que hubieren sido objeto de la acusación y de la aceptación por el acusado, así como de la defensa de éste;
    c) La exposición clara, lógica y completa de cada uno de los hechos que se dieren por probados sobre la base de la aceptación que el acusado hubiere manifestado respecto a los antecedentes de la investigación, así como el mérito de éstos, valorados en la forma prevista en el artículo 297;
    d) Las razones legales o doctrinales que sirvieren para calificar jurídicamente cada uno de los hechos y sus circunstancias y para fundar su fallo;
    e) La resolución que condenare o absolviere al acusado. La sentencia condenatoria fijará las penas y se pronunciará sobre la aplicación de alguna de las penas sustitutivas a la privación o restricción de libertad previstas en la ley;
    f) El pronunciamiento sobre las costas, y g) La firma del juez que la hubiere dictado.
    La sentencia que condenare a una pena temporal deberá expresar con toda precisión el día desde el cual empezará ésta a contarse y fijará el tiempo de detención o prisión preventiva que deberá servir de abono para su cumplimiento.
    La sentencia condenatoria dispondrá también el comiso de los instrumentos o efectos del delito o su restitución, cuando fuere procedente.
    Si el fiscal solicita el comiso de ganancias o el comiso por valor equivalente de efectos o instrumentos del delito por un monto igual o inferior a 400 unidades tributarias mensuales, el juez se pronunciará acerca de su procedencia en la sentencia. Si el monto es superior o si el comiso afecta a terceros, se estará a lo dispuesto en el artículo 348 bis.



    Artículo 414.- Recursos en contra de la sentencia dictada en el procedimiento abreviado. La sentencia definitiva dictada por el juez de garantía en el procedimiento abreviado sólo será impugnable por apelación, que se deberá conceder en ambos efectos.
    En el conocimiento del recurso de apelación la Corte podrá pronunciarse acerca de la concurrencia de los supuestos del procedimiento abreviado previstos en el artículo 406.
    Artículo 415.- Normas aplicables en el procedimiento abreviado. Se aplicarán al procedimiento abreviado las disposiciones consignadas en este Título, y en lo no previsto en él, las normas comunes previstas en este Código y las disposiciones del procedimiento ordinario.

    Título III bis
    Procedimiento relativo a la imposición de comiso sin condena previa


    Artículo 415 bis.- Ámbito de aplicación. Las reglas de este Título son aplicables en los casos en que la ley dispone el comiso de bienes o activos obtenidos a través de la comisión del hecho ilícito o utilizados en su perpetración sin sujetar su procedencia a la dictación de una sentencia condenatoria relativa al hecho.
    Es competente para conocer del procedimiento relativo al comiso sin condena el tribunal que haya dictado la resolución que ponga término a la investigación o juicio respectivo.

    Artículo 415 ter.- Inicio del procedimiento. Habiéndose incautado bienes o habiéndolos asegurado conforme al artículo 157, el Ministerio Público o el querellante solicitará mediante requerimiento escrito presentado ante el tribunal que se cite a audiencia especial para hacer efectivo el comiso. La solicitud deberá ser presentada en un plazo no superior a diez días contado desde que quede ejecutoriada la última resolución que recaiga sobre la respectiva investigación o juicio, poniéndole término temporal o definitivo.
    Transcurrido este plazo sin que se haya deducido el requerimiento, el tribunal abrirá un plazo máximo de cinco días para que el fiscal deduzca el requerimiento o comunique fundadamente su decisión de no hacerlo, y dará cuenta de inmediato de ello al Fiscal Regional. De no deducirse requerimiento dentro de este plazo, de oficio el tribunal dejará sin efecto la incautación y las medidas cautelares que se hayan dispuesto.

    Artículo 415 quáter.- Contenido del requerimiento. El requerimiento deberá contener:

    a) La individualización de todas las personas que conforme a la ley podrían ser afectadas en su propiedad o patrimonio por la imposición del comiso, cuando los hubiere.

    b) Una relación sucinta del hecho que se le atribuyó, y las razones manifestadas en la resolución que puso término al procedimiento, de su término sin condena.

    c) La exposición de los antecedentes o elementos que fundan la solicitud.

    d) La exposición del monto y de los bienes muebles e inmuebles cuyo comiso se solicita.

    e) La individualización y firma del requirente.

    Artículo 415 quinquies.- Citación a audiencia. En la resolución que provee el requerimiento se citará a audiencia especial de comiso, la que no podrá tener lugar antes de treinta ni después de sesenta días.
    En la citación el juez ordenará que las partes comparezcan a la audiencia con todos sus medios de prueba. Si alguna de las partes requiere de la citación de testigos o peritos por medio del tribunal, deberá formular la respectiva solicitud, al menos, diez días antes de la fecha de la audiencia.
    El requerimiento y la resolución que recaiga sobre éste serán notificados a todas las personas señaladas en la letra a) del artículo precedente y, en su caso, a los demás intervinientes en la respectiva investigación o juicio, con a lo menos quince días de anticipación a la fecha de la audiencia.

    Artículo 415 sexies.- Desarrollo de la audiencia. La audiencia comenzará con la lectura del requerimiento de aplicación del comiso formulada por el Ministerio Público o el querellante y la presentación de los antecedentes que serán ofrecidos por las demás partes. En caso de que alguna de las partes lo solicite, el tribunal podrá disponer la realización de una audiencia de preparación. De lo contrario, la audiencia seguirá su curso procediéndose a recibir la prueba ofrecida.
    En aquello que no sea incompatible con la naturaleza de este procedimiento, la audiencia se regirá por las normas del juicio simplificado.
    La prueba de los hechos de los que depende la procedencia del comiso, incluido su monto, será producida conforme a lo dispuesto en el artículo 295 y apreciada conforme a lo dispuesto en el artículo 297. El tribunal formará su convicción sobre la base de la prueba preponderante producida durante la audiencia.
    En caso de no existir oposición, el juez podrá fallar con el sólo mérito del contenido del requerimiento de comiso presentado y debidamente notificado.

    Artículo 415 septies.- Contenido de la sentencia. La sentencia en el procedimiento de comiso sin condena previa contendrá:

    a) La mención del tribunal, la fecha de su dictación y la identificación de los intervinientes y la certificación de haberse cursado las notificaciones a las que se refiere el artículo 415 quinquies, inciso tercero.

    b) La enunciación de la solicitud del Ministerio Público, de la querellante y de las defensas de los afectados, si los hubiere, y sus fundamentos respectivos.

    c) El análisis breve de la prueba producida.

    d) Las razones de hecho y de derecho que sirven de fundamento a la sentencia, en particular las que se refieren a la existencia del hecho ilícito del que proceden las ganancias o su conexión con los instrumentos o efectos de que se trate.

    e) La decisión del asunto, imponiendo el comiso o denegándolo, y en el primer caso determinando el monto por el cual se lo impone.

    Artículo 415 octies.- Recursos. Contra la sentencia definitiva podrá interponerse el recurso de nulidad previsto en el Título IV del Libro III, en cuanto se pretenda la impugnación de la imposición o denegación del comiso. Si lo impugnado fuere el monto procederá el recurso de apelación, el cual podrá en su caso interponerse en subsidio del recurso de nulidad.
    El fiscal requirente y el querellante, en su caso, sólo podrán recurrir si concurrieron al juicio.
    El tribunal que conozca del recurso podrá decretar la nulidad de la audiencia prevista en el artículo 415 sexies o, de tratarse exclusivamente de un error de derecho, anulará la sentencia y dictará sentencia de reemplazo.

    Artículo 415 nonies.- Ejecución. Una vez ejecutoriada la sentencia que impone el comiso, ella será ejecutada conforme a lo dispuesto en el artículo 468 bis.
    Título IV
    Procedimiento relativo a personas que gozan de fuero constitucional

    Párrafo 1º Personas que tienen el fuero del artículo 58 de la Constitución Política



    Artículo 416.- Solicitud de desafuero. Una vez cerrada la investigación, si el fiscal estimare que procediere formular acusación por crimen o simple delito en contra de una persona que tenga el fuero a que se refieren los incisos segundo a cuarto del artículo 58 de la Constitución Política, remitirá los antecedentes a la Corte de Apelaciones correspondiente, a fin de que, si hallare mérito, declare que ha lugar a formación de causa.
    Igual declaración requerirá si, durante la investigación, el fiscal quisiere solicitar al juez de garantía la prisión preventiva del aforado u otra medida cautelar en su contra.
    Si se tratare de un delito de acción privada, el querellante deberá ocurrir ante la Corte de Apelaciones solicitando igual declaración, antes de que se admitiere a tramitación su querella por el juez de garantía.

    Artículo 417.- Detención in fraganti. Si el aforado fuere detenido por habérsele sorprendido en delito flagrante, el fiscal lo pondrá inmediatamente a disposición de la Corte de Apelaciones respectiva. Asimismo, remitirá la copia del registro de las diligencias que se hubieren practicado y que fueren conducentes para resolver el asunto.
    Artículo 418.- Apelación. La resolución que se pronunciare sobre la petición de desafuero será apelable para ante la Corte Suprema.

    Artículo 419.- Comunicación en caso de desafuero de diputado o senador. Si la persona desaforada fuere un diputado o un senador, una vez que se hallare firme la resolución que declarare haber lugar a formación de causa, será comunicada por la Corte de Apelaciones respectiva a la rama del Congreso Nacional a que perteneciere el imputado. Desde la fecha de esa comunicación, el diputado o senador quedará suspendido de su cargo.
    Artículo 420.- Efectos de la resolución que diere lugar a formación de causa. Si se diere lugar a formación de causa, se seguirá el procedimiento conforme a las reglas generales.
    Sin embargo, en el caso a que se refiere el inciso primero del artículo 416, el juez de garantía fijará de inmediato la fecha de la audiencia de preparación del juicio oral, la que deberá efectuarse dentro de los quince días siguientes a la recepción de los antecedentes por el juzgado de garantía. A su vez, la audiencia del juicio oral deberá iniciarse dentro del plazo de quince días contado desde la notificación del auto de apertura del juicio oral. Con todo, se aplicarán los plazos previstos en las reglas generales cuando el imputado lo solicitare para preparar su defensa.
    Artículo 421.- Efectos de la resolución que no diere lugar a formación de causa. Si, en el caso del inciso primero del artículo 416, la Corte de Apelaciones declarare no haber lugar a formación de causa, esta resolución producirá los efectos del sobreseimiento definitivo respecto del aforado favorecido con aquella declaración.
    Tratándose de la situación contemplada en el inciso tercero del mismo artículo, el juez de garantía no admitirá a tramitación la querella y archivará los antecedentes.

    Artículo 422.- Pluralidad de sujetos. Si aparecieren implicados individuos que no gozaren de fuero, se seguirá adelante el procedimiento en relación con ellos.
    Párrafo 2º Delegados Presidenciales Regionales, Delegados Presidenciales Provinciales y Gobernadores Regionales



    Artículo 423.- Remisión a normas del Párrafo 1º. El procedimiento establecido en el Párrafo 1º de este Título es aplicable a los casos de desafuero de gobernadores regionales, delegados presidenciales regionales o delegados presidenciales provinciales, en lo que fuere pertinente.




NOTA
      El N° 2 del Art. 13 de la Ley 21073, publicada el 22.02.2018, dispuso la sustitución en el presente artículo de la expresión "de un intendente, de un gobernador o de un presidente de consejo regional" por "de un delegado presidencial regional, de un delegado presidencial provincial o de un gobernador regional", sin embargo la frase a sustituir no existe en este texto por cuanto fue reemplazada por la Ley 21074 publicada el 15.02.2018, por lo que no se pudo efectuar la modificación.
    Título V
    Querella de capítulos
    Artículo 424.- Objeto de la querella de capítulos. La querella de capítulos tiene por objeto hacer efectiva la responsabilidad criminal de los jueces, fiscales judiciales y fiscales del ministerio público por actos que hubieren ejecutado en el ejercicio de sus funciones e importaren una infracción penada por la ley.
    Artículo 425.- Solicitud de admisibilidad de los capítulos de acusación. Una vez cerrada la investigación, si el fiscal estimare que procede formular acusación por crimen o simple delito contra un juez, un fiscal judicial o un fiscal del ministerio público, remitirá los antecedentes a la Corte de Apelaciones correspondiente, a fin de que, si hallare mérito, declare admisibles los capítulos de acusación.
    En el escrito de querella se especificarán los capítulos de acusación, y se indicarán los hechos que constituyeren la infracción de la ley penal cometida por el funcionario capitulado.
    Igual declaración a la prevista en el inciso primero requerirá el fiscal si, durante la investigación, quisiere solicitar al juez de garantía la prisión preventiva de algunas de esas personas u otra medida cautelar en su contra.
    Si se tratare de un delito de acción privada, el querellante deberá ocurrir ante la Corte de Apelaciones solicitando igual declaración, antes de que se admitiere a tramitación por el juez de garantía la querella que hubiere presentado por el delito.
    Artículo 426.- Juez, fiscal judicial o fiscal detenido in fraganti. Si un juez, un fiscal judicial o un fiscal del ministerio público fuere detenido por habérsele sorprendido en delito flagrante, el fiscal lo pondrá inmediatamente a disposición de la Corte de Apelaciones respectiva. Asimismo, remitirá la copia del registro de las diligencias que se hubieren practicado y que fueren conducentes para resolver el asunto.
    Artículo 427.- Apelación. La resolución que se pronunciare sobre la querella de capítulos será apelable para ante la Corte Suprema.
    Artículo 428.- Efectos de la sentencia que declara admisible la querella de capítulos. Cuando por sentencia firme se hubieren declarado admisibles todos o alguno de los capítulos de acusación, el funcionario capitulado quedará suspendido del ejercicio de sus funciones y el procedimiento penal continuará de acuerdo a las reglas generales.
    Sin embargo, en el caso a que se refiere el inciso primero del artículo 425, el juez de garantía fijará de inmediato la fecha de la audiencia de preparación del juicio oral la que deberá verificarse dentro de los quince días siguientes a la recepción de los antecedentes por el juzgado de garantía. A su vez, la audiencia del juicio oral deberá iniciarse dentro del plazo de quince días contado desde la notificación del auto de apertura del juicio oral. Con todo, se aplicarán los plazos previstos en las reglas generales cuando el imputado lo solicitare para preparar su defensa.
    Artículo 429.- Efectos de la sentencia que declara inadmisible la querella de capítulos. Si, en el caso del inciso primero del artículo 425, la Corte de Apelaciones declarare inadmisibles todos los capítulos de acusación comprendidos en la querella, tal resolución producirá los efectos del sobreseimiento definitivo respecto del juez, fiscal judicial o fiscal del ministerio público favorecido con aquella declaración.
    Tratándose de la situación contemplada en el inciso final del mismo artículo, el juez de garantía no admitirá a tramitación la querella que ante él se hubiere presentado y archivará los antecedentes.
    Artículo 430.- Pluralidad de sujetos. Si en el mismo procedimiento aparecieren implicados otros individuos que no fueren jueces, fiscales judiciales o fiscales del ministerio público, se seguirá adelante en relación con ellos.
    Título VI
    Extradición

    Párrafo 1º Extradición activa




    Artículo 431.- Procedencia de la extradición activa. Cuando en la tramitación de un procedimiento penal se hubiere formalizado la investigación por un delito que tuviere señalada en la ley una pena privativa de libertad cuya duración mínima excediere de un año, respecto de un individuo que se encontrare en país extranjero, el ministerio público deberá solicitar del juez de garantía que eleve los antecedentes a la Corte de Apelaciones, a fin de que este tribunal, si estimare procedente la extradición del imputado al país en el que actualmente se encontrare, ordene sea pedida. Igual solicitud podrá hacer el querellante, si no la formulare el ministerio público.
    El mismo procedimiento se empleará en los casos enumerados en el artículo 6º del Código Orgánico de Tribunales.
    La extradición procederá, asimismo, con el objeto de hacer cumplir en el país una sentencia definitiva condenatoria a una pena privativa de libertad de cumplimiento efectivo superior a un año.
    Artículo 432.- Tramitación ante el juez de garantía. Se podrá formalizar la investigación respecto del imputado ausente, el que será representado en la audiencia respectiva por un defensor penal público, si no contare con defensor particular.
    Al término de la audiencia, previo debate, el juez de garantía accederá a la solicitud de extradición si estimare que en la especie concurren los requisitos del artículo 140.
    Si el juez de garantía diere lugar a la solicitud de extradición a petición del fiscal o del querellante, declarará la procedencia de pedir, en el país extranjero, la prisión preventiva u otra medida cautelar personal respecto del imputado, en caso de que se cumplan las condiciones que permitirían decretar en Chile la medida respectiva.
    Para que el juez eleve los antecedentes a la Corte de Apelaciones, será necesario que conste en el procedimiento el país y lugar en que el imputado se encontrare en la actualidad.
    Artículo 433.- Audiencia ante la Corte de Apelaciones. Recibidos los antecedentes por la Corte de Apelaciones, ésta fijará una audiencia para fecha próxima, a la cual citará al ministerio público, al querellante, si éste hubiere solicitado la extradición y al defensor del imputado. La audiencia, que tendrá lugar con los litigantes que asistieren y que no se podrá suspender a petición de éstos, se iniciará con una relación pública de los antecedentes que motivaren la solicitud; luego, se concederá la palabra al fiscal, en su caso al querellante y al defensor.
    Artículo 434.- Solicitud de detención previa u otra medida cautelar personal. Durante la tramitación de la extradición, a petición del fiscal o del querellante que la hubiere requerido, la Corte de Apelaciones podrá solicitar del Ministerio de Relaciones Exteriores que se pida al país en que se encontrare el imputado que ordene la detención previa de éste o adopte otra medida destinada a evitar la fuga de la persona cuya extradición se solicitará, cuando el juez de garantía hubiere comprobado la concurrencia de los requisitos que admitirían decretar la prisión preventiva u otra medida cautelar personal.
    La solicitud de la Corte de Apelaciones deberá consignar los antecedentes que exigiere el tratado aplicable para solicitar la detención previa o, a falta de tratado, al menos los antecedentes contemplados en el artículo 442.
    Artículo 435.- Fallo de la solicitud de extradición activa. Finalizada la audiencia, la Corte de Apelaciones resolverá en un auto fundado si debiere o no solicitarse la extradición del imputado.
    En contra de la resolución de la Corte de Apelaciones que se pronunciare sobre la solicitud de extradición, no procederá recurso alguno.
    Artículo 436.- Fallo que acoge la solicitud de extradición activa. En caso de acoger la solicitud de extradición, la Corte de Apelaciones se dirigirá al Ministerio de Relaciones Exteriores, al que hará llegar copia de la resolución de que se trata en el artículo anterior, pidiendo que se practiquen las gestiones diplomáticas que fueren necesarias para obtener la extradición.
    Acompañará, además, copia de la formalización de la investigación que se hubiere formulado en contra del imputado; de los antecedentes que la hubieren motivado o de la resolución firme que hubiere recaído en el procedimiento, si se tratare de un condenado; de los textos legales que tipificaren y sancionaren el delito, de los referentes a la prescripción de la acción y de la pena, y toda la información conocida sobre la filiación, identidad, nacionalidad y residencia del imputado.
    Cumplidos estos trámites, la Corte de Apelaciones devolverá los antecedentes al tribunal de origen.
    Artículo 437.- Tramitación del fallo que acoge la solicitud de extradición activa. El Ministerio de Relaciones Exteriores legalizará y traducirá los documentos acompañados, si fuere del caso, y hará las gestiones necesarias para dar cumplimiento a la resolución de la Corte de Apelaciones. Si se obtuviere la extradición del imputado, lo hará conducir del país en que se encontrare, hasta ponerlo a disposición de aquel tribunal.
    En este último caso, la Corte de Apelaciones ordenará que el imputado sea puesto a disposición del tribunal competente, a fin de que el procedimiento siga su curso o de que cumpla su condena, si se hubiere pronunciado sentencia firme.

    Artículo 438.- Extradición activa improcedente o no concedida. Si la Corte de Apelaciones declarare no ser procedente la extradición se devolverán los antecedentes al tribunal, a fin de que proceda según corresponda.
    Si la extradición no fuere concedida por las autoridades del país en que el imputado se encontrare, se comunicará el hecho al tribunal de garantía, para idéntico fin.
    Artículo 439.- Multiplicidad de imputados en un mismo procedimiento. Si el procedimiento comprendiere a un imputado que se encontrare en el extranjero y a otros imputados presentes, se observarán las disposiciones anteriores en cuanto al primero y, sin perjuicio de su cumplimiento, se proseguirá sin interrupción en contra de los segundos.
    Párrafo 2º Extradición pasiva
    Artículo 440.- Procedencia de la extradición pasiva. Cuando un país extranjero solicitare a Chile la extradición de individuos que se encontraren en el territorio nacional y que en el país requirente estuvieren imputados de un delito o condenados a una pena privativa de libertad de duración superior a un año, el Ministerio de Relaciones Exteriores remitirá la petición y sus antecedentes a la Corte Suprema.
    Artículo 441.- Tribunal de primera instancia en la extradición pasiva. Recibidos los antecedentes, se designará al ministro de la Corte Suprema que conocerá en primera instancia de la solicitud de extradición, quien fijará, desde luego, día y hora para la realización de la audiencia a que se refiere el artículo 448 y pondrá la petición y sus antecedentes en conocimiento del representante del Estado requirente y del imputado, a menos que se hubieren solicitado medidas cautelares personales en contra de este último. Si se hubieren pedido tales medidas, el conocimiento de la petición y los antecedentes se suministrará al imputado una vez que las mismas se hubieren decretado.
    Artículo 442.- Detención previa. Antes de recibirse la solicitud formal de extradición, el Ministro de la Corte Suprema podrá decretar la detención del imputado, si así se hubiere estipulado en el tratado respectivo o lo requiriere el Estado extranjero mediante una solicitud que contemple las siguientes menciones mínimas:
    a) La identificación del imputado;
    b) La existencia de una sentencia condenatoria firme o de una orden restrictiva o privativa de la libertad personal del imputado;
    c) La calificación del delito que motivare la solicitud, el lugar y la fecha de comisión de aquél, y d) La declaración de que se solicitará formalmente la extradición.
    La detención previa se decretará por el plazo que determinare el tratado aplicable o, en su defecto, por un máximo de dos meses a contar de la fecha en que el Estado requirente fuere notificado del hecho de haberse producido la detención previa del imputado.
    Artículo 443.- Representación del Estado requirente. El ministerio público representará el interés del Estado requirente en el procedimiento de extradición pasiva, lo que no obstará al cumplimiento de lo dispuesto en su ley orgánica constitucional.
    En cualquier momento, antes de la audiencia a que se refiere el artículo 448, el Estado requirente podrá designar otro representante, caso en el cual cesará la intervención del ministerio público.
    Artículo 444.- Ofrecimiento y producción de pruebas. Si el Estado requirente y el imputado quisieren rendir prueba testimonial, pericial o documental, la deberán ofrecer con a lo menos tres días de anticipación a la audiencia, individualizando a los testigos, si los hubiere, en la solicitud que presentaren. Esta prueba se producirá en la audiencia a que se refiere el artículo 448.
    Artículo 445.- Declaración del imputado. En la audiencia prevista en el artículo 448, el imputado tendrá derecho siempre a prestar declaración, ocasión en la que podrá ser libre y directamente interrogado por el representante del Estado requirente y por su defensor.
    Artículo 446.- Procedencia de la prisión preventiva y de otras medidas cautelares personales. Presentada la solicitud de extradición, el Estado requirente podrá solicitar la prisión preventiva del individuo cuya extradición se requiriere, u otras medidas cautelares personales, que se decretarán si se cumplieren los requisitos que disponga el tratado respectivo o, en su defecto, los previstos en el Título V del Libro I.
    Artículo 447. De la modificación, revocación o sustitución de las medidas cautelares personales. En cualquier estado del procedimiento se podrán modificar, revocar o sustituir las medidas cautelares personales que se hubieren decretado, de acuerdo a las reglas generales, pero el Ministro de la Corte Suprema tomará las medidas que estimare necesarias para evitar la fuga del imputado.

    Artículo 448.- Audiencia en la extradición pasiva. La audiencia será pública, y a su inicio el representante del Estado requirente dará breve cuenta de los antecedentes en que se funda la petición de extradición. Si fuere el ministerio público, hará saber también los hechos y circunstancias que obraren en beneficio del imputado.
    A continuación se rendirá la prueba testimonial, pericial o documental que las partes hubieren ofrecido.
    Una vez rendida la prueba, si el imputado lo deseare podrá prestar declaración y, de hacerlo, pondrá ser contrainterrogado.
    En caso de que se hubiere rendido prueba o hubiere declarado el imputado, se le concederá la palabra al representante del Estado requirente, para que exponga sus conclusiones.
    Luego, se le concederá la palabra al imputado para que, personalmente o a través de su defensor, efectuare las argumentaciones que estimare procedentes.

    Artículo 449.- Fallo de la extradición pasiva. El tribunal concederá la extradición si estimare comprobada la existencia de las siguientes circunstancias:
    a) La identidad de la persona cuya extradición se solicitare;
    b) Que el delito que se le imputare o aquél por el cual se le hubiere condenado sea de aquellos que autorizan la extradición según los tratados vigentes o, a falta de éstos, en conformidad con los principios de derecho internacional, y
    c) Que de los antecedentes del procedimiento pudiere presumirse que en Chile se deduciría acusación en contra del imputado por los hechos que se le atribuyen.
    La sentencia correspondiente se dictará, por escrito, dentro de quinto día de finalizada la audiencia.

    Artículo 450.- Recursos en contra de la sentencia que falla la petición de extradición. En contra de la sentencia que se pronunciare sobre la extradición procederán el recurso de apelación y el recurso de nulidad, el que sólo podrá fundarse en una o más de las causales previstas en los artículos 373, letra a), y 374. Corresponderá conocer de estos recursos a la Corte Suprema.
    En el evento de interponerse ambos recursos, deberán deducirse en forma conjunta en un mismo escrito, uno en subsidio del otro y dentro del plazo previsto para el recurso de apelación.
    La Corte Suprema conocerá del recurso en conformidad a las reglas generales previstas en este Código para la tramitación de los recursos.
    Artículo 451.- Sentencia que concede la extradición pasiva. Ejecutoriada que fuere la sentencia que concediere la extradición, el Ministro de la Corte Suprema pondrá al sujeto requerido a disposición del Ministerio de Relaciones Exteriores, a fin de que sea entregado al país que la hubiere solicitado.
    Artículo 452.- Sentencia que deniega la extradición pasiva. Si la sentencia denegare la extradición, aun cuando no se encontrare ejecutoriada, el Ministro de la Corte Suprema procederá a decretar el cese de cualquier medida cautelar personal que se hubiere decretado en contra del sujeto cuya extradición se solicitare.
    Ejecutoriada la sentencia que denegare la extradición, el Ministro de la Corte comunicará al Ministerio de Relaciones Exteriores el resultado del procedimiento, incluyendo copia autorizada de la sentencia que en él hubiere recaído.
    Artículo 453.- Desistimiento del Estado requirente. Se sobreseerá definitivamente en cualquier etapa del procedimiento en que el Estado requirente se desistiere de su solicitud.
    Artículo 454.- Extradición pasiva simplificada. Si la persona cuya extradición se requiriere, luego de ser informada acerca de sus derechos a un procedimiento formal de extradición y de la protección que éste le brinda, con asistencia letrada, expresa ante el Ministro de la Corte Suprema que conociere de la causa, su conformidad en ser entregada al Estado solicitante, el Ministro concederá sin más trámite la extradición, procediéndose en este caso en conformidad con el artículo 451.
    Título VII

    Procedimiento para la aplicación exclusiva de
medidas de seguridad

    Párrafo 1º Disposiciones generales
    Artículo 455.- Procedencia de la aplicación de medidas de seguridad. En el proceso penal sólo podrá aplicarse una medida de seguridad al enajenado mental que hubiere realizado un hecho típico y antijurídico y siempre que existieren antecedentes calificados que permitieren presumir que atentará contra sí mismo o contra otras personas.
    Artículo 456.- Supletoriedad de las normas del Libro Segundo para la aplicación de medidas de seguridad. El procedimiento para la aplicación de medidas de seguridad se rige por las reglas contenidas en este Título y en lo que éste no prevea expresamente, por las disposiciones del Libro Segundo, en cuanto no fueren contradictorias.
    Artículo 457.- Clases de medidas de seguridad. Podrán imponerse al enajenado mental, según la gravedad del caso, la internación en un establecimiento psiquiátrico o su custodia y tratamiento.
    En ningún caso la medida de seguridad podrá llevarse a cabo en un establecimiento carcelario. Si la persona se encontrare recluida, será trasladada a una institución especializada para realizar la custodia, tratamiento o la internación. Si no lo hubiere en el lugar, se habilitará un recinto especial en el hospital público más cercano.
    La internación se efectuará en la forma y condiciones que se establecieren en la sentencia que impone la medida. Cuando la sentencia dispusiere la medida de custodia y tratamiento, fijará las condiciones de éstos y se entregará al enajenado mental a su familia, a su guardador, o a alguna institución pública o particular de beneficencia, socorro o caridad.
    Párrafo 2º Sujeto inimputable por enajenación
mental
    Artículo 458.- Imputado enajenado mental. Cuando en el curso del procedimiento aparecieren antecedentes que permitieren presumir la inimputabilidad por enajenación mental del imputado, el ministerio público o juez de garantía, de oficio o a petición de parte, solicitará el informe psiquiátrico correspondiente, explicitando la conducta punible que se investiga en relación a éste. El juez ordenará la suspensión del procedimiento hasta tanto no se remitiere el informe requerido, sin perjuicio de continuarse respecto de los demás coimputados, si los hubiere.
    Artículo 459.- Designación de curador. Existiendo antecedentes acerca de la enajenación mental del imputado, sus derechos serán ejercidos por un curador ad litem designado al efecto.
    Artículo 460.- Actuación del ministerio público. Si el fiscal hallare mérito para sobreseer temporal o definitivamente la causa, efectuará la solicitud respectiva en la oportunidad señalada en el artículo 248, caso en el cual procederá de acuerdo a las reglas generales.
    Con todo, si al concluir su investigación, el fiscal estimare concurrente la causal de extinción de responsabilidad criminal prevista en el artículo 10, número 1°, del Código Penal y, además, considerare aplicable una medida de seguridad, deberá solicitar que se proceda conforme a las reglas previstas en este Título.
    Artículo 461.- Requerimiento de medidas de seguridad. En el caso previsto en el inciso segundo del artículo anterior, el fiscal requerirá la medida de seguridad, mediante solicitud escrita, que deberá contener, en lo pertinente, las menciones exigidas en el escrito de acusación.
    El fiscal no podrá, en caso alguno, solicitar la aplicación del procedimiento abreviado o la suspensión condicional del procedimiento.
    En los casos previstos en este artículo, el querellante podrá acompañar al escrito a que se refiere el artículo 261 los antecedentes que considerare demostrativos de la imputabilidad de la persona requerida.
    Artículo 462.- Resolución del requerimiento. Formulado el requerimiento, corresponderá al juez de garantía declarar que el sujeto requerido se encuentra en la situación prevista en el artículo 10, número 1°, del Código Penal. Si el juez apreciare que los antecedentes no permiten establecer con certeza la inimputabilidad, rechazará el requerimiento.
    Al mismo tiempo, dispondrá que la acusación se formulare por el querellante, siempre que éste se hubiere opuesto al requerimiento del fiscal, para que la sostuviere en lo sucesivo en los mismos términos que este Código establece para el ministerio público. En caso contrario, ordenará al ministerio público la formulación de la acusación conforme al trámite ordinario.
    Los escritos de acusación podrán contener peticiones subsidiarias relativas a la imposición de medidas de seguridad.
    Artículo 463.- Reglas especiales relativas a la aplicación de medidas de seguridad. Cuando se proceda en conformidad a las normas de este Párrafo, se aplicarán las siguientes reglas especiales:
    a) El procedimiento no se podrá seguir conjuntamente contra sujetos enajenados mentales y otros que no lo fueren;
    b) El juicio se realizará a puerta cerrada, sin la presencia del enajenado mental, cuando su estado imposibilite la audiencia, y
    c) La sentencia absolverá si no se constatare la existencia de un hecho típico y antijurídico o la participación del imputado en él, o, en caso contrario, podrá imponer al inimputable una medida de seguridad.
    Artículo 464.- Internación provisional del imputado. Durante el procedimiento el tribunal podrá ordenar, a petición de alguno de los intervinientes, la internación provisional del imputado en un establecimiento asistencial, cuando concurrieren los requisitos señalados en los artículos 140 y 141, y el informe psiquiátrico practicado al imputado señalare que éste sufre una grave alteración o insuficiencia en sus facultades mentales que hicieren temer que atentará contra sí o contra otras personas.
    Se aplicarán, en lo que fueren pertinentes, las normas contenidas en los párrafos 4º, 5º y 6º del Título V del Libro Primero.
    Párrafo 3º Imputado que cae en enajenación durante
el procedimiento
    Artículo 465.- Imputado que cae en enajenación mental. Si, después de iniciado el procedimiento, el imputado cayere en enajenación mental, el juez de garantía decretará, a petición del fiscal o de cualquiera de los intervinientes, previo informe psiquiátrico, el sobreseimiento temporal del procedimiento hasta que desapareciere la incapacidad del imputado o el sobreseimiento definitivo si se tratare de una enajenación mental incurable.
    La regla anterior sólo se aplicará cuando no procediere la terminación del procedimiento por cualquier otra causa.
    Si en el momento de caer en enajenación el imputado se hubiere formalizado la investigación o se hubiere deducido acusación en su contra, y se estimare que corresponde adoptar una medida de seguridad, se aplicará lo dispuesto en el Párrafo 2º de este Título.
    Título VIII
    Ejecución de las sentencias condenatorias y medidas
de seguridad

    Párrafo 1º. Intervinientes
    Artículo 466.- Intervinientes . Durante la ejecución de la pena o de la medida de seguridad, sólo podrán intervenir ante el competente juez de garantía el ministerio público, el imputado, su defensor y el delegado a cargo de la pena sustitutiva de prestación de servicios en beneficio de la comunidad, de libertad vigilada o de libertad vigilada intensiva, según corresponda.
    El condenado o el curador, en su caso, podrán ejercer durante la ejecución de la pena o medida de seguridad todos los derechos y facultades que la normativa penal y penitenciaria le otorgare.
    El Consejo de Defensa del Estado podrá tener la calidad de interviniente para todos los efectos de la ejecución de la pena en su aspecto patrimonial y especialmente respecto del cumplimiento del comiso impuesto en la sentencia, haya o no comparecido en la causa respectiva.
    Sin perjuicio de lo previsto en el inciso precedente y para sus mismos efectos, tratándose de los delitos contemplados en la ley Nº19.913, que crea la Unidad de Análisis Financiero y modifica diversas disposiciones en materia de lavado y blanqueo de activos, y en la ley Nº20.000, que sustituye la ley Nº19.366, que sanciona el tráfico ilícito de estupefacientes y sustancias psicotrópicas, podrán tener, además, la calidad de intervinientes, tanto el Ministerio del Interior y Seguridad Pública como el Servicio Nacional para la Prevención y Rehabilitación del Consumo de Drogas y Alcohol, hayan o no comparecido en la causa respectiva.

    Párrafo 2º Ejecución de las sentencias
    Artículo 467.- Normas aplicables a la ejecución de sentencias penales. La ejecución de las sentencias penales se efectuará de acuerdo con las normas de este Párrafo y con las establecidas por el Código Penal y demás leyes especiales.
    Artículo 468.- Ejecución de la sentencia penal. Las sentencias condenatorias penales no podrán ser cumplidas sino cuando se encontraren ejecutoriadas. Cuando la sentencia se hallare firme, el tribunal decretará una a una todas las diligencias y comunicaciones que se requirieren para dar total cumplimiento al fallo.
    Cuando el condenado debiere cumplir pena privativa de libertad, el tribunal remitirá copia de la sentencia, con el atestado de hallarse firme, al establecimiento penitenciario correspondiente, dando orden de ingreso. Si el condenado estuviere en libertad, el tribunal ordenará inmediatamente su aprehensión y, una vez efectuada, procederá conforme a la regla anterior.
    Si la sentencia hubiere concedido una pena sustitutiva a las penas privativas o restrictivas de libertad consideradas en la ley, remitirá copia de la misma a la institución encargada de su ejecución.
    Asimismo, ordenará y controlará el efectivo cumplimiento de las multas y comisos impuestos en la sentencia, ejecutará las cauciones en conformidad con el artículo 147, cuando procediere, y dirigirá las comunicaciones que correspondiere a los organismos públicos o autoridades que deban intervenir en la ejecución de lo resuelto.

    Artículo 468 bis.- Ejecución del comiso de ganancias. Toda sentencia que imponga el comiso de las ganancias provenientes del delito será ejecutada como decisión civil dictada por un tribunal con competencia en lo penal.
    Si los bienes decomisados son dinero o derechos a sumas de dinero, se los transferirá al Fisco. Los fondos obtenidos mediante la realización de los bienes decomisados también serán transferidos al Fisco.
    El comiso de inmuebles o de bienes de propiedad registral conlleva la facultad de realizar aquellas inscripciones necesarias para ejecutar eficazmente el bien decomisado.
    El Conservador de Bienes Raíces respectivo, efectuadas las cancelaciones e inscripciones que procedan, deberá remitir copia de dichas inscripciones al tribunal que decretó el comiso, el que deberá oficiar a la Dirección General del Crédito Prendario y acompañar copia de las nuevas inscripciones de propiedad a nombre del Fisco de Chile y copia autorizada de la sentencia para que proceda a rematarlo en subasta pública.
    Los notarios, archiveros, conservadores de bienes raíces, el Servicio de Registro Civil e Identificación y demás organismos, autoridades y empleados públicos deberán realizar las actuaciones y diligencias y otorgar las copias de los instrumentos que les sean solicitados para efectuar la subasta o destrucción de las especies, según corresponda, en forma gratuita y exentas de toda clase de derechos, tasas e impuestos.
    Toda actuación o diligencia previa a la subasta pública que deba efectuar la Dirección General del Crédito Prendario con el objeto de que los bienes queden en condiciones de ser subastados, se efectuará con auxilio de la fuerza pública a solicitud de la referida institución.
    Lo dispuesto en el presente artículo será aplicable también a la ejecución de todo comiso impuesto sin condena previa.

    Artículo 469.- Destino de las especies decomisadas. Fuera de los casos previstos en el artículo precedente, los dineros y otros valores decomisados se destinarán a la Corporación Administrativa del Poder Judicial.
    Si el tribunal estimare necesario ordenar la destrucción de las especies, se llevará a cabo bajo la responsabilidad del administrador del tribunal, salvo que se le encomendare a otro organismo público. En todo caso, se registrará la ejecución de la diligencia.
    Las demás especies decomisadas se pondrán a disposición de la Dirección General del Crédito Prendario para que proceda a su enajenación en subasta pública, o a destruirlas si carecieren de valor. El producto de la enajenación tendrá el mismo destino que se señala en el inciso primero.
    En los casos de los artículos 367 quáter, incisos primero y segundo, 367 quinquies y 367 septies del Código Penal, el tribunal destinará los instrumentos tecnológicos decomisados, tales como computadores, reproductores de imágenes o sonidos y otros similares, al Servicio Nacional de Menores o a los departamentos especializados en la materia de los organismos policiales que correspondan.

    Artículo 470.- Especies retenidas y no decomisadas. Transcurridos a lo menos seis meses desde la fecha de la resolución firme que hubiere puesto término al juicio, sin que hubieren sido reclamadas por su legítimo titular las cosas corporales muebles retenidas y no decomisadas que se encontraren a disposición del tribunal, deberá procederse de acuerdo a lo dispuesto en los incisos siguientes.
    Si se tratare de especies, el administrador del tribunal, previo acuerdo del comité de jueces, las venderá en pública subasta. Los remates se podrán efectuar dos veces al año.
    El producto de los remates, así como los dineros o valores retenidos y no decomisados, se destinarán a la Corporación Administrativa del Poder Judicial.
    Si se hubiere decretado el sobreseimiento temporal o la suspensión condicional del procedimiento, el plazo señalado en el inciso primero será de un año.
    Las especies que se encontraren bajo la custodia o a disposición del Ministerio Público, transcurridos a lo menos seis meses desde la fecha en que se dictare alguna de las resoluciones o decisiones a que se refieren los artículos 167, 168, 170 y 248 letra c), de este Código, serán remitidas a la Dirección General del Crédito Prendario, para que proceda de conformidad a lo dispuesto en el inciso tercero del artículo anterior.
    Lo dispuesto en los incisos anteriores no tendrá aplicación tratándose de especies de carácter ilícito. En tales casos, el fiscal solicitará al juez que le autorice proceder a su destrucción.

    Artículo 471.- Control sobre las especies puestas a disposición del tribunal. En el mes de junio de cada año, los tribunales con competencia en materia criminal presentarán a la respectiva Corte de Apelaciones un informe detallado sobre el destino dado a las especies que hubieren sido puestas a disposición del tribunal.
    Artículo 472.- Ejecución civil. En el cumplimiento de la decisión civil de la sentencia, regirán las disposiciones sobre ejecución de las resoluciones judiciales que establece el Código de Procedimiento Civil.
    Párrafo 3º. Revisión de las sentencias firmes
    Artículo 473.- Procedencia de la revisión. La Corte Suprema podrá rever extraordinariamente las sentencias firmes en que se hubiere condenado a alguien por un crimen o simple delito, para anularlas, en los siguientes casos:
    a) Cuando, en virtud de sentencias contradictorias, estuvieren sufriendo condena dos o más personas por un mismo delito que no hubiere podido ser cometido más que por una sola;
    b) Cuando alguno estuviere sufriendo condena como autor, cómplice o encubridor del homicidio de una persona cuya existencia se comprobare después de la condena;
    c) Cuando alguno estuviere sufriendo condena en virtud de sentencia fundada en un documento o en el testimonio de una o más personas, siempre que dicho documento o dicho testimonio hubiere sido declarado falso por sentencia firme en causa criminal;
    d) Cuando, con posterioridad a la sentencia condenatoria, ocurriere o se descubriere algún hecho o apareciere algún documento desconocido durante el proceso, que fuere de tal naturaleza que bastare para establecer la inocencia del condenado, y
    e) Cuando la sentencia condenatoria hubiere sido pronunciada a consecuencia de prevaricación o cohecho del juez que la hubiere dictado o de uno o más de los jueces que hubieren concurrido a su dictación, cuya existencia hubiere sido declarada por sentencia judicial firme.
    Artículo 474.- Plazo y titulares de la solicitud de revisión. La revisión de la sentencia firme podrá ser pedida, en cualquier tiempo, por el ministerio público, por el condenado, o su cónyuge o conviviente civil, ascendientes, descendientes o hermanos de éste. Asimismo, podrá interponer tal solicitud quien hubiere cumplido su condena o sus herederos, cuando el condenado hubiere muerto y se tratare de rehabilitar su memoria.

    Artículo 475.- Formalidades de la solicitud de revisión. La solicitud se presentará ante la secretaría de la Corte Suprema; deberá expresar con precisión su fundamento legal y acompañar copia fiel de la sentencia cuya anulación se solicitare y los documentos que comprobaren los hechos en que se sustenta.
    Si la causal alegada fuere la de la letra b) del artículo 473, la solicitud deberá indicar los medios con que se intentare probar que la persona víctima del pretendido homicidio hubiere vivido después de la fecha en que la sentencia la supone fallecida; y si fuere la de la letra d), indicará el hecho o el documento desconocido durante el proceso, expresará los medios con que se pretendiere acreditar el hecho y se acompañará, en su caso, el documento o, si no fuere posible, se manifestará al menos su naturaleza y el lugar y archivo en que se encuentra.
    La solicitud que no se conformare a estas prescripciones o que adolezca de manifiesta falta de fundamento será rechazada de plano, decisión que deberá tomarse por la unanimidad del tribunal.
    Apareciendo interpuesta en forma legal, se dará traslado de la petición al fiscal, o al condenado, si el recurrente fuere el ministerio público; en seguida, se mandará traer la causa en relación, y, vista en la forma ordinaria, se fallará sin más trámite.
    Artículo 476.- Improcedencia de la prueba testimonial. No podrá probarse por testigos los hechos en que se funda la solicitud de revisión.
    Artículo 477.- Efectos de la interposición de la solicitud de revisión. La solicitud de revisión no suspenderá el cumplimiento de la sentencia que se intentare anular.
    Con todo, si el tribunal lo estimare conveniente, en cualquier momento del trámite podrá suspender la ejecución de la sentencia recurrida y aplicar, si correspondiere, alguna de las medidas cautelares personales a que se refiere el Párrafo 6º del Título V del Libro Primero.

    Artículo 478.- Decisión del tribunal. La resolución de la Corte Suprema que acogiere la solicitud de revisión declarará la nulidad de la sentencia.
    Si de los antecedentes resultare fehacientemente acreditada la inocencia del condenado, el tribunal además dictará, acto seguido y sin nueva vista pero separadamente, la sentencia de reemplazo que corresponda.
    Asimismo, cuando hubiere mérito para ello y así lo hubiere recabado quien hubiere solicitado la revisión, la Corte podrá pronunciarse de inmediato sobre la procedencia de la indemnización a que se refiere el artículo 19, Nº 7, letra i), de la Constitución Política.
    Artículo 479.- Efectos de la sentencia. Si la sentencia de la Corte Suprema o, en caso de que hubiere nuevo juicio, la que pronunciare el tribunal que conociere de él, comprobare la completa inocencia del condenado por la sentencia anulada, éste podrá exigir que dicha sentencia se publique en el Diario Oficial a costa del Fisco y que se devuelvan por quien las hubiere percibido las sumas que hubiere pagado en razón de multas, costas e indemnización de perjuicios en cumplimiento de la sentencia anulada.
    El cumplimiento del fallo en lo atinente a las acciones civiles que emanan de él será conocido por el juez de letras en lo civil que corresponda, en juicio sumario.
    Los mismos derechos corresponderán a los herederos del condenado que hubiere fallecido.
    Además, la sentencia ordenará, según el caso, la libertad del imputado y la cesación de la inhabilitación.
    Artículo 480.- Información de la revisión en un nuevo juicio. Si el ministerio público resolviere formalizar investigación por los mismos hechos sobre los cuales recayó la sentencia anulada, el fiscal acompañará en la audiencia respectiva copia fiel del fallo que acogió la revisión solicitada.
    Párrafo 4º Ejecución de medidas de seguridad
    Artículo 481.- Duración y control de las medidas de seguridad. Las medidas de seguridad impuestas al enajenado mental sólo podrán durar mientras subsistieren las condiciones que las hubieren hecho necesarias, y en ningún caso podrán extenderse más allá de la sanción restrictiva o privativa de libertad que hubiere podido imponérsele o del tiempo que correspondiere a la pena mínima probable, el que será señalado por el tribunal en su fallo.
    Se entiende por pena mínima probable, para estos efectos, el tiempo mínimo de privación o restricción de libertad que la ley prescribiere para el delito o delitos por los cuales se hubiere dirigido el procedimiento en contra del sujeto enajenado mental, formalizado la investigación o acusado, según correspondiere.
    La persona o institución que tuviere a su cargo al enajenado mental deberá informar semestralmente sobre la evolución de su condición al ministerio público y a su curador o a sus familiares, en el orden de prelación mencionado en el artículo 108.
    El ministerio público, el curador o familiar respectivo podrá solicitar al juez de garantía la suspensión de la medida o la modificación de las condiciones de la misma, cuando el caso lo aconsejare.
    Sin perjuicio de lo anterior, el ministerio público deberá inspeccionar, cada seis meses, los establecimientos psiquiátricos o instituciones donde se encontraren internados o se hallaren cumpliendo un tratamiento enajenados mentales, en virtud de las medidas de seguridad que se les hubieren impuesto, e informará del resultado al juez de garantía, solicitando la adopción de las medidas que fueren necesarias para poner remedio a todo error, abuso o deficiencia que observare en la ejecución de la medida de seguridad.
    El juez de garantía, con el solo mérito de los antecedentes que se le proporcionaren, adoptará de inmediato las providencias que fueren urgentes, y citará a una audiencia al ministerio público y al representante legal del enajenado mental, sin perjuicio de recabar cualquier informe que estimare necesario, para decidir la continuación o cesación de la medida, o la modificación de las condiciones de aquélla o del establecimiento en el cual se llevare a efecto.
    Artículo 482.- Condenado que cae en enajenación mental. Si después de dictada la sentencia, el condenado cayere en enajenación mental, el tribunal, oyendo al fiscal y al defensor, dictará una resolución fundada declarando que no se deberá cumplir la sanción restrictiva o privativa de libertad y dispondrá, según el caso, la medida de seguridad que correspondiere. El tribunal velará por el inmediato cumplimiento de su resolución. En lo demás, regirán las disposiciones de este Párrafo.
    Título Final
    Entrada en vigencia de este Código
    Artículo 483.- Aplicación de las disposiciones del Código. Las disposiciones de este Código sólo se aplicarán a los hechos acaecidos con posterioridad a su entrada en vigencia.
    Artículo 484.- Entrada en vigencia respecto de hechos acaecidos en el territorio nacional. Este Código comenzará a regir, para las distintas Regiones del país, al término de los plazos que establece el artículo 4º transitorio de la Ley Nº 19.640, Orgánica Constitucional del Ministerio Público.
    En consecuencia, regirá para las regiones de Coquimbo y de la Araucanía, desde el 16 de diciembre de 2000; para las regiones de Antofagasta, Atacama y del Maule, desde el 16 de octubre de 2001; para las regiones de Tarapacá, de Aisén del General Carlos Ibáñez del Campo y de Magallanes y de la Antártica Chilena, desde el 16 de diciembre de 2002; para las regiones de Valparaíso, del Libertador General Bernardo O'Higgins, del Bío Bío y de Los Lagos, desde el 16 de diciembre de 2003, y para la Región Metropolitana de Santiago, desde el 16 de junio de 2005.

    Artículo 485. Entrada en vigencia respecto de hechos acaecidos en el extranjero. Este Código se aplicará a los hechos que acaecieren en el extranjero con posterioridad a su entrada en vigencia en la Región Metropolitana de Santiago y fueren de competencia de los tribunales nacionales. Asimismo, se aplicará a las solicitudes de asistencia de autoridades competentes de país extranjero que digan relación con hechos ocurridos con posterioridad al 16 de diciembre de 2000.

    A partir del 16 de junio de 2005, también se aplicará a las solicitudes de extradición pasiva y detención previa a las mismas que reciba la Corte Suprema, que versen sobre hechos ocurridos en el extranjero con posterioridad a la entrada en vigencia de este Código en la Región Metropolitana de Santiago. En consecuencia, los Ministros de esa Corte a quienes, en virtud del número 3° del artículo 52 del Código Orgánico de Tribunales, correspondiere conocer las extradiciones pasivas que versen sobre hechos acaecidos con anterioridad a dicha entrada en vigencia, continuarán aplicando el procedimiento establecido en el Código de Procedimiento Penal.

    Artículo transitorio.- Reglas para la aplicación de las penas por tribunales con competencia en lo criminal sujetos a distintos procedimientos. Si una persona hubiere cometido distintos hechos, debido a los cuales fuere juzgada por un juzgado de letras del crimen o con competencia en lo criminal, con sujeción al Código de Procedimiento Penal, y también lo fuere por un juzgado de garantía o un tribunal oral en lo penal conforme a este Código, en el pronunciamiento de las sentencias condenatorias que se dictaren con posterioridad a la primera se estará a lo previsto en el artículo 164 del Código Orgánico de Tribunales.''.

    Y por cuanto he tenido a bien aprobarlo y sancionarlo; por tanto promúlguese y llévese a efecto como Ley de la República.

    Santiago, 29 de septiembre de 2000.- RICARDO LAGOS ESCOBAR, Presidente de la República.- José Antonio Gómez Urrutia,  Ministro de Justicia.

    Lo que transcribo a Ud. para su conocimiento.- Saluda atentamente a Ud., Jaime Arellano Quintana, Subsecretario de Justicia.